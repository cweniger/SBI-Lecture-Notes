\documentclass[a4paper, 12pt, titlepage=false]{scrreprt}
% ============================================================================
% ENCODING AND FONTS
% ============================================================================
\usepackage[utf8]{inputenc}
\usepackage{libertinus}

% ============================================================================
% GRAPHICS AND FIGURES
% ============================================================================
\usepackage{graphicx}

% ============================================================================
% MATHEMATICS
% ============================================================================
\usepackage{mathtools}
\usepackage{amsmath}
\usepackage{amsthm}
\usepackage{amsfonts}
\usepackage{bbold}
\usepackage{MnSymbol}

% Figure numbering (requires amsmath)
\numberwithin{figure}{section}

% Math operators
\DeclareMathOperator*{\argmax}{arg\,max}
\DeclareMathOperator*{\argmin}{arg\,min}

% Theorem environments
\newtheorem{theorem}{Theorem}

% ============================================================================
% TABLES AND FORMATTING
% ============================================================================
\usepackage{array}
\usepackage{booktabs}
\usepackage{enumitem}
\usepackage{rotating}
\usepackage{cancel}

% ============================================================================
% PAGE LAYOUT
% ============================================================================
\usepackage{geometry}
\usepackage{marginnote}

% ============================================================================
% SECTION NUMBERING
% ============================================================================
\setcounter{secnumdepth}{3}
\setcounter{tocdepth}{3}

% ============================================================================
% COLORS AND BOXES
% ============================================================================
\usepackage[most]{tcolorbox}

\tcbset{
  inlinebox/.style={
    colback=gray!3!white,
    colframe=gray!70!black,
    boxrule=0.4pt,
    arc=2mm,
    left=4pt,
    right=4pt,
    top=4pt,
    bottom=4pt,
    enhanced,
    boxsep=2pt
  }
}

% ============================================================================
% ALGORITHMS
% ============================================================================
\usepackage{algorithm}
\usepackage{algpseudocode}

% ============================================================================
% ICONS AND SYMBOLS
% ============================================================================
\usepackage{fontawesome}

% ============================================================================
% HYPERLINKS AND REFERENCES
% ============================================================================
\usepackage[
    colorlinks=true,
    linkcolor=blue!50!black,
    citecolor=blue!50!black,
    urlcolor=blue!70!black,
]{hyperref}

% ============================================================================
% BIBLIOGRAPHY
% ============================================================================
\usepackage[authoryear]{natbib}
\setcitestyle{round}

% ============================================================================
% CUSTOM COMMANDS - TEXT
% ============================================================================
\newcommand{\fex}{\textit{e.g.,}}
\newcommand{\cw}[1]{{{\scriptsize\color{red} (CW: #1)}}}

% ============================================================================
% CUSTOM COMMANDS - VECTORS AND MATRICES
% ============================================================================
\newcommand{\bx}{\mathbf{x}}
\newcommand{\bxobs}{{\mathbf{x}_\text{obs}}}
\newcommand{\bxnew}{{\mathbf{x}_\text{new}}}
\newcommand{\by}{\mathbf{y}}
\newcommand{\bz}{\mathbf{z}}
\newcommand{\bs}{\mathbf{s}}
\newcommand{\bt}{\mathbf{t}}
\newcommand{\bT}{\mathbf{T}}
\newcommand{\bu}{\mathbf{u}}
\newcommand{\bv}{\mathbf{v}}
\newcommand{\bw}{\mathbf{w}}
\newcommand{\bn}{\boldsymbol{n}}

% ============================================================================
% CUSTOM COMMANDS - GREEK LETTERS
% ============================================================================
\newcommand{\btheta}{\boldsymbol{\theta}}
\newcommand{\boldeta}{\boldsymbol{\eta}}
\newcommand{\bepsilon}{\boldsymbol{\epsilon}}
\newcommand{\bmu}{\boldsymbol{\mu}}
\newcommand{\bphi}{\boldsymbol{\phi}}
\newcommand{\bvarphi}{\boldsymbol{\varphi}}
\newcommand{\bpsi}{\boldsymbol{\psi}}
\newcommand{\balpha}{\boldsymbol{\alpha}}
\newcommand{\blambda}{\boldsymbol{\lambda}}
\newcommand{\bgamma}{\boldsymbol{\gamma}}
\newcommand{\bPhi}{{\boldsymbol{\Phi}}}

% ============================================================================
% CUSTOM COMMANDS - SPECIAL SYMBOLS
% ============================================================================
\newcommand{\bbE}{{\mathbb{E}}}
\newcommand{\bbR}{{\mathbb{R}}}
\newcommand{\iidsim}{\overset{\text{i.i.d.}}{\sim}}

% ============================================================================
% DOCUMENT METADATA
% ============================================================================
\title{TASI Lectures on Structured Reasoning for Simulation-Based Inference\footnote{Extended version of TASI lectures from May 2024.}}
\author{Christoph Weniger (University of Amsterdam)}
\date{\today}


\begin{document}

\maketitle

\begin{abstract}
These lecture notes provide a structured overview of simulation-based inference (SBI), with a focus on neural methods for posterior, likelihood, and ratio estimation. Particular attention is given to the role of data summaries, inference diagnostics, and validation strategies. A key conceptual contribution is the introduction of a taxonomy of epistemic uncertainties—lossy compression, inference approximation, and model misspecification—that helps clarify failure modes and guide method development. Core methods such as NPE, NLE, and NRE are introduced from both algorithmic and information-theoretic perspectives. The material emphasizes practical reasoning over formal generality, aiming to equip researchers with a clear framework for building and troubleshooting SBI pipelines in complex scientific applications.
\end{abstract}

\chapter*{Introduction and Preface}

%\section*{Introduction and Preface}

\emph{These lecture notes grew from my experience of repeatedly hitting the wall with traditional and modern likelihood-based inference techniques, and from a conviction that simulation-based inference offers a path to solve some of the most central scientific analysis challenges we face today. The prospect of unearthing knowledge from complex observations, of faithfully connecting theory with data in ways previously beyond reach, of unraveling the un-unravelable, feels like a form of scientific magic. We are still far from realizing SBI's full potential. What follows is an attempt to equip students entering the field with the theoretical scaffolding to reason systematically about this rapidly evolving frontier, bridging the machine learning literature with the physical sciences.}


\subsection*{Motivation}

The interpretation of a wide range of observations—from \textcolor{red}{TODO-ADD-APPLICATIONS (e.g., gravitational wave signals, particle collision events, neural population recordings, epidemic trajectories, climate patterns)}—relies on increasingly complex simulations. These simulators encode our understanding of physical, biological, or observational processes, but traditional inference methods often hit computational barriers when confronted with intractable likelihoods due to complex marginalization over latent variables, expensive, slow, or inaccurate simulation models, and high-dimensional parameter and data spaces that make MCMC prohibitively costly.

Simulation-based inference (SBI) offers a promising alternative: by training neural networks on simulated data, we can identify relevant patterns and learn to perform statistical inference directly from simulations, bypassing the need for explicit likelihood evaluation and integration. SBI is a quickly developing research field with many early successes \textcolor{red}{TODO-ADD-SPECIFIC-EXAMPLES}. It is likely that SBI will play a significant role in a wide range of analysis tasks across the sciences in the years to come.

\subsection*{Scope and Approach}

These notes are not a practical quick-start tutorial for implementing SBI pipelines.\footnote{Excellent practical introductions can be found in \textcolor{red}{TODO-ADD-TUTORIAL-RESOURCES (e.g., the \texttt{sbi} package documentation, \ldots)}} Instead, they aim to help students develop the \textbf{theoretical scaffolding} necessary for \textbf{structured reasoning} about SBI opportunities and pitfalls. 

We cover foundational aspects of classical simulation-based inference and ABC; data summaries and information-theoretic principles; core neural SBI methods (NPE, NLE, NRE); an extensive treatment of diagnostic and validation methods; and advanced topics including inference under model misspecification.

The research field is new and somewhat scattered across different communities. We attempt to provide conceptual structure by carefully exposing connections and differences between methods. Key unifying perspectives introduced in these notes include a taxonomy of epistemic uncertainties (Type A: model misspecification; Type B: lossy compression; Type C: inexact inference) that clarifies failure modes and guides diagnostic strategy; a generalized rank diagnostics framework that reveals the common structure underlying seemingly disparate validation approaches; and an information-theoretic approach to robust summary learning that formalizes invariance principles for handling simulator uncertainty. These represent both pedagogical synthesis and new conceptual contributions aimed at bringing coherence to the field.

\textbf{Assumed background:} Familiarity with Bayesian inference, basic probability theory, and standard machine learning concepts (neural networks, gradient descent, optimization). We introduce information-theoretic concepts as needed.

\subsection*{Structure}

\begin{description}
\item[Section 1 (Foundations)] establishes the conceptual groundwork, contrasting simulation-based and likelihood-based inference, introducing ABC, and developing information-theoretic perspectives on data compression and summary statistics.

\item[Section 2 (Neural Methods)] presents the main algorithmic approaches—NPE, NLE, and NRE—emphasizing their information-theoretic interpretations, architectural requirements, and practical trade-offs.

\item[Section 3 (Diagnostics)] provides an extensive treatment of validation methods organized around the taxonomy of epistemic uncertainties. We discuss reference posterior comparisons, forward-backward diagnostics (SBC, coverage tests, C2ST), model-based rank diagnostics, and model misspecification tests.

\item[Section 4 (Advanced Topics)] explores robust inference under simulator uncertainty, formalizing asymptotic notions of robustness and presenting strategies for learning invariant summaries.

\item[Section 5 (Concluding Remarks)] offers perspective on the future of the field and additional resources.
\end{description}

\subsection*{Acknowledgments}

An early version of these lectures was delivered at the Theoretical Advanced Study Institute (TASI) in May 2024. I thank the organizers for the invitation and the students for their engagement, excellent questions and feedback throughout the school.

I am furthermore grateful to 
James Alvey,
Noemi Anau Montel,
Patrick Forre,
Guillermo Franco Abellan,
Mathis Gerdes,
Will Handley,
Konstantin Karchev,
Huifang Lyu,
Benjamin Miller,
Joy Sanghavi,
Oleg Savchenko,
Ayman Stitou,
Roberto Trotta,
\cw{TODO: Check list, mention family and GRAPPA}
for numerous discussions and collaborations that have shaped my thinking about simulation-based inference over the years.

These notes are pedagogical rather than comprehensive. References are selective, chosen to illustrate concepts or provide entry points to the literature, rather than to exhaustively survey the field. I apologize to the many researchers whose important contributions are not cited—this reflects constraints of scope rather than judgments of significance. The field is evolving rapidly, and these notes represent a snapshot of methods and perspectives as of 2024--2025. We plan to extend and update them as the field develops.

\vfill
{\subsection*{Note on AI Assistance}

Large language models (in particular Claude by Anthropic and ChatGPT by OpenAI) played a significant role in developing these lecture notes through a human-AI collaborative writing process. I used LLMs as tools for structuring material, refining exposition, generating example codes and figures, condensing text, identifying optimal formulations, LaTeX formatting, and as a sparring partner for clarifying and questioning concepts.  All scientific perspectives, conceptual frameworks, synthesis decisions and pedagogical choices reflect my own understanding and judgment---both about the science and about what approaches will best serve our research field moving forward. 
}


\tableofcontents

\chapter{Foundations of Simulation-Based Inference}
\label{chap:found}

\begin{quotation}
    \textit{``[...] one of the great scientific advantages of simulation analysis of Bayesian methods is the freedom it gives the researcher to formulate appropriate models rather than be overly interested in analytically neat but scientifically inappropriate models.''}

%    \hfill Gelman and Rubin, 1996 \cw{TODO: Check}
    \hfill --- \cite{gelman_markov_1996}\footnote{In the quote from \cite{gelman_markov_1996}, `simulation analysis' actually refers to likelihood-based inference such as MCMC, which is contrasted with exact symbolic methods.  It is telling that the same statement can be equally well used to contrast simulation-based and likelihood-based inference methods, the latter of which commonly rests on the availability of `analytically neat' likelihood functions.}
\end{quotation}


\section{Classical Inference and the Case for Simulation-Based Methods}
\label{sec:found:classical}

\begin{figure}[ht]
    \centering
    \includegraphics[width=0.95\linewidth]{figures/Fig1.drawio.pdf}
    \caption{Likelihood-based inference algorithm have access to the prior and the evaluated likelihood, which is a single scalar that quantifies closeness to the observation $\mathbf x_o$.  Simulation-based inference algorithms have access both to the observation $\mathbf x_o$ as well as simulation results $\mathbf x$, and can determine their own optimal distance measures.}
\label{fig:SBI_vs_LBI}
\end{figure}

We start with a compact introduction to Bayesian inference\footnote{\cw{TODO Discuss frequentism}}, focusing on essential elements relevant for the lecture notes. For comprehensive excellent historical and more recent introductions we refer to ~\cite{raiffa_applied_1961, rubin_bayesianly_1984, mackay_information_2003, gelman_bayesian_2013}.

\smallskip

Consider a probabilistic model for observed data \( \bx \), described by the conditional distribution \( p(\bx \mid \btheta) \), where \( \btheta \) denotes a set of model parameters. Both \( \bx \) and \( \btheta \) may represent complex or structured quantities—for instance, \( \bx \) might consist of gene expression levels in biology, spectral lines in chemistry, time series from astronomical observations, or raw measurements from a physics experiment. Correspondingly, \( \btheta \) could encode molecular concentrations, reaction rates, cosmological parameters, or theory coefficients.
%
Given access to the likelihood \( p(\bx \mid \btheta) \), Bayes' theorem~\citep[\fex][]{gelman_bayesian_2013} allows direct computation of the posterior:
\begin{equation}
    p(\btheta \mid \bx) = \frac{p(\bx \mid \btheta) \, p(\btheta)}{p(\bx)}\;.
    \label{eqn:Bayes_theorem}
\end{equation}
Here, $p(\btheta)$ is the prior parameter distribution, and $p(\bx)$ a normalizing constant usually referred to as Bayesian evidence or marginal likelihood. The posterior distribution embodies everything we know about the parameter $\btheta$ after seeing the data $\bx$ \citep[\fex][]{mackay_information_2003}.

Despite the simple analytic form of Bayes' theorem, accessing the posterior usually brings either analytically or computationally challenging problems—or both. This is related to the fact that the likelihood function typically involves a large number of latent (unobserved) parameters, which might include measurement noise, data masks, uncertainties of the experimental apparatus, population parameters, etc. These parameters, here collectively called $\boldeta$, lead to a likelihood function with hierarchical form\footnote{
Note that the parameters $\boldeta$ might themselves depend on $\btheta$, otherwise we would have $p(\boldeta \mid \btheta) = p(\boldeta)$. 
}
\begin{equation}
    p(\bx \mid \btheta) = \int d\boldeta\, p(\bx \mid \btheta, \boldeta) p(\boldeta \mid \btheta)\;.
    \label{eqn:likelihood_with_nuisance}
\end{equation}
%
Solving the integration problem in Eq.~\eqref{eqn:likelihood_with_nuisance}---and related integration problems connected to the Bayesian evidence $p(\bx)$---represents one of the key aspects that makes Bayesian (but also Frequentist) inference challenging in practice.



\subsection{About Explicit and Implicit Models}
\label{sec:found:classical:implicit}

There exist two complementary ways of accessing the likelihood model—depending on whether the likelihood is defined \emph{explicitly}, as a function that can be evaluated, or \emph{implicitly}, as a distribution that can be sampled from (see \cite{diggle_monte_1984} for an early discussion in the context of Monte Carlo methods, and \cite{mohamed_learning_2017} for a discussion in the context of generative models).  

It is useful to highlight the distinct computational challenges for explicit and implicit likelihood models, which can be related to how those models handle the process of marginalizing (`integrating out') unobserved quantities. From a computational perspective, this distinction manifests as follows:

\begin{itemize}
    \item \textbf{Explicit model access (evaluation).} Explicit likelihood models provide a deterministic routine, \textsc{LogLike}(\( \bx, \btheta \)), which evaluates the log-likelihood \( \log p(\bx \mid \btheta) \) for given inputs. Optionally, a related routine \textsc{Score}(\( \bx, \btheta \)) may return the gradient \( \nabla_{\btheta} \log p(\bx \mid \btheta) \), and possibly higher-order derivatives.
    
    \textit{Example:} Evaluating the hierarchical likelihood $p(\bx
    \mid \btheta)$ in Eq.~\eqref{eqn:likelihood_with_nuisance} for a given value
    of $\btheta$ requires integrating (marginalizing) over $\boldeta$, either analytically or with Monte Carlo methods.  Computational costs scale with the complexity of integrating over latent space parameters $\boldeta$.

    \item \textbf{Implicit model access (sampling).} Implicit likelihood models take the form of a stochastic simulator, \textsc{Sim}(\( \btheta \)), which generates random outputs \( \bx \sim p(\bx \mid \btheta) \). This corresponds to forward simulation. The simulator is stochastic in the sense that repeated calls with the same \( \btheta \) produce different \( \bx \), thereby implicitly defining the likelihood.

    \textit{Example:} For a given value of $\btheta$, sampling from the latent parameter prior and subsequently from the data likelihood generates a sample from $p(\bx \mid \btheta)$,
    \[
    \boldeta \sim p(\boldeta \mid \btheta), \quad \bx \sim p(\bx \mid \btheta, \boldeta)
    \quad \text{which gives} \quad
    \bx \sim p(\bx \mid \btheta).
    \]
    No explicit computation of the likelihood $p(\bx \mid \btheta)$ is needed.  Computational costs are \emph{independent} of the complexity of integrating over the latent space.
\end{itemize}
%
While these modes describe access to the \emph{same} underlying distribution $p(\bx \mid \btheta)$, only one of them may be tractable (\textit{i.e.}, solvable or manageable with reasonable computational effort) in a given application. In many scientific contexts---particularly in physics, biology, and astronomy---\textsc{Sim}(\( \btheta \)) is available as a domain-specific simulator, while \textsc{LogLike} and \textsc{Score} are intractable, \fex\ due to complex integration tasks related to latent variables. Conversely, in analytically tractable models---such as Gaussian likelihoods for aggregated measurements or Poisson models for counting data---\textsc{LogLike} and \textsc{Score} may be explicitly available, while constructing a generative simulator \textsc{Sim}(\( \btheta \)) that faithfully samples \( \bx \sim p(\bx \mid \btheta) \) may be challenging due to the high-dimensionality of data $\bx$ or latent variables $\boldeta$ or sharp prior constraints.

\medskip

An important subtlety is that a single explicit likelihood model may correspond to many different implicit simulation tasks with vastly different levels of difficulty. Even when the posterior \( p(\btheta \mid \bx) \) is well-defined and tractable for likelihood-based methods, the computational challenge of simulation-based inference depends critically on how the data \( \bx \) is generated, represented, and processed by the simulator. A seemingly simple explicit likelihood—such as a low-dimensional Gaussian—may correspond to either a trivial or highly challenging simulation-based inference problem, depending on the structure and dimensionality of the simulated observations. For instance, inferring a scalar parameter from its noisy measurement is straightforward, while inferring the same parameter from high-dimensional images generated by a complex forward model presents entirely different computational challenges. As such, evaluating the difficulty of an inference task—and comparing the performance of explicit versus implicit approaches—requires careful attention to the simulation model and data representation, not just the underlying likelihood function.

\subsection{Inference Algorithms and Accessible Models}
\label{sec:found:classical:accessible}

A wide range of techniques have been developed to address the computational challenges of explicit and implicit likelihood models in the context of Bayesian inference.  In broad strokes, these can be separated into symbolic, likelihood-based methods (for explicit models) and simulation-based techniques (for implicit models), as shown in Fig.~\ref{fig:SBI_vs_LBI}.

\medskip

Symbolic Bayesian inference techniques, using analytical tools like conjugate priors and related methods, were popularized in economics and decision theory in the 1950s and 1960s (see \cite{raiffa_applied_1961} for an in-depth account). The applicability of these methods typically requires that the likelihood and prior follow restricted classes of analytical functions (exponential families, conjugate priors, etc.) that are amenable to analytic treatment. The result is exact and explicit posterior functions, $p(\btheta \mid \bx)$.

In the 1990s, Bayesian inference was revolutionized by the adoption of MCMC methods to sample from marginal distributions including posteriors, introduced in a seminal paper by \cite{gelfand_sampling-based_1990}. 
These techniques allowed to handle explicit likelihood models that do not fall into the narrow categories of symbolic inference. Among the most widely adopted MCMC algorithms are the Metropolis–Hastings algorithm \citep{metropolis_equation_1953, hastings_monte_1970}, which was initially developed in the context of statistical physics, and the Hamiltonian Monte Carlo algorithm \citep{duane_hybrid_1987, neal_mcmc_2011}, originally introduced for applications in particle physics (specifically lattice QCD). 
During this decade, variational inference (VI) methods also emerged, which find explicit approximations to the posterior~\citep[\fex][]{jordan_introduction_1999}.
%and later also complementary approaches such as nested sampling \citep{skilling_nested_2006}, were developed.  
Most of these techniques provide implicit access (in the form of samples) to the exact posterior, $\btheta \sim p(\btheta \mid \bx)$.

\medskip

The question then becomes: how can we perform Bayesian inference when we can simulate but cannot evaluate the likelihood?  Since the 1980s (and formally the early 2000s), inference methods based on implicit models started receiving attention and were formally developed.  These methods are the main focus of these lecture notes, and typically provide implicit (in terms of samples) or explicit (in terms of density evaluations) access to \emph{approximations} to the posterior, $\btheta \sim q_\phi(\btheta \mid \bx) \simeq p(\btheta \mid \bx)$.  Those methods provide the greatest flexibility in terms of model definition.  However, due to the approximate nature of the inference tasks, they also require a careful understanding of failure modes and associated challenges.


\cw{TODO: Discuss concrete examples in terms of parameter mapping, "resonance example" etc}


\section{Approximate Bayesian Computation}
\label{sec:found:abc}

The first mention of the general ABC methodology is attributed to a seminal article on the adequacy of Bayesian statistics for scientific inference by \cite{rubin_bayesianly_1984}. There, ABC is used to illustrate the \emph{frequency behavior} of Bayesian posterior distributions across repeated experiments. The first practical applications of ABC algorithms were developed in the context of genetics in the 1990s. In \cite{pritchard_population_1999}, all key ingredients that we discuss below were already present: prior samples, summary statistics, a similarity metric, and a tolerance threshold. Historical overviews of these early developments can be found in \cite{beaumont_approximate_2002} and \cite{marin_approximate_2011}. ABC algorithms served as both a precursor and a source of inspiration for modern simulation-based inference (SBI) techniques. Understanding them provides conceptual insights into the specific challenges of SBI and informs the development of deep learning--based solutions.

\medskip

Simulation‑based inference (SBI) treats the likelihood \(p(\bx\mid\btheta)\) as
\emph{implicit}: instead of evaluating it at a fixed observation \(\bxobs\),
we only require a stochastic program.
%
\begin{equation}
\bx=\textsc{Sim}(\btheta,\epsilon)\;,
\qquad 
\epsilon\sim p(\epsilon)\;,
\end{equation}
%
that produces draws \(\bx\sim p(\bx\mid\btheta)\).
Here \(\btheta\) are inference parameters of interest, while \(\epsilon\) represents internal randomness of the simulator (\fex\ measurement noise, general latent parameters as in Eq.~\eqref{eqn:likelihood_with_nuisance}, etc) and are \emph{not} inferred.\footnote{Note that we can think of the simulator either as a stochastic program \(\bx = \text{Sim}(\btheta)\),
which returns a different \(\bx\) on each run, or as a deterministic function
\(\bx = \text{Sim}(\btheta, \epsilon)\) where the internal randomness has been
made explicit. Throughout these notes, we use both perspectives interchangeably.}


The simulator defines the \emph{joint} or \emph{generative} model
\(p(\bx,\btheta)=p(\bx\mid\btheta)p(\btheta)\).
From it we can harvest a data set of \(N\) i.i.d.\ pairs

\begin{equation}
  \mathcal{S}= \bigl\{(\bx^{(n)},\btheta^{(n)})\bigr\}_{n=1}^{N},
  \qquad
  (\bx^{(n)},\btheta^{(n)})\sim p(\bx\mid\btheta)\,p(\btheta).
\label{eq:sim_samples}
\end{equation}

\vspace{-0.5em}
\[
\mathcal{S}\quad
\stackrel{\text{SBI}}{\rightarrow}\quad
q_\phi(\btheta\mid\bx)
\]

\noindent
\emph{Question addressed by SBI}:  
given only the simulated sample set \(\mathcal{S}\),  
how can we construct faithful approximations to statistical quantities, such as the approximations \(q_\phi(\btheta \mid \bx) \approx p(\btheta\mid\bx)\) of the true posterior?


\subsection{Rejection ABC}
\label{sec:found:abc:rejection}

As discussed above, the goal is to infer \(p(\btheta \mid \bx)\) using samples from the joint distribution \(p(\bx, \btheta)\).
Approximate Bayesian Computation (ABC) provides a simple likelihood-free approach: we select those \((\bx, \btheta)\) pairs from simulations for which \(\bx\) is sufficiently close to the observed data \(\bx_{\mathrm{obs}}\).

To formalize this idea, ABC introduces a distance function \(d(\bx, \bx')\) that quantifies the similarity between two datasets.
In our context, this distance is used to compare simulated data \(\bx\) to the observed data \(\bx_{\mathrm{obs}}\). A common choice is the squared Euclidean distance:
%
\begin{equation}
    d(\bx, \bx')  = \|\bx - \bx'\|^2\;.
\end{equation}
%
This function determines which simulations are considered acceptable. Small distances indicate a good match between simulated and observed data, while large distances suggest that the parameters used to generate \(\bx\) are unlikely to have produced \(\bx_{\mathrm{obs}}\).
%
The most basic variant of ABC is the accept/reject scheme known as \emph{rejection ABC}~\citep{pritchard_population_1999}.
It proceeds by generating simulated pairs \((\bx, \btheta)\) and accepting only those for which \(d(\bx, \bx_{\mathrm{obs}}) \leq \epsilon\). This is illustrated formally in Algorithm~\ref{alg:ABC}.

\begin{algorithm}[ht]
\caption{Rejection Approximate Bayesian Computation}\label{alg:ABC}
\begin{algorithmic}[1]
\State \textbf{Input:}
\State \hspace{\algorithmicindent} Observed data $\bxobs$
\State \hspace{\algorithmicindent} Prior distribution $p(\btheta)$
\State \hspace{\algorithmicindent} Likelihood model $p(\bx \mid \btheta)$
\State \hspace{\algorithmicindent} Tolerance level $\epsilon$
\State \hspace{\algorithmicindent} Number of samples $N$
\State \textbf{Output:} Approximate posterior samples $\{\btheta^{(i)}\}_{i=1}^{N}$
\State Initialize an empty set of accepted parameters $\Theta \gets \emptyset$
\For{$i = 1$ to $N$}
    \State Sample $\btheta^{*}$ from the prior $p(\btheta)$
    \State Generate synthetic data $\bx^{*}$ from the model $p(\bx\mid \btheta^{*})$
    \If{$d(\bx^{*}, \bxobs) \leq \epsilon$}
        \State Accept $\btheta^{*}$: $\Theta \gets \Theta \cup \{\btheta^{*}\}$
    \EndIf
\EndFor
\State \textbf{Return} $\Theta$
\end{algorithmic}
\end{algorithm}

\medskip

The rejection ABC algorithm yields \(\btheta\) samples that follow a conditional distribution defined through the acceptance criterion, which we can formally write as
%
\begin{equation}
    p(\btheta \mid d(\bxobs, \bx) \leq \epsilon ) = \frac
    {\int_{d(\bxobs, \bx) < \epsilon} d\bx\, p(\btheta \mid \bx) p(\bx)}
    {\int_{d(\bxobs, \bx) < \epsilon} d\bx\, p(\bx)}\;.
\end{equation}
%
In the limit \(\epsilon \to 0\), and under mild regularity conditions, the accepted samples asymptotically follow the true posterior distribution:
%
\begin{equation}
    p(\btheta \mid d(\bxobs, \bx) \leq \epsilon )
    \overset{\epsilon \to 0}{\to} p(\btheta \mid \bxobs)\;.
\end{equation}
%
This limiting case highlights that ABC approximates the true posterior only in the idealized regime of vanishing \(\epsilon\), which is unattainable in practice. The need to accept samples with \(d(\bx_{\mathrm{obs}}, \bx) > 0\) is precisely what makes the method \emph{approximate}.

\medskip

The acceptance rate of the above algorithm for an observation \(\bxobs \in \mathbb{R}^d\) can be estimated as
%
\begin{equation}
    A = \int_{d(\bxobs, \bx) < \epsilon} d\bx\, p(\bx) \approx \epsilon^d\, p(\bxobs)\;,
\end{equation}
%
where the last step assumes that all \(d\) data dimensions contribute equally to the scaling with \(\epsilon\). This relation illustrates a key limitation of ABC: even in settings with moderately high-dimensional data $\bx$, the acceptance rate can become impractically low.  On the other hand, the dimensionality of the parameters $\btheta$, or of any other latent parameters, does not directly affect the acceptance rate---unless wide priors reduce the Bayesian evidence $p(\bxobs)$.

\subsection{Data Summaries and Density-Based ABC}
\label{sec:found:abc:summaries}

The severe drop in acceptance rate in settings with high-dimensional data motivates a natural strategy: instead of comparing full data vectors $\bx$, one can compress the data into low-dimensional representations before computing distances.  

In many practical applications, ABC relies on such handcrafted or learned summary statistics \(T(\bx) \in \mathbb{R}^k\), with \(k \ll d\), in order to improve acceptance efficiency.  In fact, even the earliest examples~\citep[\fex][]{pritchard_population_1999} relied already on the definition of suitable summary statistics. Ideally, these summaries retain all information relevant for inferring \(\btheta\), so that the ABC posterior remains faithful in the limit of small \(\epsilon\):
%
\begin{equation}
    p(\btheta \mid d(\bs(\bxobs), \bs(\bx)) \leq \epsilon )
    \underset{\bs(\cdot)\text{ sufficient for $\btheta$}}{\overset{\epsilon \to 0}{\to}}
    p(\btheta \mid \bxobs)\;.
\end{equation}
%
Compressing the representation of data $\bx$ in a way that still provides sufficient information about parameters $\btheta$ significantly increase the acceptance rate of rejection ABC, but plays also a critical role for modern SBI techniques in general.

\medskip

In rejection ABC, precise posteriors are obtained in the limit of a vanishing acceptance threshold, $\epsilon \to 0$, which cannot be taken because it implies a vanishing acceptance rate. This problem can be overcome by replacing the simple reject/accept step with density estimation~\citep[see, \fex][]{beaumont_approximate_2002, fan_approximate_2013}. 
Classical methods along these lines can be denoted as \emph{density-based ABC}. 

In density-based ABC, the generated \((\btheta, \bx)\) pairs are mapped to summaries \(\bs = T(\bx)\), and a density estimator \(q(\btheta, \bs)\) is fit to the resulting joint distribution. This density estimator can then be used to estimate the parameter posterior, formally given by
%
\begin{equation}
    q_{\text{density-ABC}}(\btheta \mid \bxobs) = 
    \frac{1}{Z}
    q(\btheta, \bs(\bxobs))\;,
\end{equation}
%
which can be sampled, \fex\ using Monte Carlo methods.

\medskip

Both the usage of summary statistics for data compression, as well as the adoption of density estimators for the purpose of precise posterior estimation, also play critical roles in modern approaches to SBI, where the task of identifying optimal data summaries and fitting flexible density estimator is done by neural networks.


\section{Informative Data Summaries}
\label{sec:found:summaries}

\begin{quotation}
    \textit{``[...] perfection is finally attained not when there is no longer anything to add, but when there is no longer anything to take away [...]''}

   \hfill --- Antoine de Saint-Exupéry (1939), Wind, Sand and Stars
\end{quotation}

Data summaries play a crucial role in applying ABC
---and simulation-based inference more generally---to real-world problems~\citep[see][for a review]{blum_comparative_2013}.
By mapping high-dimensional observations onto lower-dimensional representations, they focus the comparison between simulated and observed data on the most relevant features and reduce the number of required simulation runs. This raises a key question that guides both our analytical understanding and the design of summary networks for neural SBI: under which circumstances, and how, can we reduce the full dataset $\bx$ to a summary $T(\bx)$ \emph{without} losing information \emph{relevant for inference} about the parameters $\btheta$?

%Summary statistics aim to condense complex data while retaining the information most relevant for inferring $\btheta$. This aligns with the general goal of dimensionality reduction: simplifying data while preserving its informative content. In practice, informative low-dimensional summaries are rare outside idealized or low-dimensional scenarios. SBI methods therefore rely on approximately sufficient statistics, which strike a balance between computational efficiency and inference accuracy. 

%In the following, we discuss formal notions of sufficiency based on the factorization theorem, information-theoretic perspectives including mutual and Fisher information, and practical considerations regarding summary dimensionality.

\subsection{Foundations of Summary Construction}
\label{sec:found:summaries:foundations}

\cw{TODO: Read this carefully again including references.}

\paragraph{Factorization theorem.} A key theoretical concept underlying the use of data summaries in SBI is the \emph{Fisher-Neyman factorization theorem}~\citep[see][for a pedagogical introduction]{casella_statistical_2002}.\footnote{The concept of sufficiency was introduced in \cite{fisher_mathematical_1922}, and further developed mathematically by \cite{neyman_use_1928, halmos_application_1949}.}
For a given likelihood function $p(\bx \mid \btheta)$, a summary $T(\bx)$ is \emph{sufficient} (lossless) for $\btheta$ if and only if there exist functions $h$ and $g$ such that
%
\begin{equation}
    p(\bx \mid \btheta) = h(\bx)\, g(\btheta, T(\bx))\;.
\end{equation}
%
This means that all the dependence of $\bx$ on $\btheta$ is mediated through the summary $T(\bx)$: once $T(\bx)$ is known, the rest of the data contains no further information about $\btheta$. The term $g$ captures the coupling between parameters and data, while $h$ is purely a function of the observations. 

While this condition offers a clear theoretical benchmark, it is usually impractical to verify in complex models—especially in simulation-based settings. 
More importantly, our interest is not merely in whether sufficient statistics exist, but in identifying \emph{minimally sufficient} summaries: the most compressed representations that still retain all information relevant for inferring $\btheta$.

\paragraph{Exponential families.} To explore the consequences of the factorization theorem and the notion of \emph{minimally} sufficient summaries, we consider likelihood models that belong to the exponential family~\citep[see][for a comprehensive overview]{brown_fundamentals_1986}. This class includes many common distributions, such as the Gaussian, Poisson, Bernoulli, and exponential distributions. Exponential families frequently arise in contexts where measurements are aggregated or averaged over time and/or space, either inherently or as a modeling choice.\footnote{This connection is formalized by the Pitman-Koopman-Darmois theorem~\citep{brown_fundamentals_1986}, which states that under mild regularity conditions, only exponential family distributions admit finite-dimensional sufficient summaries that do not depend on the sample size.}

All likelihood functions that belong to the exponential family take the general form
%
\begin{equation}
    p(\bx \mid \btheta)
    = h(\bx)\, \exp\left( \boldeta(\btheta)^T \bphi(\bx) - A(\btheta) \right)\;,
\end{equation}
%
where $\bphi(\bx)$ are often referred to as canonical data summaries, and $\boldeta(\btheta)$ as the natural parameters. The coupling between data $\bx$ and parameters $\btheta$---corresponding to the function $g$ in the factorization theorem---is entirely mediated through the scalar product between these two quantities.


Common examples include Gaussian models with fixed covariance, where $\bx \sim \mathcal{N}(\bmu(\btheta), \Sigma)$, leading to $\bphi(\bx) = \bx$ and $\boldeta(\btheta) = \Sigma^{-1} \bmu(\btheta)$; Poisson models for counting experiments, where $\bx \sim \text{Pois}(\bmu(\btheta))$, with $\bphi(\bx) = \bx$ and $\boldeta(\btheta) = \log \bmu(\btheta)$; and models for variance inference, such as $\bx \sim \mathcal{N}(0, \Sigma(\btheta))$, where the summaries are quadratic products $x_i x_j$ and the natural parameters are entries of the precision matrix $\Sigma^{-1}(\btheta)$. These familiar cases illustrate how exponential families naturally encompass many statistical models.

\medskip

The question, then,  is how to construct low-dimensional   ---and potentially \emph{minimal}---summaries $T(\bx)$ given the canonical summaries $\bphi(\bx)$ and the natural parameters $\boldeta(\btheta)$. Technically, a  \emph{minimal} sufficient statistics is one  that is  a function of all  other sufficient  statistics~\citep{lehmann_completeness_1950}. This has been rigorously developed~\citep[see][for an overview]{lehmann_theory_1998}, but we resort here to a few examples.

Consider, for instance, the special case of a Gaussian model where the mean depends linearly on the parameters, $\bmu(\btheta) = \sum_{i=1}^k \theta_i \bmu_i$. The corresponding sufficient summaries are $T_i(\bx) = \bx^T \Sigma^{-1} \bmu_i$, i.e., scalar products between the whitened data and model templates. Notably, the dimension of the summary $\bT(\bx) \in \mathbb{R}^k$ depends on the number of model parameters $k$, not on the dimensionality of the data.

More generally, in exponential family models, low-dimensional sufficient summaries can be obtained by projecting $\bphi(\bx)$ onto the space spanned by the natural parameters:
%
\begin{equation}
    T(\bx) = \mathbb{P}_S \bphi(\bx)
    \quad \text{with} \quad
    S = \operatorname{span}\{\boldeta(\btheta) \mid \btheta \in \Theta\}\;.
\end{equation}
%
This highlights that, in the case of exponential families, the intrinsic dimensionality of a compressed sufficient summary is governed by the variation of the model across parameter space—not by the original dimensionality of the data. This projection compresses $\bphi(\bx)$ just enough to retain the directions in which $\btheta$ influences the distribution, discarding those that are uninformative for inference.

\paragraph{Exponential families as a guide.}
While exponential families offer valuable structure, most real-world simulation models do not strictly belong to this class.
Common examples include hierarchical models, partial marginalization over latent variables or parameters, and mixture models—such as those incorporating different types of noise. These operations typically lead to likelihoods that no longer admit a finite-dimensional sufficient statistic or factorize in exponential family form.

This highlights an important limitation: in many real-life applications, exact finite-dimensional sufficient statistics do not exist, or at least cannot be derived in closed form. As a result, one is generally forced to work with \emph{approximately} sufficient summaries that balance a trade-off between computational tractability and inferential accuracy. 

Nevertheless, exponential family models remain highly valuable as idealized cases. As demonstrated above, they offer conceptual guidance for constructing informative low-dimensional summaries and can  inform the design of neural network architectures with inductive biases that promote generalization---even when applied to models that only approximately follow exponential-family structure.


\subsection{Quantifying Informativeness}
\label{sec:found:summaries:informativeness}

\begin{figure}[t]
    \centering
    \includegraphics[width=1\linewidth]{figures/Venn.pdf}
    \caption{Information-theoretic decomposition in simulation-based inference. The total mutual information \(\mathcal{I}(\bx, \btheta)\) (green) quantifies all available information about parameters. A summary \(T(\bx)\) retains \(\mathcal{I}(T(\bx); \btheta)\), with the loss corresponding to the expected divergence between the full and summary-conditioned posteriors, \(\mathbb{E}_{p(\bx)}[D_{\mathrm{KL}}(p(\btheta \mid \bx) \mid\mid p(\btheta \mid T(\bx)))]\). The conditional entropy \(\mathcal{H}(T(\bx) \mid \btheta)\) reflects uninformative variability in the summary.}
    \label{fig:mutual_information }
\end{figure}

As discussed in the previous subsection, reducing data to low-dimensional summaries typically leads to some loss of information. The goal in constructing approximately sufficient summaries is to minimize this information loss as much as possible. Ideally, the loss should only affect regions of the likelihood function $p(\bx \mid \btheta)$ where the likelihood is negligible, and hence does not meaningfully contribute to the inference of $\btheta$. To quantify this, we need a principled way to measure how much \emph{information} an observation $\bx$ carries about the parameters $\btheta$.

\paragraph{Mutual information.} A good starting point for quantifying the loss of relevant information is the concept of \emph{mutual information} \citep[see][for an introduction to information theory]{cover_elements_2006}.
Mutual information provides a general measure of how much information is shared between two random variables—in this case, between $\btheta$ and $\bx$—without requiring specific assumptions about the form of the likelihood or the prior.  
It is defined as
\begin{equation}
    \mathcal{I}(\btheta; \bx) 
    \equiv \mathbb{E}_{p(\btheta, \bx)}\left[\log \frac{p(\btheta, \bx)}{p(\btheta)p(\bx)}\right]\;,
\end{equation}
where the joint distribution $p(\btheta, \bx)$ is compared to the product of marginals $p(\btheta)p(\bx)$. Intuitively, $\mathcal{I}(\btheta; \bx)$ quantifies the expected reduction in uncertainty about $\btheta$ after observing $\bx$.

To interpret the concept of mutual information, it is helpful to discuss how information content and information gain connect to (differential) \emph{entropy}~\citep{shannon_mathematical_1948}
\begin{equation}
    \mathcal{H}[p(\btheta)] \equiv \bbE_{p(\btheta)} \bigl[-\log p(\btheta)\bigr]\;.
\end{equation}
Entropy quantifies how ``spread out'' or uncertain a random variable is.  
For instance, a univariate Gaussian distribution with variance $\sigma^2$ has differential entropy $\mathcal{H} = \tfrac{1}{2}\,\log\bigl(2\pi e\,\sigma^2\bigr)$.
In contrast, for a discrete random variable with $N$ uniformly likely outcomes, the entropy is $\log(N)$.

From an information-theoretic perspective, the entropy of a parameter $\btheta$ quantifies our uncertainty before observing any data.  
The mutual information then measures the expected amount by which this uncertainty is reduced after seeing $\bx$ (in bits or nats).\footnote{%
    The unit depends on the logarithm's base: natural logarithms yield information in \emph{nats}, while base-2 gives \emph{bits}.
    For example, learning the value of a uniformly distributed 8-bit integer yields 8 bits of information, corresponding to $\log(256) \approx 5.5$ nats.
}
This reduction is expressed via
\begin{equation}
    \mathcal{I}(\bx;\,\btheta)
    =
    \mathcal{H}[p(\btheta)]
    -
    \mathbb{E}_{p(\bx)} \bigl[\mathcal{H}[p(\btheta \mid \bx)]\bigr].
\end{equation}
Essentially, we compare the parameter entropy before observing $\bx$ to the expected posterior entropy after observing $\bx$.  
If measuring $\bx$ significantly decreases the entropy of $\btheta$, then $\bx$ provides a substantial information gain about $\btheta$.  

\medskip

Generally, mutual information satisfies the inequality
%
\begin{equation}
    \mathcal{I}(\btheta; \bx) \geq 0\;,
\end{equation}
%
with equality if and only if $\btheta$ and $\bx$ are statistically independent. In that case, $p(\btheta, \bx) = p(\btheta)p(\bx)$, and the posterior becomes equal to the prior, $p(\btheta \mid \bx) = p(\btheta)$, implying that observing $\bx$ provides no information about the parameters $\btheta$.

As a simple example, consider the case where both $\btheta$ and $\bx$ are scalar Gaussian variables with unit variance, so that the joint distribution $p(\btheta, \bx)$ is a bivariate normal. In this case, mutual information reduces to $\mathcal{I}(\btheta; \bx) = -\frac{1}{2} \log(1 - \rho^2)$, where $\rho$ is the correlation coefficient between $\btheta$ and $\bx$. This expression provides a direct and intuitive link between statistical dependence and information content. \cw{TODO: check math}


\paragraph{Information processing inequality.}

In general, no deterministic or stochastic transformation of $\bx$ can increase the information it contains about $\btheta$.  
Hence, replacing $\bx$ with a summary $T(\bx)$ often entails some loss of information, an effect captured by the \emph{data-processing inequality}~\citep{cover_elements_2006}
\begin{equation}
    \mathcal{I}(\btheta; \bx) \geq \mathcal{I}(\btheta; T(\bx)).
\end{equation}
This states that $T(\bx)$ cannot contain more information about $\btheta$ than $\bx$ itself.  
When the inequality is saturated, there is no information loss, and $T(\bx)$ is a \emph{sufficient} summary.  
In that case, $T(\bx)$ also satisfies the Neyman-Fisher factorization theorem from the previous subsection~\citep{kullback_information_1959, cover_elements_2006}.

Maximizing mutual information provides a principled approach to finding approximately sufficient summaries.  
Because mutual information is computed as an expectation under $p(\btheta, \bx)$, it prioritizes regions where $p(\bx \mid \btheta)$ is large.  
This naturally focuses on the portions of parameter–data space that matter most for inference, while assigning less weight to regions in the likelihood tails.

\bigskip

\paragraph{Connection to Fisher information.}

A fundamentally different, though related, notion of ``information'' that often appears in the SBI literature is \emph{Fisher information}~\citep{fisher_mathematical_1922, lehmann_theory_1998}. It is based on the \emph{score function}, defined as
\[
    \bs(\btheta;\,\bx)
    \equiv
    \nabla_{\btheta}\,\log p(\bx \mid \btheta).
\]
Intuitively, the more steeply the likelihood changes with respect to $\btheta$, the more informative the observation $\bx$ is about the parameter.  
Conversely, if the likelihood is flat in $\btheta$, the data is uninformative.  
In practice, the score function is typically evaluated at a reference point $\btheta^*$. Its expected value vanishes under the model,
\[
    \bbE_{p(\bx \mid \btheta^*)}[\bs(\btheta^*;\,\bx)] = 0,
\]
so the natural way to measure its informativeness is through its covariance.

This leads to the definition of the \emph{Fisher information matrix} at $\btheta^*$:
\begin{equation}
    \mathcal{J}(\btheta^*)
    \equiv
    \bbE_{p(\bx \mid \btheta^*)}
    \left[
        \bs(\btheta^*;\,\bx)^T\,\bs(\btheta^*;\,\bx)
    \right].
\end{equation}
According to the Cramér–Rao bound, no unbiased estimator of $\btheta$ can have a variance smaller than the inverse of $\mathcal{J}(\btheta^*)$.  
In the large-sample limit, the posterior often approximates a Gaussian with covariance matrix given by the inverse Fisher information, so $\mathcal{J}(\btheta^*)$ controls the asymptotic uncertainty of parameter estimates.

To connect this to mutual information, note that in the large-sample regime (\textit{i.e.}, when the dataset is large enough for the posterior to become sharply peaked and approximately Gaussian), the locally averaged posterior entropy satisfies~\citep[see][for a similar discussion]{clarke_information-theoretic_1990}
%
\begin{equation}
    \bbE_{p(\bx \mid \btheta^*)} \bigl[\mathcal{H}[p(\btheta \mid \bx)]\bigr]
    \approx
    \tfrac{1}{2}\,\log\det
    \bigl(
        2\pi\, \mathcal{J}^{-1}(\btheta^*)
    \bigr).
\end{equation}
\cw{TODO: Double-check}
Under this approximation, minimizing posterior entropy corresponds to maximizing the determinant of the Fisher information matrix.  
This implies that, in the Gaussian limit, maximizing mutual information is approximately equivalent to maximizing Fisher information.  
It is important to note, however, that Fisher information is a local quantity—defined at a specific parameter value $\btheta^*$—and reflects only the local curvature of the likelihood.  
In contrast, mutual information captures global dependence between $\btheta$ and $\bx$ and remains well-defined in discrete, multi-modal, or non-differentiable settings where the Gaussian approximation does not hold.

- cite \cite{charnock_automatic_2018} for information maximizing neural networks

%similar to https://en.wikipedia.org/wiki/Mutual_information#/media/File:Figchannel2017ab.svg

\subsection{How Many Summaries Are Enough?}
\label{sec:found:summaries:dimensions}

We now revisit the question of how many dimensions an approximately sufficient summary must have, even when no strictly low-dimensional sufficient statistic exists.  
Intuitively, the required dimensionality relates to how many parameters are needed to describe the posterior distribution $p(\btheta \mid \bx)$ across relevant data.  
While formal results are limited, we can gain insight by rewriting the mutual information in a convenient form.

\medskip

The mutual information between $\btheta$ and a compressed representation $T(\bx)$ can be written as~\citep{cover_elements_2006}
\begin{equation}
    \mathcal{I}(T(\bx);\;\btheta)
    =
    \mathcal{I}(\bx;\;\btheta)
    -
    \bbE_{p(\bx)}
    \bigl[
        D_{\mathrm{KL}}
        \bigl(
            p(\btheta \mid \bx) 
            \,\big\|\, 
            p(\btheta \mid T(\bx))
        \bigr)
    \bigr],
\end{equation}
where $D_{\mathrm{KL}}(p \,\|\, q)$ denotes the Kullback-Leibler (KL) divergence,
\begin{equation}
    D_{\mathrm{KL}}(p(\btheta) \,\|\, q(\btheta))
    \equiv
    \bbE_{p(\btheta)}
    \left[
        \log \frac{p(\btheta)}{q(\btheta)}
    \right],
\end{equation}
a nonnegative measure of dissimilarity between distributions.\footnote{The non-negativity is a consequence of the Gibbs' inequality, see \fex, \cite{mackay_information_2003}.}  
Since mutual information itself is a KL divergence between joint and product distributions, it is natural that KL reappears here.

This decomposition shows that $T(\bx)$ should preserve the aspects of $\bx$ most predictive of the full posterior.  
If the posterior is nearly Gaussian, a small number of summary components—often related to the dimensionality of $\btheta$—may suffice.  
More generally, if the posterior's shape is fully determined by a low-dimensional feature of the data (e.g., its mean), then \emph{a single summary per parameter may be enough}.  

However, when the posterior shape itself varies with $\bx$, more summary dimensions are required to capture these changes.  
One caveat is that KL-based objectives tend to emphasize regions of high posterior density, which can lead to underrepresentation of tail structure—potentially requiring additional summary dimensions to capture rare but important features.
While this reasoning is heuristic, it aligns with empirical findings and provides practical guidance for summary design.  
In practice, including some redundancy in $T(\bx)$ can also improve training dynamics and robustness, particularly when summaries are learned using neural networks.


\begin{figure}[t]
\centering
    \includegraphics[width=0.80\linewidth]{figures/fig2.drawio.pdf}
\caption{
Mutual information between parameters and data before and after compression. 
Equality $\mathcal{I}(\boldsymbol{\theta}; \mathbf{x}) = \mathcal{I}(\boldsymbol{\theta}; T(\mathbf{x}))$ characterizes sufficient summaries. 
In general, the KL term in the decomposition quantifies information loss relevant for inference.
}
\label{fig:summary_mi}
\end{figure}


%\section{A Projectile Simulator with a Wall-Mounted Detector}

%--------------------------------------------------------------
% Toy projectile simulator with a wall‑mounted detector
%--------------------------------------------------------------
\begin{algorithm}[t]
\caption{Projectile simulator with selective detector}\label{alg:proj}
\begin{algorithmic}[1]
\State \textbf{Input:} 
\State \hspace{\algorithmicindent} Priors  
$p(v_{\text{ini}})=\mathcal N(0,\sigma_v^{2})$ (truncated to $v_{\text{ini}}>0$)  
\State \hspace{\algorithmicindent} \phantom{Priors} $p(\phi_{\text{ini}})=\text{Uniform}\bigl[\phi_{\min},\phi_{\max}\bigr]$  
\State \hspace{\algorithmicindent} \phantom{Priors} $p(M)=\text{Poisson}(\lambda)$ \Comment{number of throws}  
\State \hspace{\algorithmicindent} \phantom{Priors} 
$p(\epsilon)=\mathcal{N}(0, \sigma_\epsilon^2)$
\State \hspace{\algorithmicindent} Disturbance $p(d)=\mathcal N(0,\sigma_d^{2})$  \Comment{e.g.\ wind kick}  
\State \hspace{\algorithmicindent} Wall position $x=L$, detector window $[h_{\min},h_{\max}]$
\State \textbf{Output:} Accepted samples $\Theta=\{(h^{(i)},v^{(i)},\phi^{(i)})\}$

\State Sample $M\sim p(M)$;  $\Theta\gets\emptyset$
\For{$i=1$ to $M$}
    \State $v\sim p(v_{\text{ini}})$,\enspace $\phi\sim p(\phi_{\text{ini}})$,\enspace $d\sim p(d)$
            \Comment{Draw from parameter priors}
    \State $h\gets \textsc{Sim}(v, \phi, d)$
    %\underbrace{L\tan\phi-\dfrac{gL^{2}}{2v^{2}\cos^{2}\phi}}_{\text{ballistic}}+d$
            \Comment{Ballistic simulator}
    \If{$h_{\min}\le h\le h_{\max}$}  \Comment{Ball hits detector}
        \State $\hat h \gets h + \epsilon$, $\epsilon \sim p(\epsilon)$
        \Comment{Add measurement noise}
        \State $\Theta \gets \Theta \cup \{(\hat h,v,\phi)\}$
    \EndIf
\EndFor
\State \Return $\Theta$
\end{algorithmic}
\end{algorithm}



- This example also illustrates several limitations of basic SBI strategies such as ABC, which motivate the neural methods introduced in the following sections:

\paragraph{Setup.}

- We consider the projectile simulator defined in Algorithm~\ref{alg:proj}, which models the ballistic trajectories of projectiles launched with uncertain initial speed and angle, subject to stochastic wind disturbance and measurement noise.

- A wall-mounted detector at horizontal position \( x = L \) records only those projectiles whose impact height falls within a specified window. The observation is a noisy version of the true height, \( \hat h = h + \epsilon \).

- This basic setup captures many features typical of simulation-based models: selective detection, stochasticity, and implicit marginalization over latent variables (e.g.\ wind and speed).

\paragraph{Inference task.}

- Our goal is to infer the posterior distribution over the initial launch angle \( \phi \), given an observed impact height \( \hat h \).

- This corresponds to computing the marginal posterior \( p(\phi \mid \hat h) \), without requiring the full joint posterior over all simulator parameters.

- As discussed in Sec.~\ref{sec:marginal_inference}, ABC and other SBI methods can directly approximate such marginals, even when the full likelihood is intractable.

\paragraph{ABC with a single projectile.}

- In the simplest case, each simulation trial corresponds to a single projectile. We apply rejection ABC (Algorithm~\ref{alg:ABC}) using the observed height \( \hat h \) and a distance metric \( d(h, \hat h) \).

- For simplicity, we use the mean height as summary statistic—trivial in this case since there's only one sample.

- [Insert plot: posterior samples for \( \phi \) from rejection ABC for one projectile, show ground truth]

\paragraph{ABC with multiple projectiles.}

- To increase the realism and complexity of the simulation, we allow the number of projectiles per trial to vary. In Algorithm~\ref{alg:proj}, this is modeled by drawing \( M \sim \text{Poisson}(\lambda) \).

- The observed data \( \hat h_1, \ldots, \hat h_M \) now forms a variable-length set. We summarize it using the mean impact height \( \bar{h} = \frac{1}{M} \sum_i \hat h_i \), or both mean and variance.

- This introduces additional latent variables (e.g.\ \( M \), wind realizations), making the likelihood highly intractable. Yet, ABC naturally integrates over them via forward simulation.

- [Insert plot: posterior for \( \phi \) inferred using ABC with multiple projectiles; compare to single-ball case]

\paragraph{Remarks.}

- This example illustrates how SBI methods—including basic rejection ABC—can handle selective data, measurement noise, and marginalization over internal simulator variables without requiring explicit likelihood evaluation.

- The difficulty of expressing \( p(\hat h \mid \phi) \) or \( p(\bar h \mid \phi) \) in closed form highlights the key motivation for simulation-based inference.

- Extensions to other tasks—e.g.\ inferring whether any projectiles hit the target (object detection), or comparing models with different disturbance levels (model selection)—can be handled similarly.


\section{Key Challenges in Simulation-Based Inference}
\label{sec:found:challenges}

The conceptual flexibility of SBI comes at the cost of a range of technical challenges that must be addressed to make SBI methods efficient, robust, and scientifically reliable. Below, we summarize key challenges, and how they can be addressed with the modern neural solutions that we will be discussed in the remainder of these lecture notes.

\cw{TODO: Read at the end again and link to various subsections exactly}

\begin{itemize}

\item \textbf{Constructing informative data summaries.}  
ABC and related methods rely on low-dimensional summaries \(T(\bx)\) to avoid the curse of dimensionality. Designing these summaries by hand is difficult and problem-specific. Neural approaches can instead learn them automatically end-to-end from simulation data---\fex\ using mutual or Fisher information, contrastive objectives, and encoder architectures like CNNs, DeepSets, and graph networks.  
We discuss these mechanisms in Sec.~\ref{chap:methods}.

\item \textbf{Modeling parameter–data relationships in a tractable way.}  
Classical ABC is computationally wasteful and must be rerun for each new observation. Neural SBI solves this by amortizing inference—learning reusable mappings from data to posteriors. Methods include neural density estimators (e.g., normalizing flows) and score- or diffusion-based surrogates.  
These models are covered in Secs.~\ref{sec:density_estimation}, \ref{sec:score_models}, and \ref{sec:transforming_distributions}.

\item \textbf{Validating neural posterior approximations.}  
Unlike classical inference, neural SBI produces learned posteriors \(\tilde p(\btheta \mid \bx)\) whose accuracy is not guaranteed. We must assess their calibration and coverage using simulation-based diagnostics, classifier comparisons, and posterior predictive checks.  
We discuss these tools in Sec.~\ref{sec:validation}, particularly \ref{sec:diagnostics} and \ref{sec:calibration}.

\item \textbf{Detecting and addressing model mis-specification.}  
When real data differ from the assumptions encoded in the simulator (e.g., incorrect noise or selection effects), inference results can be biased. We introduce methods to test model fit and detect mis-specification—such as residual checks, classifier-based discrepancy measures, and internal consistency tests.  
See Sec.~\ref{sec:mis_specification}.

\item \textbf{Building robustness against known mismatch.}  
In many applications, mis-specification is anticipated and must be accommodated. Robust SBI methods aim to absorb such deviations—e.g., through summary learning with robustness objectives or uncertainty-aware posteriors.  
We explore these strategies in Sec.~\ref{sec:model_uncertainty}.

\item \textbf{Focusing simulation effort on informative regions.}  
Classical rejection-based approaches waste simulations far from the posterior. Modern sequential SBI strategies guide simulation adaptively toward informative regions using active learning and adaptive proposals.  
These methods are introduced in Sec.~\ref{sec:sequential_sbi}.

\end{itemize}

\chapter{Neural Methods for Simulation-Based Inference}
\label{chap:methods}

\begin{quotation}
    \textit{``A complex system that works is invariably found to have evolved from a simple system that worked. A complex system designed from scratch never works and cannot be patched up to make it work. You have to start over with a working simple system.''}

    \hfill --- John Gall, 1977, The Systems Bible
    
\end{quotation}

\cw{TODO: Restructure where necessary subsection vs paragraph etc, add references}

\section{Neural Networks for SBI}
\label{sec:methods:nn}

Neural networks are flexible tools for learning parametric functions from data. In the context of SBI, this typically takes the form of supervised learning: we generate synthetic observations $\bx$ from a simulator and record the corresponding parameters, $\btheta$, that produced them.  The general goal is to learn statistical functions like $p(\btheta\mid \bx )$ and $p(\bx \mid \btheta)$, or summary statistics that preserve information about this relationship.

We will here just provide a very minimal crash-course to provide context.  Excellent introductions to neural networks can be found in standard references such as  \cite{nielsen_neural_2015} and
\cite{goodfellow_deep_2016}.

\subsection{Function Learning with Neural Networks}  
\label{sec:methods:nn:mlps}

\subsubsection{The Multi-Layer Perceptron}

A fundamental building block of deep learning is the \emph{multi-layer perceptron} (MLP).  It is a parametric function that maps inputs $\mathbf{x} \in \mathbb{R}^{d_\text{in}}$ onto output targets $\mathbf{t} \in \mathbb{R}^{d_\text{out}}$, 
\[
f_\phi: \mathbb{R}^{d_\text{in}} \to \mathbb{R}^{d_\text{out}},
\]
realized as a feedforward neural network.  It consists of a sequence of hidden layers with intermediate representations \( \mathbf{h}_l \in \mathbb{R}^{n_l} \), defined recursively by
\[
\mathbf{h}_0 \equiv \mathbf{x}, \quad 
\mathbf{h}_l = a(\mathbf{W}_l \mathbf{h}_{l-1} + \mathbf{b}_l), \quad l = 1, \dots, L{-}1, \quad 
\mathbf{h}_L = \mathbf{W}_L \mathbf{h}_{L-1} + \mathbf{b}_L,
\]
with output \( \mathbf{t} \equiv \mathbf{h}_L \in \mathbb{R}^{d_\text{out}} \). The parameters \( \phi = \{\mathbf{W}_l, \mathbf{b}_l\}_{l=1}^L \) include weight matrices \( \mathbf{W}_l \in \mathbb{R}^{n_l \times n_{l-1}} \) and biases \( \mathbf{b}_l \in \mathbb{R}^{n_l} \), where \( n_0 = d_\text{in} \) and \( n_L = d_\text{out} \).

The function \( a \) is a so-called \emph{activation function}---a non-linear scalar function applied element-wise to each layer’s output. The intermediate quantities \( \mathbf{t}_l \) are also referred to as the \emph{activations} of layer \( l \). The activation function enables the network to model non-linear functions. A key design feature is that it should be non-saturating (\textit{i.e.}, its derivative does not vanish for large input values) to avoid vanishing gradients. The simplest and historically most common choice is the rectified linear unit (ReLU), defined as \( a(x) = \max(0, x) \), though many other variants exist.
With sufficient width, MLPs can approximate arbitrary continuous functions (\emph{universal approximation theorem}, for details see \cite{goodfellow_deep_2016}).


\paragraph{Training with stochastic gradient descent}

Neural networks are commonly trained by minimizing loss functions averaged over training data, which might come in the form of $(\bt_i, \bx_i)$ tuples where $i=1, \dots, N_{\rm train}$. A common choice is the squared loss, which when averaging over all $N_\text{train}$ training examples yields the loss function
\begin{equation}
\mathcal{L}_{\text{train}}[\boldsymbol{\phi}] = \frac{1}{N_{\text{train}}} \sum_{i=1}^{N_{\text{train}}} \| f_{\boldsymbol{\phi}}(\mathbf{x}_i)- \mathbf{t}_i \|^2\;,
\label{eqn:MSE_loss}
\end{equation}
where $f_{\boldsymbol{\phi}}$ is a neural network parametrized by weights and biases collectively denoted $\boldsymbol{\phi}$.
Training is typically performed using stochastic gradient descent (SGD), which updates the parameters based on a small randomly sampled subset (mini-batch) of the training data. Instead of computing gradients over the full training set, SGD approximates $\mathcal{L}_{\text{train}}[\boldsymbol{\phi}]$ using a mini-batch of size $B$:
\[
\mathcal{L}_{\text{batch}}[\boldsymbol{\phi}] = \frac{1}{B} \sum_{i \in \mathcal{B}} \| f_{\boldsymbol{\phi}}(\mathbf{x}_i)- \mathbf{t}_i \|^2,
\]
where $\mathcal{B}$ is a randomly sampled subset of $B$ training examples. The parameter update rule is:
\[
\boldsymbol{\phi}_{k+1} = \boldsymbol{\phi}_k - \eta_k \nabla_{\boldsymbol{\phi}} \, \mathcal{L}_{\text{batch}}[\boldsymbol{\phi}],
\]
where $\eta_k$ is the learning rate.  To prevent overfitting, the final parameter version is selected based on its performance on a separate validation set of size $N_{\text{val}}$. This procedure, known as \emph{early stopping}, selects the model that generalizes best to unseen data (again see \cite{goodfellow_deep_2016} for details).


\subsubsection{What Networks Learn}

Minimizing the squared loss function evidently leads to a network that approximates the relationship $f_\phi(\mathbf{x}) \approx \mathbf{t}$. However, different loss functions make different assumptions and focus on specific aspects of this relationship. To understand what the network learns in the limit of infinite data, let us assume that training and validation data are drawn from some underlying statistical model $\mathbf{x}, \mathbf{t} \sim p(\mathbf{x}, \mathbf{t})$.

In the limit $N_{\text{train}} \to \infty$, the empirical training loss converges to the expected loss:
\begin{equation}
\mathcal{L}_{\infty}[f] = \int p(\mathbf{x}, \mathbf{t}) \, \| f(\mathbf{x}) - \mathbf{t} \|^2 \, d\mathbf{x} \, d\mathbf{t}.
\end{equation}

Our task is to find the function $f^*$ that minimizes this expected loss. Assuming the network has infinite capacity and flexibility, we can use functional calculus. Taking the functional derivative with respect to $f$ and setting it to zero:
\begin{align}
\frac{\delta \mathcal{L}_{\infty}}{\delta f} &= \int p(\mathbf{x}, \mathbf{t}) \, 2(f(\mathbf{x}) - \mathbf{t}) \, d\mathbf{t} = 0 \\
&= 2 p(\mathbf{x}) \left[ f(\mathbf{x}) \int p(\mathbf{t} \mid \mathbf{x}) \, d\mathbf{t} - \int \mathbf{t} \, p(\mathbf{t} \mid \mathbf{x}) \, d\mathbf{t} \right] = 0,
\end{align}
where we used $p(\mathbf{x}, \mathbf{t}) = p(\mathbf{x}) p(\mathbf{t} \mid \mathbf{x})$. Since $\int p(\mathbf{t} \mid \mathbf{x}) \, d\mathbf{t} = 1$, this yields:
\[
f^*(\mathbf{x}) = \int \mathbf{t} \, p(\mathbf{t} \mid \mathbf{x}) \, d\mathbf{t} = \mathbb{E}_{p(\mathbf{t} \mid \mathbf{x})}[\mathbf{t}].
\]
That is, minimizing mean squared error leads to a predictor that returns the mean value of the target variable $\mathbf{t}$, conditioned on the input $\mathbf{x}$.  


\paragraph{From point to density estimation} 

A standard MLP with squared loss function acts as a point estimator, learning only the conditional mean $\mathbb{E}[{\bt} \mid {\bx}]$~\citep[\fex][]{bishop_pattern_2006}. But what if we want to learn the full conditional distribution $p(\bt \mid \bx)$, not just its mean?

\medskip

As a first step toward density estimation, consider modeling the posterior as a 1-dim Gaussian: $q_\phi(t \mid \bx) = \mathcal{N}(t; 
\mu = f_\phi(\bx), \sigma^2 = g_\phi(\bx))$ where the network outputs both mean and variance. Maximizing the likelihood of this model yields the loss function
\begin{equation}
\mathcal{L}_{\text{train}}[\boldsymbol{\phi}] = \frac{1}{N_{\text{train}}} \sum_{i=1}^{N_{\text{train}}} 
\left[\frac{( f_{\boldsymbol{\phi}}(\mathbf{x}_i)- t_i )^2}{g_\phi(\bx_i)}
+\log g_\phi(\bx_i)\right]\;.
\label{eqn:MSE_loss_second}
\end{equation}
Like above, one can use functional derivative calculus to show that in the limit of infinite training data and network capacity, the loss ensures that the network learns both the mean and the variance of the posterior distribution,
$$
f^*(\bx) = \mathbb{E}[t \mid \bx]
\quad \text{and}\quad 
g^*(\bx) = \text{Var}[t \mid \bx]\;.
$$
This constitutes our first example of neural density estimation, albeit limited to Gaussian distributions. The approach of using neural networks to output both mean and variance for uncertainty quantification dates back to at least \cite{nix_estimating_1994}, and has become a fundamental technique in probabilistic deep learning. In our context, the scalar networks $f_\phi(\bx)$ and $g_\phi(\bx)$ learn the two statistics needed to parametrize a Gaussian posterior distribution---effectively training data summaries that provide information about the parameter and its uncertainty.

\medskip

In general, loss functions \emph{define} the statistical quantities that networks learn to approximate. Through different loss functions and architectural choices, neural networks can capture increasingly rich statistical structures, from simple point estimates to full multimodal distributions $p({\bt} \mid {\bx})$, as we will explore in subsequent sections.


\subsubsection{From Theory to Practice} 

Training neural networks remains more akin to hands-on bench work than idealized engineering. Like crafting a delicate clockwork, successful training requires balancing competing considerations: adaptive learning rates, regularization, and architectural choices that encode appropriate inductive biases. These skills develop through practice and accumulated experience. We highlight here some relevant aspects that underlie this craft.

\begin{description}
   \item \textbf{Optimization beyond SGD.} While SGD is conceptually simple, it is often replaced or enhanced by adaptive optimizers such as \emph{Adam}~\citep{kingma_adam_2017}, which adjust each parameter's update based on running estimates of the gradient's mean and variance. Beyond this, learning rate scheduling is critical for stabilizing results and achieving convergence~\citep[see][]{goodfellow_deep_2016}.

   \item \textbf{Initialization and normalization.} Proper weight initialization and input/output normalization are crucial for efficient network training, as they strongly affect gradient propagation and convergence behavior. A typical design goal is that each layer approximately maps standard normally distributed inputs to outputs with similar statistics. Common initialization schemes (Xavier, Kaiming) are standard in most deep learning frameworks~\citep{goodfellow_deep_2016}. In deeper architectures, batch normalization~\citep{ioffe_batch_2015} and skip connections further stabilize activations and gradients. As a corollary, input features and target variables should be normalized to zero mean and unit variance.
   
   \item \textbf{Inductive bias and architecture.} An \emph{inductive bias} refers to structural assumptions built into the network architecture that favor certain types of solutions~\citep{goodfellow_deep_2016}. MLPs have minimal inductive bias—they are flexible non-linear function approximators capable of representing virtually any sufficiently smooth function. Many advanced architectures (convolutional, graph-based) can be viewed as structured MLP variants that incorporate constraints suited to specific data modalities, often including operations beyond linear transformations such as pooling, softmax, or quadratic interactions.
   
   \item \textbf{Overparameterization and implicit regularization.} Modern neural networks typically possess far more parameters than needed to fit the training data. While naive expectation suggests a vast landscape of local minima, SGD remarkably tends to select low-complexity functions that generalize well—a phenomenon attributed to implicit regularization~\cite[\fex][]{mehta_high-bias_2019}. Recent work on mode connectivity reveals that many minima lie on connected manifolds in parameter space, forming continuous paths of near-constant loss~\citep[\fex][]{garipov_loss_2018}.
\end{description}


\subsection{Neural Networks for Modeling Statistical Relations}
\label{sec:methods:nn:relations}

The general architecture of networks and training in simulation-based inference is shown in Fig.~\ref{fig:sbi_overview}.  Simulated data is fed through a learnable \emph{summary extract network}, $T_\phi(\bx)$, whose output is combined with the model parameters $\btheta$ in a trainable \emph{inference head}.  Depending on the \emph{inference loss} function, the neural networks of the inference head learn to approximate different statistical relations, such as the parameter posterior or the score. 


\begin{figure}[ht]
\centering
    \includegraphics[width=0.90\linewidth]{figures/fig3.drawio.pdf}
    \caption{Basic architecture for simulation-based inference. \textit{Green box:} Simulation of parameters $\btheta \sim p(\btheta)$ and corresponding observations $\mathbf{x} \sim p(\mathbf{x} \mid \btheta)$. \textit{Yellow box:} An (optional) summary extractor $T_\phi(\mathbf{x})$ to compresses high-dimensional data into informative features. \textit{Red box:} The inference head approximates the target statistical relationship. \textit{Gray box:} The inference loss $\mathcal{L}$ defines what statistical quantity is learned. \textit{Right panels:} Example outputs.}
    \label{fig:sbi_overview}
\end{figure}

Table~\ref{tab:sbi_method_comparison} provides an overview of the main approaches we will explore in detail throughout this section. Looking at Bayes' theorem from Eq.~\eqref{eqn:Bayes_theorem},  we see multiple entry points for neural network approximation. Different SBI method target different components and aspects of this fundamental statistical relationship:

\begin{itemize}
   \item \textbf{Direct posterior estimation (NPE):} Learn $q_\phi(\btheta \mid \bx) \approx p(\btheta \mid \bx)$ directly, bypassing the need to model individual components.  This enables end-to-end (E2E) learning of information maximizing feature extraction networks, fast sampling and explicit evaluation of approximate posterior densities. See Sec.~\ref{sec:methods:density}.
   
   \item \textbf{Likelihood estimation (NLE):} Learn $q_\phi(\bx \mid \btheta) \approx p(\bx \mid \btheta)$, then apply Bayes' rule with the known prior to obtain the posterior.  This enables, fast sampling of approximate simulated data and explicit likelihood evaluation. Posterior sampling has to be done with, \fex, MCMC, and E2E learning of feature extractors is not possible. See Sec.~\ref{sec:methods:density}.
   
   \item \textbf{Ratio estimation (NRE):} Learn the likelihood-to-evidence ratio $r_\phi(\btheta, \bx) \approx p(\bx \mid \btheta)/p(\bx)$, which equals the posterior-to-prior ratio. It enables E2E learning of feature extractors, but requires posterior sampling via MCMC.  See Sec.~\ref{sec:methods:ratio}.
   
   \item \textbf{Score-based methods (SSM):} Learn quantities related to the score function $\nabla_{\btheta} \log p(\btheta \mid \bx)$, which defines the posterior up to normalization. It enables E2E learning of feature extractors, and sampling via Langevin dynamics-related methods. See Sec.~\ref{sec:core_score}
   
\end{itemize}

Each approach makes different trade-offs between computational efficiency, architectural requirements, and the type of access they provide to the posterior, as listed in Tab.~\ref{tab:sbi_method_comparison}.
The following subsections explore each approach in detail.



\begin{table}[h!]
\centering
\small
\renewcommand{\arraystretch}{1.3}
\begin{tabular}{@{}p{1.4cm}p{3.0cm}p{3.3cm}p{1.8cm}p{3.2cm}@{}}
\toprule
\textbf{Method} &
\textbf{Training target} &
\textbf{Posterior access} &
\textbf{E2E summary} &
\textbf{Loss function} \\
\midrule
\textbf{NPE} &
$q_\phi(\theta \mid x)$ &
Direct (explicit posterior) &
\faCheck\ Yes &
Negative log-likelihood: $-\log q_\phi(\theta \mid x)$ \\
\textbf{NLE} &
$q_\phi(x \mid \theta)$ &
Via Bayes' rule: use prior and normalizer &
\faTimes\ No or pretrained &
Negative log-likelihood: $-\log q_\phi(x \mid \theta)$ \\
\textbf{NRE} &
$r_\phi(\theta, x) \approx \frac{p(x \mid \theta)}{p(x)}$ &
Via rejection sampling or importance weighting &
\faCheck\ Yes&
Binary cross-entropy (classification loss) \\
\textbf{SSM} &
Score $\nabla_x \log p(x)$ &
Indirect; posterior via integration or Langevin sampling &
\faCheck\ Yes&
Score-matching loss or denoising loss \\
\textbf{ABC} &
None (sample filtering) &
Via accepted samples only (implicit) &
\faTimes\ No or pretrained&
Accept/reject threshold on $d(T(x), T(x_0))$ \\
\bottomrule
\end{tabular}
\caption{Comparison of key simulation-based inference methods in terms of what they learn, how they access the posterior, whether they support end-to-end summary learning, and what loss function they use.}
\label{tab:sbi_method_comparison}
\end{table}



\section{Density Estimation}
\label{sec:methods:density}


\subsection{Neural Posterior Estimation}
\label{sec:methods:density:npe}


\begin{algorithm}[t]
\begin{algorithmic}[1]
\State \textbf{Input:}
\State \hspace{\algorithmicindent} Implicit prior distribution $p(\btheta)$
\State \hspace{\algorithmicindent} Implicit likelihood model $p(\bx \mid \btheta)$
\State \hspace{\algorithmicindent} Neural density estimator $q_\phi(\btheta \mid \bx)$
\State \hspace{\algorithmicindent} Number of simulations $N$
\State \textbf{Output:} Learned posterior $q_\phi(\btheta \mid \bx)$
\State
\State Initialize training dataset $\mathcal{D} \gets \emptyset$
\For{$i = 1$ to $N$}
    \State Sample $\btheta^{(i)} \sim p(\btheta)$
    \State Simulate $\bx^{(i)} \sim p(\bx \mid \btheta^{(i)})$
    \State Add the pair $(\bx^{(i)}, \btheta^{(i)})$ to the training dataset $\mathcal{D}$
\EndFor
\State Initialize neural network $q_\phi(\btheta \mid \bx)$ with parameters $\phi$
\While{not converged}
    \State Sample a mini-batch $\mathcal{B} \subset \mathcal{D}$
    \State Compute the loss:
    \State \hspace{\algorithmicindent} 
    $\mathcal{L}_\text{NPE}[\phi, \mathcal{B}] = -\frac{1}{|\mathcal{B}|} \sum_{(\bx, \btheta) \in \mathcal{B}} \log q_{\phi}(\btheta \mid \bx)$
    \State Update parameters $\phi$ via gradient descent on $\mathcal{L}_\text{NPE}$
\EndWhile
\State \textbf{Return} Trained network $q_\phi(\btheta \mid \bx)$
\end{algorithmic}
\caption{Neural Posterior Estimation (NPE). The algorithm trains a conditional density estimator $q_\phi(\btheta \mid \bx) \approx p(\btheta \mid \bx)$, amortized over $\bx \sim p(\bx)$. 
Neural Likelihood Estimation (NLE) is realised by replacing $q_\phi(\btheta \mid \bx)$ with $q_\phi(\bx \mid \btheta)$.
Validation tests are not made explicit for breivty. 
}
\label{alg:NPE}
\end{algorithm}

Neural Posterior Estimation (NPE) is one of the most widely used approaches for amortized simulation-based inference. Its objective is to approximate the intractable posterior distribution $p(\btheta \mid \bx)$ with a flexible conditional density estimator $q_\phi(\btheta \mid \bx) \approx p(\btheta \mid \bx)$. The algorithm is simple to implement and interpret, but requires specialized network architectures---such as the normalising flow models that we will discuss in Sec.~\ref{sec:methods:density:flows}---that represent probability densities throughout training.

The foundations of this approach can be traced to early work on conditional neural density estimation in the context of Sequential Monte Carlo and explicit graphical models~\cite{gu_neural_2015, paige_inference_2016, Le2016InferenceCompilationUniversal}, building on even earlier ideas for neural network-based uncertainty quantification~\citep{nix_estimating_1994}. A key breakthrough was the recognition that these techniques could be used to perform Bayesian posterior estimation for \emph{implicit} likelihood models---simulators where the likelihood cannot be evaluated directly---both in sequential~\cite{Papamakarios2016FastFreeInference} and fully amortized~\cite{Ambrogioni2018ForwardAmortizedInference} settings.




\paragraph{Training objective.}

As in Approximate Bayesian Computation (ABC), training data in NPE consist of independent samples from the joint generative model \( p(\bx, \btheta) = p(\bx \mid \btheta) p(\btheta) \). These samples are obtained by first drawing \( \btheta \sim p(\btheta) \), and then simulating \( \bx \sim p(\bx \mid \btheta) \). Generating \( N \) samples in this way yields a training dataset
\[
\mathcal{D} = \{(\boldsymbol{\theta}_i, \mathbf{x}_i)\}_{i=1}^N\,, \quad
(\btheta_i, \bx_i) \iidsim p(\bx, \btheta)\;.
\]
In practice, the dataset \( \mathcal{D} \) is typically split into training and validation sets as described in Sec.~\ref{sec:methods:nn:mlps}.

The training objective is to maximize the probability of the parameters \( \btheta \) under the conditional model \( q_\phi(\btheta \mid \bx) \), evaluated on the training data. This leads to the following loss function:
\begin{equation}
\mathcal{L}_\text{NPE} = - \frac{1}{|\mathcal{D}|}
\sum_{(\btheta, \bx) \in \mathcal{D}} \log q_\phi(\btheta \mid \bx)\;.
\label{eqn:NPE_loss}
\end{equation}
In the limit of infinite training data \( N \to \infty \), the empirical loss converges to an expectation over the joint data distribution:
\begin{equation}
\mathcal{L}_\text{NPE} \underset{N \to \infty}{\longrightarrow}
\mathbb{E}_{\bx \sim p(\bx)}
\left\Big[
D_{\text{KL}}(p(\btheta \mid \bx) \, \| \, q_\phi(\btheta \mid \bx))
+
\mathcal{H}(p(\btheta \mid \bx))
\right\Big].
\label{eqn:NPE_limit}
\end{equation}
Since the KL divergence is non-negative and vanishes only when \( q_\phi(\btheta \mid \bx) = p(\btheta \mid \bx) \), it is evident that the loss is minimized precisely when the learned posterior matches the true posterior for all \( \bx \). In practice, how closely this goal is achieved depends on the network architecture, optimization method, data representation, and training strategy. The full training procedure is summarized in Alg.~\ref{alg:NPE}.

As noted above, the use of NPE requires density estimators that produce normalized, non-negative outputs for all input values. Normalizing flows are a common choice, as discussed in Sec.~\ref{sec:methods:density:flows}, since they allow for tractable density evaluation and efficient sampling.

\paragraph{Learning information-maximizing summaries.}

A key strength of NPE is that it enables the joint learning of data summaries and posterior estimation. In practice, many architectures used for NPE include a summary network \( T_\phi(\bx) \), which maps high-dimensional observations to a lower-dimensional representation. The density estimator then operates on this compressed form: \( q_\phi(\btheta \mid \bx) \to q_\phi(\btheta \mid T_\phi(\bx)) \).

In this setup, Eq.~\eqref{eqn:NPE_limit} implies\footnote{This follows from the non-negativity of the KL divergence. The reasoning is analogous to the derivation of the Evidence Lower Bound (ELBO) in variational inference.} that minimizing the NPE loss also minimizes an upper bound on the expected posterior entropy conditioned on the summary:
\[
\mathcal{L}_{\text{NPE}} 
\;\geq\;
\mathbb{E}_{p(\bx)}
\mathcal{H}(p(\btheta \mid T_\phi(\bx)))
\;\geq\;
\mathbb{E}_{p(\bx)}
\mathcal{H}(p(\btheta \mid \bx))\;.
\]
The bound becomes tight when the conditional estimator is exact, \( q_\phi(\btheta \mid T_\phi(\bx)) = p(\btheta \mid T_\phi(\bx)) \). Therefore, optimizing the summary network \( T_\phi(\bx) \) end-to-end encourages summaries that minimize posterior uncertainty.

Using the identity
\begin{equation}
\mathbb{E}_{p(\bx)}
\mathcal{H}(p(\btheta \mid T_\phi(\bx)))
= \mathcal{H}(p(\btheta)) - \mathcal{I}(\btheta; T_\phi(\bx))\;,
\label{eqn:NPE_MI}
\end{equation}
it follows that minimizing posterior entropy is equivalent to maximizing the mutual information between \( \btheta \) and the summary \( T_\phi(\bx) \). In this sense, the learned summaries are \emph{information-maximizing}.

This property is one of the main practical advantages of NPE, but also a potential limitation: if no efficient summary representation exists for the problem, posterior accuracy may be fundamentally limited.


\paragraph{The amortization trade-off: generality vs efficiency.}

As seen in Eq.~\eqref{eqn:NPE_limit}, the loss function used in neural posterior estimation (NPE) involves an average over simulated observations, \( \bx \sim p(\bx) \). As a consequence, the trained posterior estimator \( q_\phi(\btheta \mid \bx) \) is amortized over the full support of the data distribution. That is, it learns to approximate \( p(\btheta \mid \bx) \) for any \( \bx \in \Omega \), rather than being tailored to a single observation. This amortization can be powerful: it enables \emph{fast inference at test time for arbitrary future observations}. However, in many scientific applications one is interested in the analysis of a single specific observation \( \bxobs \), and full amortization may be \emph{unnecessarily expensive}.

\cw{TODO: Add example for enforced amortization as virtue}

\smallskip

To see this more concretely, consider the ideal—but intractable—objective of directly minimizing the negative log-likelihood over the posterior of interest:
\[
\mathcal{L}[q_\phi] = \mathbb{E}_{\btheta \sim p(\btheta \mid \bxobs)} \left[ -\log q_\phi(\btheta) \right].
\]
This objective would lead to a non-amortized density estimator for \( p(\btheta \mid \bxobs) \). However, it is not practical, since it requires access to posterior samples, i.e., to the solution of the very inference problem we wish to solve. The strategy in NPE and related methods is to replace this intractable loss with a tractable, amortized variant:
\[
\mathcal{L}[q_\phi] = \mathbb{E}_{\bx \sim p(\bx)} \, \mathbb{E}_{\btheta \sim p(\btheta \mid \bx)} \left[ -\log q_\phi(\btheta \mid \bx) \right].
\]
This formulation removes the need for posterior samples and instead leverages simulator access to sample from the joint distribution \( p(\bx, \btheta) = p(\bx \mid \btheta)\,p(\btheta) \). In doing so, we eliminate the original sampling bottleneck—\emph{but at the cost of training a powerful conditional density estimator} that must generalize across the full support of \( p(\bx) \). \emph{The difficulty of the original inference problem is not avoided, but transferred: from sampling to learning a conditional flow.}

\medskip

This enforced amortization is a defining feature of simulation-based inference (SBI), and contrasts with most likelihood-based inference (LBI) approaches, where amortization is usually absent or optional (e.g., in variational autoencoders).

Whether this amortization is \emph{a virtue or a limitation} depends on the use case. In applications requiring fast inference for many observations, amortization is essential. But \emph{in situations where only a single observation is relevant, this generality may lead to disproportionate computational demands} in terms of both simulation effort and model capacity. Sequential methods to alleviate this issue will be discussed in Sec.~\ref{sec:sequential_sbi}.


\subsection{Modeling Probability Density Functions with Normalizing Flows}
\label{sec:methods:density:flows}

The success of NPE depends critically on the choice of neural architecture for the conditional density estimator $q_\phi(\btheta \mid \bx)$. As we will see, this is not a trivial modeling challenge.

Neural networks that directly model probability density functions must satisfy strict constraints: their outputs must be non-negative, and the density distribution must integrate to one,
%
\begin{equation}
q_\phi(\btheta \mid \bx ) \geq 0\,, \quad \text{and} \quad
\int d\btheta\, q_\phi(\btheta \mid \bx) = 1 \quad
\forall \quad \bx \in \mathcal{X}\;.
\label{eqn:normalization}
\end{equation}
%
Ideally, they should also support efficient sampling from the learned distribution.

\medskip

\emph{Normalizing flows} satify the above criteria, and can approximate complex, multimodal distributions while maintaining exact probability density evaluation and efficient sampling  \citep[for a recent review see][]{papamakarios_normalizing_2021}. A normalizing flow transforms a simple base distribution \( p_Z(\mathbf{z}) \)—typically a standard Gaussian in \( \mathbb{R}^d \)—into a complex target distribution \( p(\btheta \mid \bx) \) by applying a sequence of invertible and differentiable mappings. Each transformation is of the form \( \mathbf{z}_l = f_{\phi, \bx}^{(l)}(\mathbf{z}_{l-1}) \), and parametrised through network parameters $\phi$.  In the context of SBI, it takes as input besides the transformation variable $\bz_{l-1}$ also the data $\bx$ that we condition on. 

The full flow is defined as a sequence of applications of the transformation functions onto a sample from the base distribution,
\[
\btheta = f_{\phi, \bx}^{(L)} \circ \cdots \circ f_{\phi, \bx}^{(1)}(\mathbf{z}_0), \quad \mathbf{z}_0 \sim p_Z(\bz)\;.
\]
%
Using the change-of-variables formula, the model density \( q_\phi(\btheta \mid \bx) \) can be computed as
\[
q_\phi(\btheta \mid \bx) = p_Z(\mathbf{z}_0) \prod_{l=1}^{L} \left| \det \left( \frac{\partial f^{(l)}_{\phi, \bx}}{\partial \mathbf{z}_{l-1}} \right)^{-1} \right|.
\]
This formula enables exact likelihood evaluation, provided that each transformation \( f_l \) is both invertible and has a Jacobian determinant that is efficient to compute.
Typical choices for the component functions \( f_l \) include \emph{affine coupling layers}, which update only a subset of variables conditioned on the others to yield triangular Jacobians, and \emph{autoregressive transforms}, which model each output dimension sequentially as a function of previous ones. These design choices ensure computational efficiency in both forward and inverse passes, as well as tractability of the log-determinant.

On advantage of flow-based models is that sampling is straightforward and highly efficient: one simply draws \( \mathbf{z}_0 \sim p_Z \) and computes \( \mathbf{x} = f(\mathbf{z}_0) \) by pushing the sample forward through the flow. Conversely, to evaluate the likelihood density \( p_X(\mathbf{x}) \), one must compute the inverse transformation \( \mathbf{z}_0 = f^{-1}(\mathbf{x}) \), which requires that each \( f_l \) be easily invertible in practice.

\medskip

For particularly complex posterior geometries, discrete flows may require many layers to achieve adequate flexibility. This motivates considering the continuous limit of the transformation process.
In \emph{continuous normalizing flows} (CNFs), the transformation is modeled as the continuous evolution of a latent variable $\mathbf{z}(t) \in \mathbb{R}^d$ under a learned vector field. Starting from $\mathbf{z}(0) \sim p_Z$, the transformed variable $\btheta \equiv \mathbf{z}(T)$ is obtained by solving the ordinary differential equation (ODE)
\[
\frac{d}{dt}\mathbf{z}(\mathbf{x}, t) = \mathbf{u}_{\phi, \mathbf{x}}^{(t)}(\mathbf{z}(t)), \quad \text{with} \quad \mathbf{z}(0) \sim p_Z.
\]
The log-density evolves according to the continuity equation:
\[
\frac{d}{dt} \log q_\phi^{(t)}(\mathbf{z}(t) \mid \mathbf{x}) = -\nabla_{\mathbf{z}} \cdot \mathbf{u}_{\phi,\mathbf{x}}^{(t)}(\mathbf{z}(t))\;.
\]
Evaluation of CNFs, $q_\phi(\btheta \mid \bx) \equiv q_\phi^{(T)}(\btheta \mid \bx)$ requires solution of both ODEs.  The vector field $\mathbf{u}_{\phi, \bx}^{(t)}(\btheta(t))$ can be then trained through the regular NPE loss, Eq.~\eqref{eqn:NPE_loss}.


\subsection{Neural Likelihood Estimation}
\label{sec:methods:density:nle}

Neural likelihood estimation (NLE) aims to approximate the data likelihood \( p(\bx \mid \btheta) \) using a conditional density estimator \( q_\phi(\bx \mid \btheta) \approx p(\bx \mid \btheta) \)~\citep{Papamakarios2018SequentialNeuralLikelihood, Alsing2018MassiveOptimalData}.
Unlike NPE, this approach does not depend on the prior \( p(\btheta) \), making it attractive in settings where the prior is either unknown, subject to change, or when the goal is frequentist rather than Bayesian inference. Applications include hypothesis testing, model comparison via likelihood ratios, or posterior inference via likelihood-based techniques.

As in NPE, training data consist of samples from the joint distribution \( p(\btheta, \bx) = p(\btheta)\, p(\bx \mid \btheta) \), obtained by sampling parameters \( \btheta \sim p(\btheta) \) and simulating observations \( \bx \sim p(\bx \mid \btheta) \). The density estimator \( q_\phi(\bx \mid \btheta) \) is then trained by maximizing the likelihood of the observations under the model. This corresponds to minimizing the negative log-likelihood loss:
\[
\mathcal{L}_\text{NLE} = - \frac{1}{|\mathcal{D}|} \sum_{(\btheta, \bx) \in \mathcal{D}} \log q_\phi(\bx \mid \btheta).
\]

In the limit of infinite training data, the empirical loss converges to the expected KL divergence between the true and approximate likelihoods, plus the entropy of the data:
\begin{equation}
\mathcal{L}_\text{NLE} \underset{N\to\infty}{\longrightarrow}
\mathbb{E}_{p(\btheta)} \left[
D_{\text{KL}}(p(\bx \mid \btheta) \,\|\, q_\phi(\bx \mid \btheta)) +
\mathcal{H}(p(\bx \mid \btheta))
\right].
\label{eqn:NLE_limit}
\end{equation}
As in the case of NPE, the loss is minimized when \( q_\phi(\bx \mid \btheta) = p(\bx \mid \btheta) \) for all \( \btheta \).

After training, the learned likelihood model can be used in different ways. In the Bayesian context, one can generate posterior samples via Markov Chain Monte Carlo (MCMC), using Bayes’ theorem with the learned likelihood:
\[
\btheta \overset{\text{MCMC}}{\sim} \frac{1}{Z(\bx)} \, q_\phi(\bx \mid \btheta) \, p(\btheta).
\]
Alternatively, the likelihood itself can be used for frequentist-style inference, such as likelihood ratio tests or goodness-of-fit evaluations.

Like NPE, NLE requires density estimators that produce normalized, non-negative outputs. Normalizing flows are commonly used for this purpose. However, a key difference is that NLE does not support end-to-end learning of data summaries. Since the model directly approximates the full data distribution, replacing \( \bx \) with a learnable summary \( T_\phi(\bx) \) would allow the network to collapse the representation to a constant value—artificially minimizing the entropy term in the loss. As a result, any dimensionality reduction or summary construction must be performed separately in advance.

Finally, NLE can be a valuable complement to NPE. For example, it enables posterior predictive checks or cross-validation of the posterior obtained via NPE by comparing it against samples drawn from the learned likelihood via MCMC.



\section{Density Ratio Estimation}
\label{sec:methods:ratio}

\begin{algorithm}[t]
\caption{Neural Ratio Estimation (NRE). The algorithm trains a network to approximate the log ratio, $f_\phi(\bx; \btheta) \approx \log r(\bx; \btheta)$, amortized over data $\bx \sim p(\bx)$.
}\label{alg:NRE}
\begin{algorithmic}[1]
\State \textbf{Input:}
%\State \hspace{\algorithmicindent} Observed data $\mathbf{D}_{obs}$
\State \hspace{\algorithmicindent} Prior distribution $\p(\btheta)$
\State \hspace{\algorithmicindent} Likelihood model $p(\bx \mid \btheta)$
\State \hspace{\algorithmicindent} Neural network architecture, $f_\phi(\bx; \btheta)$
\State \hspace{\algorithmicindent} Number of simulations $N$
\State \textbf{Output:} Approximate likelihood-to-evidence ratio, $f_\phi(\bx; \btheta) \approx \log r(\bx; \btheta)$
\State
\State Initialize training dataset $\mathcal{D} \gets \emptyset$
\For{$i = 1$ to $N$}
    \State Sample $\btheta^{(i)}$ from the prior $p(\btheta)$
    \State Generate synthetic data $\bx^{(i)}$ from the model $p(\bx \mid \btheta)$
    \State Add the pair $(\bx^{(i)}, \btheta^{(i)})$ to the training dataset $\mathcal{D}$
\EndFor
\State Initialize neural network $f_\phi(\bx; \btheta)$ with parameters $\phi$
\While{not converged}
    \State Sample a mini-batch $\mathcal{B}$ from $\mathcal{D}$
    \State Sample a mini-batch $\mathcal{B}_c$ from $\mathcal{D}_c$ \Comment $\mathcal{B}_c$ can be obtained by scrambling $\mathcal{B}$
%    \For{each $(\bx, \btheta)$ in $\mathcal{B}$}
%        \State Compute $f_\phi(\bx; \btheta)$
%        \State Compute $f_\phi(\bx; \btheta)$
        \State Compute the binary classification loss:
        \State \hspace{\algorithmicindent} $\mathcal{L}_\text{NLE}
        %(\phi, \mathcal{B}, \mathcal{B}_c) 
        = 
- \frac1{|\mathcal{B}|}\sum_{\btheta, \bx \in \mathcal{B}} \log \sigma \left(f_\phi(\bx; \btheta) \right) 
- 
\frac1{|\mathcal{B}_c|}\sum_{\btheta, \bx \in \mathcal{B}_c} \log \sigma \left(-f_\phi(\bx; \btheta) \right)$
    \State Update $\phi$ using gradient descent on accumulated loss $\mathcal{L}_\text{NLE}(\phi, \mathcal{B}, \mathcal{B}_c)$
\EndWhile

\State \textbf{Return} Trained neural network $f_\phi(\bx; \btheta)$, representing $\log r(\bx; \btheta)$
\end{algorithmic}
\end{algorithm}

Neural ratio estimation (NRE) is a class of simulation-based inference techniques that, instead of learning full densities, focuses on estimating \emph{density ratios}. Such ratios naturally arise in classification tasks, where the optimal Bayesian classifier is determined by the ratio of class-conditional densities. Density ratios are also of interest in their own right---for instance, in likelihood ratio tests in a frequentist setting, or for obtaining indirect information about unknown densities by comparing them to known ones.

The approach of estimating density ratios for Bayesian inference predates the use of neural networks~\cite{izbicki_high-dimensional_2014, dinev_dynamic_2018, thomas_likelihood-free_2020}. Early neural implementations focused on likelihood ratios for frequentist inference, estimating $p(\mathbf{x} \mid \boldsymbol{\theta})/p(\mathbf{x} \mid \boldsymbol{\theta}')$~\cite{Cranmer2015ApproximatingLikelihoodRatios}. However, recent work has shifted toward estimating likelihood-to-evidence ratios, which has been found to be numerically more stable because the contrasted distributions have overlapping support by definition~\cite{Hermans2019LikelihoodfreeMCMCAmortized, rhodes_telescoping_2020}.


\subsection{Neural Ratio Estimation}
\label{sec:methods:ratios:nre}

One of the most common and practically useful targets in this context is the \emph{likelihood-to-evidence ratio}. It is defined as the ratio between the joint distribution \( p(\bx, \btheta) = p(\bx \mid \btheta) p(\btheta) \) and the product of marginals \( p(\bx)p(\btheta) \):
\begin{equation}
r(\bx; \btheta) \equiv 
\frac{p(\bx, \btheta)}{p(\bx)p(\btheta)}
= \frac{p(\bx \mid \btheta)}{p(\bx)}
= \frac{p(\btheta \mid \bx)}{p(\btheta)}\;.
\label{eqn:lte-ratio}
\end{equation}
This ratio reveals deep connections between posterior inference, marginal likelihoods, and joint densities. In particular, when the prior \( p(\btheta) \) is known, access to the ratio \( r(\bx; \btheta) \) directly enables recovery of the posterior via Bayes' theorem.

\paragraph{Training objective.}

To estimate \( r(\bx; \btheta) \), one constructs a binary classification problem distinguishing between samples from the joint distribution and the product of marginals. The two datasets are defined as
\[
\mathcal{D} = \{(\btheta_i, \bx_i)\}_{i=1}^N\,, \quad (\btheta, \bx) \sim p(\bx, \btheta)\,,
\]
and
\[
\mathcal{D}_c = \{(\btheta_i, \bx_i)\}_{i=1}^N\,, \quad (\btheta, \bx) \sim p(\btheta) p(\bx)\,.
\]
Samples from the product distribution can be constructed by drawing two independent prior samples \( \btheta, \btheta' \sim p(\btheta) \), simulating \( \bx \sim p(\bx \mid \btheta') \), and then discarding \( \btheta' \). In practice, a common shortcut is to generate \( \mathcal{D}_c \) by shuffling parameter–data pairs from \( \mathcal{D} \), creating mismatched pairs \( (\btheta_i, \bx_j) \) with \( i \neq j \).

A standard loss for this classification task is the binary cross-entropy (BCE) loss.\footnote{
Training is most commonly based on minimising the binary cross-entropy loss. However, multiple statistically equivalent options exist, that can be numerically better behaved in some cases \citep{jeffrey_evidence_2024, rizvi_learning_2024}.}
Using a neural network \( f_\phi(\bx; \btheta) \in \mathbb{R} \) and sigmoid activation \( \sigma(\cdot) \), the loss becomes
\begin{equation}
\label{eqn:NRE_loss}
\mathcal{L}_\text{NRE} = 
- \frac{1}{|\mathcal{D}|} \sum_{(\bx, \btheta) \in \mathcal{D}} \log \sigma \left(f_\phi(\bx; \btheta) \right)
- \frac{1}{|\mathcal{D}_c|} \sum_{(\bx, \btheta) \in \mathcal{D}_c}
\log \sigma \left(-f_\phi(\bx; \btheta) \right)\;.
\end{equation}
The network \( f_\phi \) takes both \( \bx \) and \( \btheta \) as input and outputs a scalar score. The full training procedure is summarized in Alg.~\ref{alg:NRE}.

\paragraph{Asymptotic behavior and posterior recovery.}

In the limit of infinite training data, the BCE loss can be written as
\begin{multline}
\mathcal{L}_\text{NRE} \underset{N \to \infty}{\longrightarrow}
\mathbb{E}_{p(\bx, \btheta)}
\log \frac{1 + e^{-f_\phi(\bx; \btheta)}}{1 + r^{-1}(\bx; \btheta)}
+
\mathbb{E}_{p(\bx)p(\btheta)}
\log \frac{1 + e^{f_\phi(\bx; \btheta)}}{1 + r(\bx; \btheta)}
\\
-
\mathbb{E}_{p(\bx)} D_{\text{JS}}(p(\btheta) \,\|\, p(\btheta \mid \bx))\;.
\label{eqn:NRE_long}
\end{multline}
The first two terms are non-negative and vanish when \( f_\phi(\bx; \btheta) = \log r(\bx; \btheta) \), i.e., when the classifier recovers the log likelihood-to-evidence ratio. The final term corresponds to the negative expected Jensen-Shannon divergence between the prior and posterior.

Once the network has been trained, it implicitly defines an energy-based model for the posterior:
\[
q_\phi(\btheta \mid \bx) = \frac{1}{Z(\bx)} \, e^{f_\phi(\bx; \btheta)} \, p(\btheta),
\]
where \( Z(\bx) \) is an unknown normalizing constant. (An energy-based model refers to any model of the form \( q(\btheta) \propto e^{-E(\btheta)} \), which defines a distribution up to a normalization constant.) In NRE, the value of \( Z(\bx) \) is typically close to 1 and is irrelevant in practice, since posterior sampling is performed using MCMC methods—just as in NLE in Sec.~\ref{sec:methods:density:nle}.

\paragraph{Comparison with density-based approaches.}

Unlike NPE and NLE, NRE does not require modeling normalized probability densities. Any expressive architecture (e.g., MLP) can be used to represent the classifier \( f_\phi(\bx; \btheta) \).

NRE shares properties of both NLE and NPE. Like NLE, it is prior-independent. This is evident from Eq.~\eqref{eqn:lte-ratio}, where the prior cancels. Like NPE, NRE also supports end-to-end learning of data summaries. In fact, one can show that minimizing the BCE loss implicitly maximizes the expected Jensen-Shannon divergence between posterior and prior:
\[
\mathcal{L}_\text{NRE} \geq 
- \mathbb{E}_{p(\bx)} D_{\text{JS}}(p(\btheta) \,\|\, p(\btheta \mid T_\phi(\bx)))
\geq 
- \mathbb{E}_{p(\bx)} D_{\text{JS}}(p(\btheta) \,\|\, p(\btheta \mid \bx)).
\]
In this sense, the learned summaries \( T_\phi(\bx) \) are information-maximizing—though with respect to JS divergence, not entropy as in NPE.

This difference leads to a practical limitation. The Jensen-Shannon divergence is bounded above by \( \log 2 \), so if the posterior is much sharper than the prior—as often occurs in high-dimensional settings—the divergence saturates. As a result, the \emph{gradient signal for the summary network \( T_\phi(\bx) \)} becomes weak. In classification terms, this corresponds to the decision boundary between joint and product-of-marginals becoming too confidently separated: when \( r(\bx; \btheta) \gg 1 \) or \( \ll 1 \), the classifier assigns near-certain labels, and the loss becomes nearly flat with respect to changes in \( f_\phi \). Consequently, updates to the summary network vanish, hindering further improvement.

As a result, NRE works well for low-dimensional inference problems, especially for estimating 1D or 2D marginal posteriors or building summary networks. In high dimensions, however, it generally struggles to capture the full joint posterior accurately. Autoregressive extensions have been proposed to mitigate this limitation~\cite{us}.


\subsection{Likelihood-Ratio Estimation}
\label{sec:methods:ratios:lre}

\cw{TODO: Drop this section? Or keep it for pedagotical reasons?}

Likelihood-ratio estimation (LRE) is one of the earliest simulation-based inference strategies to be explored~\citep{Cranmer2015ApproximatingLikelihoodRatios}. It is particularly appealing in frequentist settings, where hypothesis testing often reduces to the comparison of likelihoods under different parameter values. The goal is to estimate the likelihood ratio
\[
r(\bx; \btheta_1, \btheta_2) = \frac{p(\bx \mid \btheta_1)}{p(\bx \mid \btheta_2)},
\]
using a neural network that takes as input the tuple \( (\bx, \btheta_1, \btheta_2) \) and returns an estimate of the logarithmic ratio.

As in neural ratio estimation (NRE), training is framed as a binary classification problem. Samples are generated from two simulation runs: one using parameter \( \btheta_1 \), and the other using \( \btheta_2 \). The task of the network is to distinguish whether a given observation \( \bx \) was drawn from \( p(\bx \mid \btheta_1) \) (label 1) or from \( p(\bx \mid \btheta_2) \) (label 0). The network \( f_\phi(\bx, \btheta_1, \btheta_2) \in \mathbb{R} \) is trained using the binary cross-entropy loss,
\[
\mathcal{L}_\text{LRE} = 
- \mathbb{E}_{\bx \sim p(\bx \mid \btheta_1)} \log \sigma(f_\phi(\bx, \btheta_1, \btheta_2))
- \mathbb{E}_{\bx \sim p(\bx \mid \btheta_2)} \log \sigma(-f_\phi(\bx, \btheta_1, \btheta_2))\;,
\]
where \( \sigma(\cdot) \) denotes the sigmoid function. At convergence, the optimal classifier satisfies
\[
f_\phi(\bx, \btheta_1, \btheta_2) \approx \log \frac{p(\bx \mid \btheta_1)}{p(\bx \mid \btheta_2)}\;,
\]
thus directly estimating the log-likelihood ratio of interest.

While conceptually straightforward and prior-independent, LRE presents practical challenges not encountered in NRE. In NRE, the network distinguishes between \( p(\bx \mid \btheta) \) and the marginal \( p(\bx) \), which guarantees full support overlap. In contrast, LRE compares two potentially disjoint conditionals. If the supports of \( p(\bx \mid \btheta_1) \) and \( p(\bx \mid \btheta_2) \) do not overlap significantly, the classifier becomes overconfident, assigning near-certain labels regardless of input. This leads to gradient saturation and poor training dynamics, especially for high-dimensional or sharply peaked posteriors.
This can be mitigated by estimating ratios between intermediate parameters with overlapping support and chaining them together as a telescoping product.

%\section{Model Comparison in Simulation-Based Inference.}

In many applications, we wish not only to assess the adequacy of a single simulator, but to compare competing models and quantify which provides a better explanation of the observed data. In the Bayesian setting, this is formalized via the \emph{Bayes factor}, often denoted \( \mathcal{K}_{01} \), and defined as
\[
\mathcal{K}_{01}(\bxobs) = \frac{p(\bxobs \mid \mathcal{H}_0)}{p(\bxobs \mid \mathcal{H}_1)} \,,
\]
where \( \mathcal{H}_0 \) and \( \mathcal{H}_1 \) are competing models, and the model evidence or marginal likelihood is given by
\[
p(\bx \mid \mathcal{H}_j) = \int p(\bx \mid \btheta, \mathcal{H}_j)\, p_j(\btheta)\, d\btheta \,.
\]
In simulation-based inference, these marginal likelihoods are typically intractable, so exact Bayes factor evaluation is not possible.

A practical alternative is to learn a discriminative classifier to distinguish between samples from the models. Given simulators for \( \mathcal{H}_0 \) and \( \mathcal{H}_1 \), we train a scalar function \( f_\phi(\bx) \in \mathbb{R} \) to approximate the log-density ratio,
\[
f_\phi(\bx) \approx \log \frac{p(\bx \mid \mathcal{H}_1)}{p(\bx \mid \mathcal{H}_0)} = - \log \mathcal{K}_{01}(\bx) \,.
\]
The sign of \( f_\phi(\bx) \) indicates which model is preferred, and its magnitude reflects the strength of evidence.

In practice, \( \bx \) may first be mapped through a learnable summary network \( T_\phi(\bx) \), and the classifier is applied to this representation. The effectiveness of the comparison then hinges on how well \( T_\phi(\bx) \) captures model-distinguishing features. A natural loss for training \( T_\phi \) is the \emph{binary cross-entropy} (BCE),
\[
\mathcal{L}_\mathrm{BCE} = \mathbb{E}_{\bx \sim p_1} \log\left(1 + e^{-f_\phi(T_\phi(\bx))} \right) + \mathbb{E}_{\bx \sim p_0} \log\left(1 + e^{f_\phi(T_\phi(\bx))} \right) \,,
\]
which is minimized when \( f_\phi(\bx) \) approximates the log-density ratio. This loss corresponds to the logit form of standard neural ratio estimation (NRE).

\medskip

It is instructive to examine the gradient signal used to update the summary representation. Assuming that \( f_\phi \) has been trained to optimality, the update to the summary network is given by
\[
\frac{\partial \mathcal{L}_\mathrm{BCE}}{\partial T_\phi(\bx)} = \sigma(\mp f_\phi(\bx)) \cdot \frac{\partial f_\phi(\bx)}{\partial T_\phi(\bx)} \,,
\]
where the sign \( \mp \) corresponds to the class label (i.e., \( \bx \sim p_1 \) or \( \bx \sim p_0 \)), and \( \sigma(f) = (1 + e^{-f})^{-1} \) is the sigmoid. As \( f_\phi(\bx) \to \pm \infty \), the sigmoid saturates and the gradient decays exponentially, leading to vanishing learning signal for the summary network \( T_\phi \). This limits the ability to further refine representations when the Bayes factor becomes large—i.e., when the models are already well-separated.

\medskip

To address this limitation, the \emph{POP-III loss}~\cite{popiii2023} modifies the BCE objective to maintain informative gradients even for extreme evidence ratios. It is defined as
\[
\mathcal{L}_\mathrm{POP\text{-}III} = \mathbb{E}_{\bx \sim p_1} \left[ \left( \log(1 + e^{-f_\phi(T_\phi(\bx))}) \right)^2 \right]
+ \mathbb{E}_{\bx \sim p_0} \left[ \left( \log(1 + e^{f_\phi(T_\phi(\bx))}) \right)^2 \right] \,.
\]
This loss defines the squared logit as a test statistic,
\[
t_\mathrm{POP\text{-}III}(\bx) = f_\phi(T_\phi(\bx))^2 \,,
\]
and yields the gradient
\[
\frac{\partial \mathcal{L}_\mathrm{POP\text{-}III}}{\partial T_\phi(\bx)} = \pm 2 f_\phi(\bx) \cdot \sigma(\mp f_\phi(\bx)) \cdot \frac{\partial f_\phi(\bx)}{\partial T_\phi(\bx)} \,.
\]
Here, the multiplicative factor \( f_\phi(\bx) \) ensures that the gradient does not vanish for large values of the log-density ratio, allowing the summary network to keep improving representations even in the presence of extreme Bayes factors.

\medskip

Classifier-based model comparison thus benefits from calibrated loss functions that preserve learning signals across the full range of evidence strengths. Losses like POP-III provide robust approximations to Bayes factors in simulation-based settings and enable effective training of summary networks without saturation.
%\section{Gradient-Based Techniques}
\label{sec:gradient_based_techniques}

\subsection{Flow Matching}

An alternative training strategy for continuous normalizing flows, as discussed in Sec.~\ref{sec:core_methods_flows}, was introduced in \cite{lipman_flow_2023} for generative models and adopted to the conditional structures of simulation-based inference in \cite{wildberger_flow_2023}. The advantage is that it does not require integration of ODEs during training, only at the evaluation stage. We will here briefly discuss the mechanisms and connection with NPE.

The goal is again to transform parameters $\btheta$ drawn from a simple distribution $p_0(\btheta)$ into parameters drawn from the posterior $p(\btheta \mid \bx)$, but following the ODE
\[
\frac{d}{dt}\btheta_t = \mathbf{v}_\phi(\btheta_t, \bx, t) \quad \text{with} \quad \btheta_0 \sim p_0(\btheta_0),
\]
where $\mathbf{v}_\phi$ is a learnable velocity field that is data dependent.

The continuity equation implies that the underlying density field changes like
\[
\frac{d}{dt} \log q_{\phi, t}(\btheta_t \mid \mathbf{x}, t) = -\nabla_{\btheta} \cdot \mathbf{v}_\phi(\btheta_t, \bx, t).
\]
The training goal is that the endpoint at $t=1$ approximates the full posterior:
\[
q_\phi(\btheta \mid \bx) \equiv q_{\phi,1}(\btheta_{1} \mid \bx, 1) \approx p(\btheta \mid \bx).
\]

Instead of training via maximum likelihood, like CNFs, we directly train the velocity field. To this end, one needs to explicitly define a path that connects the simple with the target distribution. There is quite some flexibility in doing this. It can be done by defining through the integral
\[
p_t(\btheta_t \mid \bx) = \int d\btheta\; p_t(\btheta_t \mid \btheta_1 = \btheta) p(\btheta \mid \bx)
\]
such that $p_0(\btheta_0 \mid \btheta_1) = p_0(\btheta_0)$ and $p_1(\btheta_1 \mid \bx) \simeq p(\btheta =\btheta_1\mid \bx)$.

Here we introduced a distortion function $p_t(\btheta_t \mid \btheta_1)$ that takes samples from the final distribution (the posterior) and redistributes them at some earlier time $t<1$. A convenient ansatz is to define a distortion function 
\[
p_t(\btheta_t \mid \btheta_1 ) = \mathcal{N}(\btheta_t \mid \alpha_t \btheta_1, \sigma_t^2 \mathbf{I})
\]
which rescales $\btheta_1$.

After defining the path, the remaining challenge is to identify the velocity field that would lead to this probability distribution path. This is generally very difficult. For the above construction, however, \cite{lipman_flow_2023} showed that knowing the vector field that generates the conditional distribution $p_t(\btheta_t \mid \btheta_1)$ is enough to know the vector field for the full one. It is not immediately obvious, but one can find through application of the continuity equation that
\[
\mathbf{u}(\btheta_t, \bx) = \mathbb{E}_{\btheta_1 \sim p(\btheta_1 \mid \btheta_t, \bx)} [\mathbf{u}(\btheta_t \mid \btheta_1)].
\]
The target velocity field is given by:
$$
\mathbf{u}(\btheta \mid \btheta_1) = \mathbb{E}[\alpha'_t \btheta_1 + \sigma'_t \boldsymbol{\varepsilon} \mid \btheta_t = \btheta]
$$

This leads to the loss function
\[
\mathcal{L}_{\text{FMPE}} = \mathbb{E}_{
t\sim \mathcal{U}(0, 1),
\btheta_1 \sim p(\btheta \mid \bx),
\btheta_t \sim p(\btheta_t \mid \btheta_1)
}
\|\mathbf{v}_\phi(\btheta_t, \bx, t) - \mathbf{u}(\btheta_t \mid \btheta_1)
\|^2.
\]

\cite{wildberger_flow_2023} show that under specific regularity conditions, there is the bound
\[
D_{\text{KL}}(p(\btheta \mid \bx) \, \| \, q_\phi(\btheta \mid \bx) ) < C \cdot \mathcal{L}_{\text{FMPE}}
\]
for some constant $C>0$. This suggests that much of the logic and empirical aspects of NPE also apply to flow matching. Like for NPE, one can train end-to-end embedding networks, evaluate log density without much cost (requires integration over the trained vector field), and perform fast sampling through forward integration.


\subsection{Score Matching}

To handle more complex distributions, stochasticity can be introduced via:
$$
d\mathbf{x} = \mathbf{u}(\mathbf{x}, t)dt + g(t)d\mathbf{W}
$$
With the standard assumption $g(t)^2 = 2\sigma_t\sigma'_t$, the drift term becomes:
$$
\mathbf{u}_{\text{diff}}(\mathbf{x}, t) = \frac{\alpha'_t}{\alpha_t}\mathbf{x} - \frac{\sigma'_t}{\sigma_t}\mathbb{E}[\boldsymbol{\varepsilon} | \mathbf{x}_t = \mathbf{x}]
$$
This shows that training to predict the noise $\boldsymbol{\varepsilon}$ suffices:
\begin{equation}
   \mathcal{L} = \mathbb{E}_{t,\mathbf{x}_1,\boldsymbol{\varepsilon}} \left[\left\|\boldsymbol{\varepsilon}_\theta(\mathbf{x}_t, t) - \boldsymbol{\varepsilon}\right\|^2\right]
\end{equation}
Sampling is performed by solving the reverse SDE from noise to data.


Using the continuity equation, the log-density evolves along trajectories according to the instantaneous change-of-variables formula (\textit{i.e.}, its infinitesimal change in time along the flow),
\cw{TODO: Restructure text such that the two ODEs are shown together, so that the connection is clear}
allowing computation of the exact likelihood via integration. CNFs enable flexible modeling of complex distributions while maintaining exact and tractable likelihoods. The dynamics \( f_\phi \) are typically parameterized by dense neural networks (MLPs) and trained by maximizing the likelihood over observed data using ODE solvers with automatic differentiation.

- dz/dt = f & d/dt log p = grad f
- loss on p(x, t=T)

- 

\medskip

\cw{TODO: Need to talk about conditional flows!!!}

\cw{TODO: Fine for now, but should be more evolved in the future}
A further generalization of continuous flows introduces stochasticity into the transformation process. Rather than following deterministic dynamics, the latent variable evolves according to a stochastic differential equation, allowing the model to inject noise and capture richer, more multimodal distributions. This extension leads to the class of \emph{diffusion models}, which have recently gained prominence in generative modeling.

In contrast to normalizing flows and CNFs---where the transformation is learned end-to-end---diffusion models define a fixed forward process that gradually corrupts data with noise, and then learn to reverse this process. The transformation is thus guided: the forward dynamics are predefined, and the learning focuses on inverting them. While diffusion models offer powerful generative capabilities, they differ fundamentally from the flow-based methods discussed here, and we will not cover them in detail.



\subsection{Denoising Score Matching}

%\cw{TODOREF - Score-based}

The Fisher divergence, Eq.~\eqref{eqn:FisherDiv}, only depends on the score function, and can in principle be used to build a simulation-based inference algorithm by focusing on the loss function
%
\begin{equation}
    \bbE_{p(x)} [D_F(q(\mathbf z \mid \mathbf x) \mid\mid  p(\mathbf z\mid \mathbf x))]
\end{equation}
%
The only thing we really did w.r.t.~Eq.~\eqref{eqn:DKL} was to replace the forward KL divergence with the Fisher divergence.  If we were able to minimize that loss, we would learn the score function of the posterior $s = \nabla_\bz \log q(\bz \mid \bx)$, which gives us access to an energy-based model of the posterior.
%
An immediate challenge for this approach is that the score function of the true posterior $p(\bz \mid \bx)$ is not known.  

\medskip

We can make however use of the following general observation.  Let us consider a noisy target model, where the implicit distribution $q(\bz)$ got convolved with some noise distribution $p(\tilde \bz \mid \bz)$.  
%
\begin{equation}
p(\tilde{\mathbf{z}})=\int d\bz\; p(\tilde{\mathbf{z}} \mid \mathbf{z}) q(\mathbf{z})\;.
\end{equation}
%
A typical example is $\tilde \bz = \bz + \boldsymbol\epsilon$, where $\boldsymbol \epsilon \sim \mathcal{N}(0, \sigma^2 \mathbb{1})$. In that case, the conditional noise distribution and its score function are tractable, $p(\tilde \bz \mid \bz) = \mathcal{N}(\tilde \bz; \bz, \sigma^2 \mathbb{1})$.

Importantly, one can now show, using integration by parts and re-arranging terms that are not directly dependent on $\phi$, that the Fisher divergence between the noisy target and a variational approximator is given by
%
\begin{equation}
\begin{aligned}
D_F\left(p(\tilde{\mathbf{z}}) \| q_{\phi}(\tilde{\mathbf{z}})\right) & =\mathbb{E}_{p(\tilde{\mathbf{z}})}\left[\frac{1}{2}\left\|\nabla_{\mathbf{z}} \log p(\tilde{\mathbf{z}})-\nabla_{\mathbf{z}} \log q_{\phi}(\tilde{\mathbf{z}})\right\|_2^2\right] \\
& =\mathbb{E}_{p(\tilde \bz \mid \bz) q(\mathbf{z} )}\left[\frac{1}{2}\left\|\nabla_{\mathbf{z}} \log p(\tilde{\mathbf{z}} \mid \mathbf{z})-\nabla_{\mathbf{z}} \log q_{\phi}(\tilde{\mathbf{z}})\right\|_2^2\right]+\text { constant }\;.
\end{aligned}
\end{equation}
%
In the last step, the intractable gradient $\nabla_\bx \log p(\tilde \bx)$ got replaced by the tractable gradient of $\nabla_\bx \log p(\tilde \bx \mid \bx)$.  Samples are jointly performed over the noisy-free and the noisy distribution.

\medskip

Using the above relation, we can now formulate a target for estimating the (score of the) posterior $p(\bz \mid \bx)$ with denoising score matching. We do this by replacing all distributions in the above equations by data-conditioned distributions, and averaging over data relatizations.
%
\begin{equation}
    \mathcal{L} = 
    \bbE_{p(\tilde \bz \mid \bz) p(\bx \mid \bz) p(\bz)}
    \left[
    \frac12
\left\|\nabla_{\mathbf{z}} \log p(\tilde{\mathbf{z}} \mid \mathbf{z})-\nabla_{\mathbf{z}} \log q_{\phi}(\tilde{\mathbf{z}} \mid \bx)\right\|_2^2
    \right]
\end{equation}
%
Minimizing that loss trains now a the score function of the posterior for the noise-added parameters $\tilde \bz$,
\begin{equation}
    \nabla_\bz \log q_\phi(\tilde \bz \mid \bx)
    \approx
    \nabla_\bz \log p(\tilde \bz \mid \bx)
\end{equation}
%
In the limit of small enough noise variance $\sigma^2$, the resulting score function will reasonably well approximate the true posterior score.

\medskip

Some additional observations.
\begin{itemize}
    \item One can show that the above optimization target becomes high variance in the limit of small $\sigma$.
    \item In contrast to neural ratio estimation, there are no constraints on the partition function $Z$.
    \item ...
\end{itemize}

\subsection{Diffusion Denoising Probabilistc Models}

\begin{equation}
\mathrm{d} \mathbf{x}_t=-\frac{1}{2} \beta(t) \mathbf{x}_t \mathrm{~d} t+\sqrt{\beta(t)} \mathrm{d} \boldsymbol{\omega}_t
\end{equation}

\begin{equation}
q_t\left(\mathbf{x}_t \mid \mathbf{x}_0\right)=\mathcal{N}\left(\mathbf{x}_t ; \gamma_t \mathbf{x}_0, \sigma_t^2 \mathbf{I}\right)
\end{equation}

\begin{equation}
\mathrm{d} \mathbf{x}_t=\left[-\frac{1}{2} \beta(t) \mathbf{x}_t-\beta(t) \nabla_{\mathbf{x}_t} \log q_t\left(\mathbf{x}_t\right)\right] \mathrm{d} t+\sqrt{\beta(t)} \mathrm{d} \overline{\boldsymbol{\omega}}_t
\end{equation}

\begin{equation}
\mathcal{L}[\boldsymbol{\theta}] =
\mathbb{E}_{t \sim \mathcal{U}(0, T)} \mathbb{E}_{\mathbf{x}_0 \sim q_0\left(\mathbf{x}_0\right)} \mathbb{E}_{\mathbf{x}_t \sim q_t\left(\mathbf{x}_t \mid \mathbf{x}_0\right)}\left\|\mathbf{s}_{\boldsymbol{\theta}}\left(\mathbf{x}_t, t\right)-\nabla_{\mathbf{x}_t} \log q_t\left(\mathbf{x}_t \mid \mathbf{x}_0\right)\right\|_2^2
\end{equation}

\begin{equation}
    \mathcal{L}[\boldsymbol{\theta}] =
    \mathbb{E}_{t \sim \mathcal{U}(0, T)} \mathbb{E}_{\mathbf{x}_0 \sim q_0\left(\mathbf{x}_0\right)} \mathbb{E}_{\boldsymbol{\epsilon} \sim \mathcal{N}(\mathbf{0}, \mathbf{I})} \frac{1}{\sigma_t^2}\left\|\boldsymbol{\epsilon}-\boldsymbol{\epsilon}_{\boldsymbol{\theta}}\left(\mathbf{x}_t, t\right)\right\|_2^2
\end{equation}


\section{Advanced Methods}
\label{sec:methods:advanced}

\cw{TODO: Write advanced method section}


\chapter{Diagnostics for Simulation-Based Inference}
\label{chap:diag}

\begin{quotation}
    \textit{``When a fail-safe system fails, it fails by failing to fail-safe.''}

    \hfill --- John Gall, 1977, The Systems Bible
\end{quotation}


\section{A Taxonomy for Epistemic Uncertainties}
\label{sec:diag:taxonomy}

To validate simulation-based inference reliably, it is crucial to distinguish between uncertainties inherent to the data-generating process and those arising from imperfections in the inference pipeline. The former, called \emph{aleatoric uncertainty}, reflects irreducible randomness in observations and remains even with perfect inference. The latter, \emph{epistemic uncertainty}, arises from limited knowledge, model mismatch, or approximation error, and can in principle be reduced through better modeling or more computation \citep{kiureghian_aleatory_2009, kendall_what_2017}.

We propose a taxonomy of epistemic uncertainties that follows the three principal components of SBI (Fig.~\ref{fig:sbi_overview}), though the underlying failure modes apply to statistical inference more broadly. While \emph{all} methods face model misspecification (Type A) and inference approximation errors (Type C), the reliance on summary statistics and associated information loss (Type B) is more explicit in SBI, though present in any inference technique. These three types are summarized in Table~\ref{tab:uncertainty_taxonomy}.

\begin{table}[h!]
\centering
\begin{tabular}{@{}lll@{}}
\toprule
\textbf{Category} & \textbf{Type} & \textbf{Description} \\
\midrule
\textbf{Epistemic}
    & Type A & Misspecified model \\
    & & \textit{``I'm simulating the wrong world''} \\
\cmidrule(l){2-3}
    & Type B & Lossy summary \\
    & & \textit{``I'm discarding relevant information''} \\
\cmidrule(l){2-3}
    & Type C & Inexact inference \\
    & & \textit{``I can't compute the posterior exactly''} \\
\midrule
\textbf{Aleatoric} & & Irreducible randomness \\
\bottomrule
\end{tabular}
\caption{Proposed taxonomy of epistemic and aleatoric uncertainties in simulation-based inference.\label{tab:uncertainty_taxonomy}}
\end{table}


\subsection{Type A: Misspecified Model}

If the assumed simulator does not sufficiently reflect the true data-generating process, this mismatch can lead to two distinct types of problems: poor description of the observed data, and systematic biases in inferring causally meaningful parameters. These challenges have been recognized for decades, with extensive literature examining how misspecification affects data description~\citep[\fex][]{white_maximum_1982} versus parameter estimation~\citep[\fex][]{hausman_specification_1978}. As \cite{box_science_1976} famously noted, ``all models are wrong, but some are useful.'' In the modern machine learning literature, related challenges include distribution shifts and domain generalization~\citep[for a recent overview see][]{zhou_domain_2022}.

\begin{description}[leftmargin=2em, labelwidth=1.5em, labelsep=0.5em, itemsep=0.75em]
\item[{\small\faSearch}] \textbf{Diagnostics:} Primarily addressed through model misspecification diagnostics~\eqref{sec:diag:misspec}, including robustness checks~\eqref{sec:diag:misspec:robustness}, posterior predictive tests~\eqref{sec:diag:misspec:ppc}, and comparative model diagnostics~\eqref{sec:diag:misspec:comparative}. When a reference posterior is available, comparison~\eqref{sec:diag:reference} can additionally reveal systematic biases arising from model mismatch.

\item[{\small\faShield}] \textbf{Mitigation:} Ideally, refine the simulator to better capture the true data-generating mechanism based on diagnostic outcomes. Alternatively, when some aspects of the model remain uncertain, design inference procedures that are robust to known sources of uncertainty via simulator uncertainty methods~\eqref{sec:adv:uncertainty}, such as robust summary learning~\eqref{sec:adv:uncertainty:robust}.
\end{description}


\subsection{Type B: Lossy Summary}

For practical reasons, inference tasks often focus on specific aspects or subsets of available data, tightly linked to the modeling choices themselves. In simulation-based methods, this takes the form of explicit reliance on summary statistics that compress the data to extract correlations with parameters of interest. In likelihood-based contexts, this reliance is typically implicit, manifested through the selection of data subsets, regions of interest, or specific observables—choices often driven by where probabilistic modeling is computationally feasible. Except in idealized cases, this data compression is necessarily lossy~\citep{lehmann_theory_1998, marin_approximate_2011}, potentially leading to reduced precision in parameter estimates while still permitting valid Bayesian inference with potentially appropriately broader posteriors. The goal is to keep this information loss sufficiently small and to maximize the extraction of \emph{relevant} information for inference.

\begin{description}[leftmargin=2em, labelwidth=1.5em, labelsep=0.5em, itemsep=0.75em]
\item[{\small\faSearch}] \textbf{Diagnostics:} In situations where the true posterior is tractable, reference posterior diagnostics~\eqref{sec:diag:reference} can quantify information loss. Otherwise, detection is challenging since many standard tests are completely insensitive to this failure mode, including common posterior coverage tests~\eqref{sec:diag:forward_back:coverage}. However, model-based forward-backward diagnostics~\eqref{sec:diag:forward_back:model_based} can detect cases of information loss.

\item[{\small\faShield}] \textbf{Mitigation:} If the achieved precision is acceptable for the scientific question at hand, or if information loss is intentionally introduced to mitigate Type~A errors, no mitigation is needed. Otherwise, increase network capacity, adopt more expressive architectures, or apply information-theoretic principles for summary design~\e qref{sec:found:summaries} to construct more informative features.
\end{description}


\subsection{Type C: Inexact Inference}

Even with a well-specified simulator and all relevant information extracted from the data, inference results can still be systematically incorrect. This arises when the inference algorithm fails to accurately approximate the true posterior defined by the chosen simulator and data summary. Common causes include inappropriate assumptions about posterior or likelihood shapes, insufficient computational budgets leading to approximation errors, non-convergence issues, and numerical inaccuracies. In likelihood-based inference, extensive diagnostic frameworks have been developed to detect such problems~\citep[\fex][]{gelman_inference_1992}. Although testing these issues using simulated data with known ground truth is often computationally costly, the development of robust diagnostic tools continues to make such validation increasingly practical and reliable.

\begin{description}[leftmargin=2em, labelwidth=1.5em, labelsep=0.5em, itemsep=0.75em]
\item[{\small\faSearch}] \textbf{Diagnostics:} When available, direct comparison to a reference posterior~\eqref{sec:diag:reference} provides a strong diagnostic, though results may be confounded with Type~B errors. Forward-backward diagnostics~\eqref{sec:diag:forward_back} offer efficient and targeted tests that can isolate inference approximation errors.

\item[{\small\faShield}] \textbf{Mitigation:} Increase training compute and data budget to improve convergence. Refine optimization procedures, adjust learning rate schedules, and experiment with alternative network architectures. More generally, follow established deep learning best practices for training stability and convergence.
\end{description}

\medskip

The classification into Types~A--C reflects different levels of the analysis pipeline, but also different severity: Type~B is often acceptable and quantifiable, Type~C can typically be reduced with better algorithms or more compute, while Type~A reflects deeper mismatches that typically require specific domain knowledge that goes beyond the scope of statistical inference. 


\section{Building Systems that Work}

\begin{quote}
\textit{``A complex system that works is invariably found to have evolved from a simple system that worked. A complex system designed from scratch never works and cannot be patched up to make it work. You have to start over with a working simple system.''}

\hfill --- John Gall, 1977, The Systems Bible
\end{quote}

\noindent
\textbf{Building a working SBI pipeline is like building a clockwork mechanism.} You do not order gears, springs, and escapements, immediately assemble everything according to your initial design, wind it up, and expect it to tick. That approach guarantees failure---and when nothing moves, you have no idea which of the hundred components is responsible.

Instead, a clockmaker works iteratively: start with the main gear system, verify it rotates smoothly. Add the escapement, check it regulates correctly. Add the pendulum, verify synchronization. At each step, the mechanism \textit{works}---it just does not yet do everything you ultimately need.

\medskip

The same principle applies to SBI. The diagnostic methods in Sections~\ref{sec:diag:reference}--\ref{sec:diag:misspec} are designed to identify specific failures in \textit{nearly-working} systems. Applied to completely broken pipelines, they generate noise. Applied to incrementally refined systems, they can pinpoint exactly what is wrong.

\cw{TODO: Polish the below list}

\paragraph{The Iterative Refinement Workflow}

\begin{enumerate}
\item \textbf{Start minimal}: Toy data, known sufficient statistics, simple networks, few parameters
\item \textbf{Validate thoroughly}: Run diagnostics (SBC, coverage, PPCs)
\item \textbf{Add one component}: Learned summaries \textit{or} realistic noise \textit{or} more parameters---not all at once
\item \textbf{Re-validate}: If diagnostics pass, proceed. If they fail, you know what broke.
\item \textbf{Iterate}: Repeat until you reach full problem complexity
\end{enumerate}

\textbf{Key principles:}
\begin{itemize}
\item When making progress: trust previously validated components
\item When stuck: question everything, including ``working'' components validated under different conditions
\item Each failure should point to a clear next step
\item If you can't explain your problem in 5 sentences, your example isn't minimal enough
\end{itemize}

The diagnostics that follow are powerful tools when applied systematically. The iterative methodology ensures they provide actionable information rather than overwhelming noise.





\section{Reference Posterior Diagnostics}
\label{sec:diag:reference}

An important question for the reliability of SBI is how well a learned posterior $q_\phi(\btheta \mid \bxobs)$ approximates the true Bayesian posterior $p(\btheta \mid \bxobs)$ under the generating model. When a reference posterior is available that is close enough to the true Bayesian one---either analytically, through Fisher approximations, or via high-quality MCMC samples---we can use it as ground truth to assess the learned posterior's accuracy:
%
\begin{equation}
    q_\phi(\btheta \mid \bxobs)
    \;\;\xleftrightarrow{\text{consistent with?}}\;\;
    p(\btheta \mid \bxobs)
\end{equation}
%
Such situations occur mainly in simplified setups or low-dimensional models where likelihood-based inference remains feasible.

From the perspective of the epistemic uncertainty taxonomy (Sec.~\ref{sec:diag:taxonomy}), different uncertainty types manifest differently in these comparisons. When both posteriors are derived from simulated data $\bx$, discrepancies typically reflect \textbf{Type B} (lossy summary) or \textbf{Type C} (inexact inference) uncertainties. Information loss usually manifests as broadened learned posteriors, while inference approximation errors often appear as biases, missing modes, or other structural distortions. When comparing posteriors derived from \emph{real} data $\bxobs$, \textbf{Type A} (misspecified model) uncertainties can introduce additional discrepancies, since SBI and other methods may respond differently under model misspecification~\citep{cannon_investigating_2022}.


\subsection{f-Divergences and Mismatch Sensitivity}
\label{sec:diag:reference:divergences}

A natural way to quantify discrepancies between a learned posterior and a reference posterior is through divergence measures. Divergences reduce differences between distributions to a single scalar but differ in their sensitivity to specific types of mismatch, such as shifts in location, changes in spread, or differences in tail behaviour. This sensitivity can help diagnose \emph{which type of error} we are dealing with.

Common examples include the Kullback--Leibler (KL), Jensen--Shannon (JS), and Neyman $\chi^2$ divergences. These divergences fall in the broad category of $f$-divergences, which are generally defined as\footnote{Historically, $f$-divergences were first introduced by \cite{renyi_measures_1960} (see \cite{sason_divergence_2022} for historical context). Mathematical definitions and many of their theoretical properties are discussed in \cite{liese_divergences_2006, pardo_statistical_2018}, while recent applications in machine learning include training energy-based models \cite{yu_training_2020} and imitation learning \cite{ke_imitation_2020}.}
\begin{equation}
    D_f(p \mid\mid q) \equiv \int d\btheta\, q(\btheta) f\left(\frac{p(\btheta)}{q(\btheta)}\right)\;.
    \label{eqn:f_divergence}
\end{equation}
Here, $f: [0, +\infty)\to (-\infty, +\infty]$ is the \emph{generator} of a particular $f$-divergence. General properties that it has to satisfy include that $f(t)$ is convex and $f(t=1) = 0$. In all cases, $D_f$ is non-negative and equals zero only when $q(\btheta)=p(\btheta)$ almost everywhere.


\subsubsection{Common $f$-Divergences}

Some common $f$-divergences can be found in Tab.~\ref{tab:divergences}. They include the Kullback--Leibler divergence, which plays a prominent role in information theory, but also related quantities such as the total variation distance, the Jensen--Shannon divergence, and others. Some of these divergences are symmetric under $q \leftrightarrow p$ by construction, which we indicate by writing $D(p; q)_f$ instead of $D(p \mid\mid q)_f$. This symmetry property is related to the fact that, in general, $f$-divergences obey $D_f(p \mid\mid q) = D_g(q \mid\mid p)$ with $g(t) = t f(1/t)$. See~\citep{liese_divergences_2006, pardo_statistical_2018} for a discussion of their connections and relations.

\begin{table}[h]
    \centering
    \begin{tabular}{llc}
        \toprule
        \textbf{Divergence} &
        \textbf{Name} & \textbf{Corresponding } $f(t)$ \\
        \midrule
        $D_{KL}(p\mid\mid q)$ & KL divergence & $t \log t$ \\
        $D_{\chi^2}(p\mid\mid q)$ & Neyman $\chi^2$-divergence & $(t - 1)^2$ \\
        $D_{J}(p; q)$ & Jeffreys divergence & $(t - 1) \log(t)$ \\
        $D_{JS}(p; q)$ & Jensen-Shannon divergence & $\frac{1}{2}\left(t \log t - (t + 1) \log \left(\frac{t + 1}{2}\right)\right)$ \\
        $D_{TV}(p; q)$ & Total variation distance & $\frac{1}{2}|t - 1|$ \\
        \bottomrule
    \end{tabular}
    \caption{Selected $f$-divergence measures $D(p \mid \mid q)_f$, and their corresponding generators $f(t)$. The usage of a semicolon, $D(p ; q)_f$, indicates that the divergence is symmetric.}
    \label{tab:divergences}
\end{table}


\begin{figure}[t]
    \centering
    \includegraphics[width=\linewidth]{figures/fig_divergences.pdf}
    \caption{
        Illustration of how different divergences respond to different types of posterior mismatch.
        (a) Reference distribution $q(\theta)$ (black dashed) and three approximate distributions exhibiting different types of mismatch: $p_1(\theta)$ with bias (blue), $p_2(\theta)$ with overdispersion (orange), and $p_3(\theta)$ with an additional mode (green). 
        (b) Divergence values between the reference and approximate distributions, demonstrating that different divergence measures exhibit varying sensitivity to different aspects of posterior mismatch. Metrics are rescaled as indicated to ease comparison and ordered by relative sensitivity to unmodeled modes. 
    }
    \label{fig:posterior_divergences}
\end{figure}


\subsubsection{Sensitivity to Common Distortions} 

In Fig.~\ref{fig:posterior_divergences}, we illustrate how the divergence measures listed in Tab.~\ref{tab:divergences} differ in their sensitivity to common types of distributional mismatch. We compare the values of various divergences between a fixed learned posterior $q$---here taken to be a Gaussian---and three different reference posteriors that exhibit, respectively, a shift, a broadening, or heavy tails. To ease comparison across divergences, the plotted values have been rescaled as indicated in the figure labels, such that the case of a shifted posterior yields comparable numbers.\footnote{The rescaling factor corresponds to the second derivative of the generator, $f''(t)\rvert_{t=1}$, which governs the leading behaviour when $p$ is close to $q$, except for TV where the factor was chosen ad hoc.}

A few observations can be made:
\begin{itemize}
    \item \textbf{Shifts and broadenings} ($p_1$ and $p_2$): After rescaling, all divergences react similarly to small location shifts or changes in spread.
    \item \textbf{Heavy tails} ($p_3$): The divergences behave very differently. This reflects how their generators $f(t)$ weight the regimes $t \ll 1$ and $t \gg 1$.
    \item \textbf{KL asymmetry}: $D_{KL}(p\mid\mid q)$ strongly responds to the additional tail in $p_3$, whereas the reverse KL $D_{KL}(q\mid\mid p)$ is largely insensitive.
    \item \textbf{Neyman $\chi^2$}: Shows an even stronger asymmetry than KL, amplifying deviations in the $t \gg 1$ regime.
    \item \textbf{Symmetric divergences}: TV is largely insensitive to the $p_3$ tail, while JS and J divergences detect it.
\end{itemize}

As we will discuss next, the high sensitivity to tail behaviour exhibited by some $f$-divergences comes at the cost of numerical instability when they are estimated from finite samples. In practice, the Jensen--Shannon divergence often provides a useful compromise: it is sufficiently sensitive to missing mass while remaining comparatively stable under sampling noise.

\subsection{Estimating Divergences from Samples}
\label{sec:diag:reference:estimation}

The integral in Eq.~\eqref{eqn:f_divergence} can be approximated using Monte Carlo methods, exploiting the identities
\begin{equation}
    D_f(p\mid\mid q)
    = \mathbb E_{q(\btheta)}[f(t)]
    = \mathbb E_{p(\btheta)}[t\, f(1/t)]
    \;,
    \label{eqn:f_div_approx}
\end{equation}
where $t(\btheta)=p(\btheta\mid\bx)/q_\phi(\btheta\mid\bx)$ denotes the density ratio. Thanks to the symmetry properties of $f$-divergences discussed above, these expectations can be estimated using samples from either $p(\btheta)$ or $q(\btheta)$. In the rare cases where both the learned and reference posteriors can be evaluated explicitly, these expressions can be used directly. In all other cases, additional estimation strategies are required, as discussed below.


\subsubsection{Density-Evaluation-Based Divergence Estimation}

In situations where the likelihood function $p(\bx \mid \btheta)$ is explicitly known, we generally only have access to the unnormalised posterior $\pi(\btheta \mid \bx) = p(\bx\mid\btheta)\,p(\btheta) = Z\,p(\btheta \mid \bx)$, precluding a direct application of Eq.~\eqref{eqn:f_div_approx}. However, if we can explicitly evaluate the density of the learned posterior $q_\phi(\btheta \mid \bx)$, we can use it to estimate the missing normalisation factor $Z$.

Assuming that we have $N$ samples $\btheta_i \sim q_\phi(\btheta \mid \bx)$, we can estimate the normalization constant of $\pi(\btheta \mid \bx)$ through
$$
Z\simeq\frac1N\sum_{i=1}^N 
\frac{\pi(\btheta \mid \bx)}{q_\phi(\btheta \mid \bx)} \;.
$$
In the limit $N\to \infty$, the sum turns into an integral over $\btheta$, yielding the definition of $Z= \int d\btheta\, \pi(\btheta \mid \bx)$.\footnote{On the other hand, if we have $N$ samples
$\btheta_i \sim p(\btheta \mid \bx)$, which one can obtain from $\pi(\btheta \mid \bx)$ by, \fex\ MCMC methods, we can estimate $Z$ via a similar expression,
$
Z = \left(\frac1N\sum_{i=1}^N \frac{q_\phi(\btheta \mid \bx)}{\pi(\btheta \mid \bx)}\right)^{-1}\;.
$}
%
We can then approximate posterior density ratios through the weights
\begin{equation}
w_i =  \frac
{\pi(\btheta_i\mid \bx)}
{Z\,q_\phi(\btheta_i\mid\bx)}
\simeq \frac{p(\btheta \mid \bx)}  
{q_\phi(\btheta_i\mid\bx)}\;.
\label{eqn:def_w_i}
\end{equation}

The application of variations of Eq.~\eqref{eqn:f_div_approx} allows then to compute any $f$-divergence of interest.  As a concrete illustrative examples, we estimate the forward KL divergences from $N$ learned posterior samples, $\btheta_i \sim q_\phi(\btheta \mid \bx)$.  The corresponding expression is 
$$
D_{KL}(q_\phi\mid\mid p) 
\equiv \int d\btheta \, q_\phi(\btheta \mid \bx)
\log \frac{q_\phi(\btheta \mid \bx)}{p(\btheta \mid \bx)}
\simeq \frac1N \sum_{i=1}^N \log w_i\;.
$$
We note that a similar expression can be found based on samples from the references posterior, using the general symmetry properties of $f$-divergences.


\subsubsection{Sampling-Based Divergence Estimation}

In situations where only samples from the reference and learned posteriors are available, a natural approach is to estimate the density ratio $t(\btheta) \equiv p(\btheta)/q(\btheta)$ directly from samples, for instance using kernel density estimates for both distributions followed by pointwise division. This then allows a direct application of Eq.~\eqref{eqn:f_div_approx}. 

A more common and often simpler strategy is to bin the samples and compute divergences between the resulting normalised histograms. Binning methods are only suitable in very low-dimensional parameter spaces (one or two dimensions), since higher dimensions inevitably lead to many bins with few or zero samples.

As an example, assume that samples from the reference posterior $p$ and the learned posterior $q_\phi$ have been binned into $K$ bins, and let $\{m_i\}$ and $\{n_i\}$ denote the corresponding counts, normalised such that $\sum_i m_i = \sum_i n_i = N$. The binned Jensen--Shannon (JS) divergence is then
\[
D_{\mathrm{JS}}(q\,\|\,p)
= \frac{1}{N} \sum_{i=1}^K 
\left[
n_i \log \frac{2n_i}{n_i + m_i}
+
m_i \log \frac{2m_i}{n_i + m_i}
\right]\;.
\]
A useful feature of the JS divergence is that it remains finite even when $n_i = 0$ or $m_i = 0$, which makes it far more stable than most other $f$-divergences under finite-sample fluctuations. This numerical robustness makes the binned JS divergence a practical tool for comparing posteriors.\footnote{For completeness, the corresponding binned KL and Pearson $\chi^2$ divergences are
\[
D_{\mathrm{KL}}(p\,\|\,q) = \frac{1}{N}\sum_{i=1}^K m_i \log\!\left(\frac{m_i}{n_i}\right)
\qquad\text{and}\qquad
D_{\chi^2}(p\,\|\,q) = \frac{1}{N}\sum_{i=1}^K \frac{(n_i - m_i)^2}{\,n_i\,}\;.
\]
Both divergences are asymmetric and become ill-defined when $n_i = 0$ (except in the trivial case $m_i = n_i = 0$ for KL). This behaviour reflects the fact that they strongly penalize the learned posterior $q_\phi$ for assigning mass to regions where the reference posterior $p$ assigns none (see Fig.~\ref{fig:posterior_divergences}).}


\subsection{Normalised Importance Sampling from the True Posterior}
\label{sec:diag:reference:nis}

% TODO: Think this through carefully again, in connection with particle filter and SMC
This subsection is not a diagnostic method per se—it briefly describes a useful technique for obtaining (approximately) exact posterior samples when the unnormalised density $\pi(\btheta \mid \bx)$ can be evaluated. When $\pi(\btheta \mid \bx)$ is available, we can use samples from $q_\phi(\btheta \mid \bx)$ to importance sample from the ground-truth posterior $p(\btheta \mid \bx)$. The strategy is simple: (1) obtain $N$ samples from $q_\phi(\btheta \mid \bx)$, (2) calculate $w_i$ as defined in Eq.~\eqref{eqn:def_w_i}, and (3) resample (with replacement) from the existing $q_\phi$ samples with probabilities $\tilde w_i = w_i / \sum_{j=1}^N w_j$. This procedure, known as \emph{normalised importance sampling} (NIS)~\citep{xxx}, yields an empirical approximation to $p(\btheta \mid \bxobs)$ that converges to the correct distribution as $N \to \infty$, provided that the support of $q_\phi(\btheta \mid \bxobs)$ includes that of $p(\btheta \mid \bxobs)$. This resampling step is the same normalised importance sampling update used in particle filters and Sequential Monte Carlo (SMC) methods.


\subsubsection{Effective Sample Size Diagnostic}

When $q_\phi(\btheta \mid \bxobs)$ deviates substantially from the true posterior, the importance weights develop a high variance and only a few samples dominate the resampling process while most contribute negligibly. To quantify this effect, we define the effective sample size (ESS) as
\[
\text{ESS} \equiv \frac{1}{\sum_{i=1}^N \tilde w_i^2} \;.
\]
By construction, $1 \leq \text{ESS} \leq N$. The upper bound is saturated when all weights are equal (perfect match between proposal and target), and the lower bound when a single sample carries essentially all weight. The ESS thus provides a direct measure of how many samples effectively contribute to Monte Carlo estimation.\footnote{For instance, consider the weighted posterior mean $\bar\theta_i = \sum_j \tilde w_j \theta_i^{(j)}$. Its variance is approximately $\mathrm{Var}(\bar \theta_i) \simeq \frac{1}{\mathrm{ESS}} \, \mathrm{Var}_{q_\phi}[\theta_i]$. Thus, a low ESS implies high variance in reweighted estimates—even if $N$ is large.}

\medskip

Interestingly, this method not only provides a diagnostic for the fidelity of the learned posterior but also a way to correct it through NIS. The only remaining approximation then arises from finite sample size. Two important caveats should be kept in mind. First, the method requires that the support of \( q_\phi(\btheta \mid \bxobs) \) covers that of the true posterior: if \( q_\phi \) assigns negligible mass to regions where \( p(\btheta \mid \bxobs) \) is large—such as missing isolated modes—then importance sampling fails, and the ESS may not reliably flag the issue. Second, the ESS is expected to become small in sufficiently high-dimensional settings, because approximation errors of the learned posterior accumulate and lead to highly variable importance weights.



\section{Forward--Backward Diagnostics}
\label{sec:diag:forward_back}

In Sec.~\ref{sec:diag:reference}, we discussed diagnostic techniques based on comparing the learned posterior with a suitable reference posterior, ideally the true Bayesian posterior. Such reference-based tests provide a strong external benchmark—SBI can still "hold on to a guiding hand."  However, in many realistic applications no such reference is available.

In this situation, a common strategy is to test the internal consistency between the generative (forward) model and the learned inference (backward) model. Concretely, we compare joint samples from the backward model, \((\bx,\btheta)\sim p(\bx)\,q_\phi(\btheta\mid\bx)\), to joint samples from the forward model, \((\bx,\btheta)\sim p(\btheta)\,p(\bx\mid\btheta)\). If the learned posterior matches the true posterior, \(q_\phi(\btheta\mid\bx)=p(\btheta\mid\bx)\), then these two joint distributions coincide; conversely, any discrepancy between forward and backward samples is a clear indicator of inference error. 

The basic consistency requirement can be summarised schematically as
\begin{equation}
q_\phi(\btheta\mid\bx)\,p(\bx)
\;\;\xleftrightarrow{\text{consistent with?}}\;\;
p(\bx\mid\btheta)\,p(\btheta)\;,
\end{equation}
and we will collectively refer to all tests based on this idea as \emph{forward--backward diagnostics}. They play the role of letting SBI "walk on its own": rather than relying on a known reference, these methods assess whether the inference model is consistent with the generative model it is meant to describe.

\medskip

From the perspective of the epistemic uncertainty taxonomy (Sec.~\ref{sec:diag:taxonomy}), forward-backward diagnostics have specific sensitivities. \textbf{Type A} (misspecified model) uncertainties are not detected, since both sides of the comparison are based on simulated data from the same assumed model. \textbf{Type B} (lossy summary) uncertainties are invisible for most common approaches, with a few notable exceptions discussed below. \textbf{Type C} (inexact inference) represents the primary target: forward--backward discrepancies can reveal most type~C epistemic uncertainties in the learned posterior.

\subsection{Simulation-Based Calibration Diagnostics}
\label{sec:diag:forward_back:sbc}

Simulation-based calibration (SBC) is the simplest type of forward--backward consistency test.  It applies to cases where the posterior distribution is one-dimensional, and thus to any one-dimensional projection of multi-dimensional posteriors. SBC already features many of the key characteristics and challenges of forward--backward consistency tests, making it a natural starting point.


\subsubsection{SBC Rank Statistics} 

SBC is a \emph{rank-based} test: for a given observation $\bx$, draw one parameter value from the true posterior, $\theta^\ast \sim p(\theta \mid \bx)$, and then draw $L$ further samples $\theta_{1:L} \equiv \theta_1, \theta_2, \dots, \theta_L \sim q_\phi(\theta \mid \bx)$. The \emph{rank} of $\theta^\ast$ within $\theta_{1:L}$ is defined as the number of $\theta_i$ that happen to fall below $\theta^\ast$. If all $\theta_i$ and $\theta^\ast$ are i.i.d.\ from the same distribution, their joint distribution is exchangeable, which means that $\theta^\ast$ is equally likely to land at any of the $L+1$ possible rank positions. This procedure is then repeated and averaged over random data samples $\bx \sim p(\bx)$, making it possible to replace the infeasible posterior draw with a draw from the forward generative model.

\medskip

More formally, rank-based diagnostics such as SBC start from the \emph{rank-test samples}
%
\begin{equation}
    \theta^\ast,
    \mathbf{x},
    \theta_1, \dots, \theta_L
    \sim 
    p(\theta^\ast)\,
    p(\mathbf{x} \mid \theta^\ast)\,
    \prod_{i=1}^L
    q_{\phi}(\theta_i \mid \mathbf{x}) 
    \;.
    \label{eqn:rank_test_samples}
\end{equation}
%
In words: draw a parameter from the prior, $\theta^\ast \sim p(\theta)$, generate an observation $\bx \sim p(\bx \mid \theta^\ast)$, and then draw $L$ samples from the learned posterior $q_\phi(\theta \mid \bx)$. By Bayes’ theorem, this is equivalent to first sampling $\bx \sim p(\bx)$ and then $\theta^\ast \sim p(\theta \mid \bx)$, followed by the same posterior draws.

To compare $\theta^\ast$ with the posterior samples, we define the (normalised) \emph{rank} of $\theta^\ast$ within $\{ \theta_1, \dots, \theta_L \}$ as
%
\begin{equation}
    F(\theta^\ast) = \frac{1}{L} \sum_{i=1}^L \mathbb{1}
    \big(\theta_i \leq \theta^\ast\big)\;,
\end{equation}
%
so that $F(\theta^\ast)$ can only take values in $\{0, 1/L, 2/L, \dots, 1\}$. We can then define the probability that the rank, averaged over rank-test samples from Eq.~\eqref{eqn:rank_test_samples}, takes a particular value $k/L$:
%
\[
\mathbb{P}\!\left(F(\theta^\ast)=\tfrac{k}{L}\right)
= 
\mathbb{E}_{\substack{
\theta^\ast \sim p(\theta),\; 
\bx \sim p(\bx \mid \theta^\ast),\\[2pt]
\theta_{1:L} \sim \prod_{i=1}^L q_\phi(\theta_i \mid \bx)
}}\!\left[
  \mathbb{1}\!\left\{ \sum_{i=1}^L \mathbb{1}(\theta_i \leq \theta^\ast) = k \right\}
\right].
\]

If $q_\phi(\theta \mid \bx) = p(\theta \mid \bx)$, then $\theta^\ast$ and the $\theta_i$ are i.i.d.\ from the same distribution. By exchangeability, each ordering is equally likely, and hence each of the $L+1$ possible rank positions is equally probable,
%
\[
\mathbb{P}\!\left(F(\theta^\ast)=\tfrac{k}{L}\right) = \frac{1}{L+1}\;.
\]
This discrete uniformity is the central calibration criterion of SBC: any systematic deviation from flatness indicates a mismatch between $q_\phi$ and the true posterior.

\paragraph{Continuous Limit: The Probability Integral Transform}
In the large sample limit $L \to \infty$, the discrete rank statistic approaches a continuous uniform distribution on $[0,1]$. The limiting object is the probability integral transform (PIT),
\begin{equation}
F(\theta^\ast) = \int_{-\infty}^{\theta^\ast} d\theta\, q_\phi(\theta \mid \bx)\;,
\label{eqn:PIT}
\end{equation}
namely, the value of the posterior CDF at the true parameter. The PIT is well known in statistics as a standard tool for checking calibration. Its advantage is that PIT-based tests provide a continuous and potentially more sensitive alternative to rank-based SBC, though this requires efficient evaluation of the integral.

The central calibration criterion, whether using discrete ranks or the continuous PIT, is the same: if $q_\phi(\theta \mid \bx) = p(\theta \mid \bx)$, then
%
\begin{equation}
    F(\theta^\ast) \sim \text{Uniform}(0, 1)\;.
\end{equation}
%
This uniformity condition underpins both SBC and PIT-based diagnostics: deviations from flatness in the rank (or PIT) distribution directly signal miscalibration of the learned posterior.

\begin{figure}[t]
    \centering
    \includegraphics[widht=\linewidth]{figures/fig_sbc.pdf}
    %\includegraphics[width=0.49\linewidth]{figures/SBC_shapes.png}
    %\includegraphics[width=0.49\linewidth]{figures/SBC_histograms.png}
    \caption{Simulation-based calibration. \emph{Left panel:} Exact, shifted, overdispersed, and underdispersed inferred posteriors, $q_\phi(\theta \mid \mathbf{x})$, compared to the ground-truth posterior, $p(\theta \mid \mathbf{x})$ (gray). 
    \emph{Right panel:} Resulting rank histogram (with $L=10$ and $N=10000$). Even small biases in the shape are detected. We assume that the type of mismodeling is uniform across different values of $\mathbf{x}$.}
    \label{fig:SBC_basics}
\end{figure}


\subsubsection{Interpreting SBC Rank Histograms}

In practice, SBC is a simple but powerful baseline check for posterior calibration, and it is a natural starting point before turning to more elaborate diagnostics. Its usefulness becomes clear once we look at how different types of miscalibration leave distinct and easily recognizable signatures in the rank histogram.

In Fig.~\ref{fig:SBC_basics}, we illustrate how bias and miscalibration appear in practice.\footnote{We assume a uniform prior, \( \theta \sim \text{Uniform}(-1, 1) \), and Gaussian measurement noise, \( x \sim \text{Normal}(\theta, \sigma^2) \) with \( \sigma = 0.1 \). The inferred posterior is modeled as \( q_\phi(\theta \mid x) = \text{Normal}(\theta; x + b, s \sigma^2) \), where \( b \neq 0 \) introduces a location bias and \( s \neq 1 \) scales the posterior variance. Underdispersion (overconfidence) corresponds to \( s < 1 \), and overdispersion (conservativeness) to \( s > 1 \).} The left panel compares the true posterior \( p(\theta \mid \bx) \) with biased and miscalibrated approximations, while the right panel shows the resulting rank histograms \( F(\theta^\ast) \), computed from \( L = 30 \) posterior samples per simulation across \( N = 1000 \) simulations.

The patterns revealed in Fig.~\ref{fig:SBC_basics} are characteristic:
\begin{itemize}
    \item \textbf{Bias} (shifted posteriors) manifests as skewed histograms;
    \item \textbf{Overdispersion} (too wide posteriors) leads to inverse U-shapes;
    \item \textbf{Underdispersion} (too narrow posteriors) gives U-shapes.
\end{itemize}
Even mild miscalibration leaves a visible trace. All of these departures can be read as deviations from the flat, uniform rank distribution, which makes SBC a practical and sensitive tool for diagnosing type~C epistemic uncertainty, even when the true posterior is not directly accessible.

\begin{figure}[th]
    \centering
    \includegraphics[width=\linewidth]{figures/fig_sbc_bias.pdf}
    %\includegraphics[width=0.49\linewidth]{figures/SBC_shapes_nonuniform.png}
    %\includegraphics[width=0.49\linewidth]{figures/SBC_histogram_nonuniform.png}
    \caption{If deviations between the inferred posterior \( q_\phi(\theta \mid \bx) \) and the ground truth \( p(\theta \mid \bx) \) vary with the observation \( \bx \), they can partially cancel out in the overall SBC histogram. \emph{Left panel:} Comparison of a uniform (i.e., \( \bx \)-independent) bias with a non-uniform bias that changes across data space. \emph{Right panel:} The rank histogram for the non-uniform case appears nearly uniform when aggregated over all \( \bx \), but stratifying by data subsets reveals the underlying miscalibration.}
    \label{fig:SBC_non_uniform_bias}
\end{figure}

\begin{figure}[th]
    \centering
    \includegraphics[width=\linewidth]{figures/fig_sbc_lossy.pdf}
    \caption{SBC and HPDR coverage tests are sensitive to overdispersion but largely insensitive to posteriors based on suboptimal data compression. Shown are two scenarios: an overdispersed posterior \( q_\phi(\theta \mid \bx) \propto p(\theta \mid \bx)^{1/2} \), and a calibrated but suboptimal posterior \( q_\phi(\theta \mid \bx) = p(\theta \mid T(\bx)) \) based on a lossy summary. Both are broader than the ground truth, but for different reasons: the overdispersed posterior overestimates uncertainty, while the suboptimal one correctly reflects additional variation due to limited information.}
    \label{fig:SBC_insufficiency}
\end{figure}


\subsubsection{Insensitivity to Spatially Varying Bias} 

SBC evaluates calibration averaged over the data distribution, \( \bx \sim p(\bx) \), which means it can miss \emph{spatially varying} miscalibration. This averaging is both its strength and its weakness.  If systematic deviations change across data space—say, a bias that flips sign depending on \( \bx \)—then these effects can cancel in the aggregate rank histogram (see~\citep{modrak_simulation-based_2025} for a detailed discussion).  The result may look flat, even though calibration is clearly violated in particular regions of \( \bx \)-space.  Figure~\ref{fig:SBC_non_uniform_bias} shows such a case: the global histogram is nearly uniform, but once we inspect different regions of data space separately, the underlying miscalibration becomes obvious. 

The lesson is that standard SBC can miss local structure.  A natural extension is to stratify the analysis, computing rank histograms for subsets of \( \bx \).  This uncovers biases that are hidden in the global test, but requires generally more samples. The tradeoff is between detecting local miscalibration and maintaining statistical power in each subset.


\subsubsection{Insensitivity to Information Loss} 
\label{sec:diag:forward_back:sbc:information}

If the learned posterior is affected by pure type~B uncertainties \emph{only}, it can always be written as \( q_\phi(\theta \mid \bx) = p(\theta \mid T(\bx)) \), for some suitable (and lossy) data summary \( \bt = T(\bx) \). Although information loss leads in general to wider posteriors, this effect is distinct from overdispersion, which systematically inflates uncertainty around the correct location, producing posteriors that are too wide relative to the available information.

Figure~\ref{fig:SBC_insufficiency} illustrates this distinction. Both posteriors shown are broader than the ground truth, but only the overdispersed one leaves a clear SBC signature. The overdispersed posterior systematically exaggerates uncertainty but is mostly centered correctly, which produces the inverse U-shape familiar from above. On the other hand, posteriors based on an insufficient summary are both broader \emph{and} more variable in location across simulations in a self-consistent way. The resulting random shifts lead to a histogram that remains flat.

Let us formally study the effect of lossy summaries on the rank test samples. Under the SBC sampling scheme, Eq.~\eqref{eqn:rank_test_samples}, we have
\begin{equation}
    \theta^\ast \sim p(\theta), \quad 
    \bx \sim p(\bx \mid \theta^\ast), \quad 
    \theta_{1:L} \sim q_\phi(\btheta \mid \bx) = p(\btheta \mid \bt = T(\bx)) \, .
\end{equation}
The samples of \( \theta^\ast \) and \( \theta_{1:L} \) are then used to calculate the rank \( F(\theta^\ast) \). This rank does not directly depend on the simulated data \( \bx \), and the samples \( \theta_{1:L} \) only depend on \( \bx \) through the lossy summary \( T(\bx) \). The rank distribution will hence be the same if we replace the rank test samples with
\begin{equation}
\theta^\ast \sim p(\theta), \quad 
\bt \sim p(\bt \mid \theta^\ast), \quad 
\theta_{1:L} \sim p(\btheta \mid \bt) \, .
\end{equation}
Here, we formally replace data samples with summary samples. If we now apply Bayes' theorem to the \( (\theta^\ast, \bt) \) sample pairs, we arrive at
\begin{equation}
\bt \sim p(\bt),
\quad 
\theta^\ast \sim p(\theta \mid \bt),
\quad 
\theta_{1:L} \sim p(\btheta \mid \bt) \, .
\end{equation}
We see that \( \theta^\ast \) and \( \theta_{1:L} \) are independent and identically distributed, resulting in a flat SBC rank distribution.

This reveals a core limitation of SBC and many other rank-based diagnostics: SBC is completely insensitive to type~B epistemic uncertainty (information loss), while remaining a necessary—but not sufficient—test for type~C (inference error).


\subsubsection{Beyond Marginals: Testing the Joint Posterior}

So far we have considered SBC in one dimension, or applied separately to each component of a parameter vector \( \btheta = (\theta_1,\dots,\theta_N) \). The latter is a straightforward extension: given rank test samples \( (\btheta^\ast, \bx, \btheta_{1:L}) \) drawn from Eq.~\eqref{eqn:rank_test_samples}, one can obtain samples for any single parameter component simply by \emph{implicit marginalization}—that is, by omitting the other coordinates. This effectively yields samples \( \theta_i^\ast,\;\bx,\; \theta_{i,1:L} \sim q_\phi(\theta_i \mid \bx) \), with the marginal posterior defined as
\[
q_\phi(\theta_i \mid \bx) = \int d\theta_1 \cdots d\theta_{i-1}\, d\theta_{i+1}\cdots d\theta_N \, q_\phi(\btheta \mid \bx)\;.
\]
Let us assume we were able to show that \( p(\theta_i \mid \bx) = q_\phi(\theta_i \mid \bx) \) for all \( i=1, \dots, N \).\footnote{As discussed above, this cannot be guaranteed with SBC, because of insensitivities towards spatially varying bias and lossy summaries. But we focus here on an idealized scenario for the sake of the argument.} A natural question is whether this suffices to guarantee that the full joint posterior is correct, \( p(\btheta \mid \bx) \;\stackrel{?}{=}\; q_\phi(\btheta \mid \bx) \). The answer, however, is no—testing coordinates one-by-one only probes the marginals, and leaves the dependence structure between parameters untouched.

Interestingly, classical probability theory arguments open a path towards completeness. By the Cramér--Wold theorem~\cite{xxx}, a multivariate distribution is uniquely determined by the set of all its one-dimensional projections. Concretely, if one can show that
\[
q_\phi(\bw^\top \btheta \mid \bx) = p(\bw^\top \btheta \mid \bx) \quad \text{for all } \bw \in \mathbb{R}^N ,
\] 
then it follows that \( q_\phi(\btheta \mid \bx) = p(\btheta \mid \bx) \). Confirming equality between learned and true posterior for all possible one-dimensional projections implies equality for the high-dimensional distributions.

Of course, checking all possible projections is infeasible in practice. Nevertheless, running SBC on a sufficiently rich set of random or carefully chosen projections is computationally cheap, and can greatly enhance the sensitivity of the SBC probe. Used this way, SBC remains an effective and versatile tool even in higher-dimensional settings—provided that other pitfalls, such as insufficient summaries or non-uniform bias, are kept under control.


\subsection{Posterior Coverage Diagnostics}
\label{sec:diag:forward_back:coverage}

Credible regions are a common way to visualise and interpret posterior density distributions. For a given observation $\bx_{\mathrm{obs}}$, a credible region $\mathcal{C}_{1-\alpha}(\bxobs)$ is a subset of the parameter space that contains a specific fraction of the posterior mass.  Examples include the central $68\%$ or $95\%$ intervals, or the highest posterior density regions, but many different construction schemes exist. These regions admit a natural Frequentist perspective: over repeated draws from the generative model $(\bx, \btheta^\ast) \sim p(\bx, \btheta)$, the true parameter should fall inside a nominal $1-\alpha$-credible interval exactly with frequency $1-\alpha$ (see \cite{dawid_well-calibrated_1982} for an early discussion).

Posterior coverage diagnostics turn this expectation into a consistency test.  They ask whether the credible intervals computed from the learned posterior $q_\phi(\btheta \mid \bx)$ actually contain---or \emph{cover}---the true parameters with the indicated frequency under the data--generating process.  In other words, they evaluate whether nominal coverage matches empirical coverage. 

Although the logic appears to be different from rank-based SBC, the technical machinery is closely related: coverage tests can be expressed in terms of the same rank-test samples introduced above, but applied to sets rather than individual scalar comparisons.  And, as is typical for forward-backward diagnostics, they are defined in expectation over $\bx$.


\subsubsection{Defining Expected Posterior Coverage}

Given the learned posterior $q_\phi(\btheta \mid \bx_{\mathrm{obs}})$ for some observation $\bx_{\mathrm{obs}}$, the highest posterior density region (HPDR) with nominal credibility \(1 - \alpha\) is defined as
%
\begin{equation}
    \label{eqn:HPDR}
    \mathcal C_{1-\alpha}(\bx) \equiv \{\btheta \in \Theta \mid q_\phi(\btheta \mid \bx) \geq \gamma_{1-\alpha} \}\;,
\end{equation}
%
where the threshold \( \gamma_{1-\alpha} \) is chosen such that
%
\begin{equation}
    \label{eqn:one_minus_alpha}
    \int_{\mathcal C_{1-\alpha}(\bx)} d\btheta\, q_\phi(\btheta \mid \bx) = 1 - \alpha\;.
\end{equation}
%
Equivalently, $\mathcal C_{1-\alpha}(\bx)$ is the super-level set of $q_\phi(\btheta \mid \bx)$ at threshold $\gamma_{1-\alpha}$, i.e.\ the region bounded by the posterior contour that encloses $1-\alpha$ of its mass. Thus, HPDRs form a nested family of regions, expanding monotonically as the nominal credibility increases. As we will see, this nesting property is convenient for diagnostics, since it allows coverage to be assessed consistently using rank statistic methods.

\medskip

If \( q_\phi(\btheta \mid \bx) \) matches the true posterior \( p(\btheta \mid \bx) \), then a fraction \(1 - \alpha\) of samples from \( p(\btheta \mid \bx) \) should fall within \( \mathcal C_{1-\alpha}(\bx) \).   When \( q_\phi(\btheta \mid \bx) \) is overdispersed (underdispersed), this fraction will tend to be higher (lower) than \(1 - \alpha\).   This motivates the definition of the \emph{realized coverage},
%
\begin{equation}
    1 - \tilde\alpha(\bx) \equiv
    \mathbb{E}_{p(\btheta^\ast \mid \bx)}
    \left[\mathbb{1}\!\left(\btheta^\ast \in 
    \mathcal C_{1-\alpha}(\bx)
    \right)\right]
    = \int_{\mathcal C_{1-\alpha}(\bx)} d\btheta\, p(\btheta \mid \bx)\;.
\end{equation}
%
Since \( \mathcal C_{1-\alpha}(\bx) \) is constructed from \( q_\phi(\btheta \mid \bx) \), this quantity measures how well the nominal region $\mathcal{C}_{1-\alpha}(\bx)$ captures mass under the true posterior $p(\btheta \mid \bx)$.   The central diagnostic question is then whether \(1 - \tilde \alpha(\bx)\) matches the nominal level \(1 - \alpha\).

\medskip

However, in the spirit of forward--backward tests, we cannot usually compute realized coverage for a fixed observation $\bx$, since this would require samples from the intractable true posterior $p(\btheta \mid \bx)$.   Instead, we draw joint samples from the generative model, \( (\bx, \btheta^\ast) \sim p(\bx, \btheta) = p(\bx \mid \btheta)\, p(\btheta) \).   For each such pair, we construct \( \mathcal C_{1-\alpha}(\bx) \) and check whether the true parameter \( \btheta^\ast \) lies inside it.   This construction is practical, but evaluates \emph{average} coverage over \( \bx \sim p(\bx) \), rather than conditioning on a specific observation.   The resulting \emph{expected realized coverage} is formally given by
%
\begin{equation}
    1 - \tilde\alpha \equiv \bbE_{p(\bx)}
    [1 - \tilde\alpha(\bx)] = 
    \bbE_{p(\bx, \btheta)}
    \left[\mathbb{1}\left(
    \btheta \in 
    \mathcal C_{1-\alpha}(\bx)
    \right)\right]\;.
    \label{eqn:expected_realized_coverage}
\end{equation}
%
If \( 1 - \tilde\alpha \) agrees with the nominal level \( 1 - \alpha \), we can conclude that the learned posterior exhibits correct coverage \emph{on average}. This connection is typically visualized by plotting the realized coverage \( 1 - \tilde\alpha \) as a function of the nominal credibility \( 1 - \alpha \); such coverage plots are often referred to as \emph{pp-plots} in the literature.


\paragraph{An equivalent formulation for Monte Carlo estimation}
The definition in Eq.~\eqref{eqn:expected_realized_coverage} is the canonical one, expressed directly in terms of HPDRs.   For Monte Carlo estimation it is, however, convenient to use an equivalent  formulation. This relies on the statistic
%
\begin{equation}
    F_{\bx}(\btheta^\ast) = \int d\btheta\, q_\phi(\btheta\mid \bx) 
    \,\mathbb 1\!\left(
    q_\phi(\btheta \mid \bx) \geq
    q_\phi(\btheta^\ast\mid \bx)
    \right)\;,
    \label{eqn:F_statistic_q}
\end{equation}
%
which measures the posterior mass contained in the super-level set of  $q_\phi(\btheta \mid \bx)$ at density $q_\phi(\btheta^\ast \mid \bx)$.   Put differently, $F_{\bx}(\btheta^\ast)$ is the posterior credibility level of the contour  passing through $\btheta^\ast$.  

With this definition, the expected realized coverage can be written as
%
\begin{equation}
    1 - \tilde\alpha = 
    \bbE_{p(\bx, \btheta)} \left[ 
    \mathbb{1}\!\left(F_{\bx}(\btheta) \leq 1 - \alpha\right) 
    \right]\;,
    \label{eqn:F_coverage_test}
\end{equation}
%
which is equivalent to Eq.~\eqref{eqn:expected_realized_coverage}.
Its main advantages are that it avoids the explicit construction of HPDRs and  directly links to Monte Carlo estimators based on rank statistics, which we will  use below to diagnose posterior coverage in practice.


\subsubsection{Monte Carlo Estimation via Rank-Test Samples}

Constructing the \emph{exact} HPDR $\mathcal{C}_{1-\alpha}(\bx)$ for a given observation $\bx$ and level $1-\alpha$ is computationally challenging. Instead, one can adopt a Monte Carlo approximation scheme based on rank-test samples, similar to Eq.~\eqref{eqn:rank_test_samples} above.

We begin by generating \( N \) rank-test samples,
%
\begin{equation}
\label{eqn:rank_test_samples_multi}
    \boldsymbol{\theta}^\ast,
    \mathbf{x},
    \boldsymbol{\theta}_{1:L}
    \sim 
    p(\boldsymbol{\theta}^\ast)
    p(\mathbf{x} \mid \boldsymbol{\theta}^\ast)
    \prod_{i=1}^L
    q_{\phi}(\boldsymbol{\theta}_i \mid \mathbf{x}) 
    \;,
\end{equation}
%
where \( \boldsymbol{\theta}^\ast \sim p(\boldsymbol{\theta}) \), \( \mathbf{x} \sim p(\mathbf{x} \mid \boldsymbol{\theta}^\ast) \), and \( \boldsymbol{\theta}_i \sim q_\phi(\boldsymbol{\theta} \mid \mathbf{x}) \). As in SBC, this sampling structure is equivalent to drawing \( \bx \sim p(\bx) \), then \( \btheta \sim p(\btheta \mid \bx) \), due to Bayes’ theorem.
%
We can now use the rank-test samples to calculate the (normalised) rank of the true parameter $\btheta^\ast$ within the learned posterior samples $\btheta_{1:L}$,
%
\begin{equation}
    F_\bx(\btheta^\ast) = \frac{1}{L} \sum_{i=1}^L \mathbb{1}
    \left(q_\phi(\btheta_i \mid \bx) \geq q_\phi(\btheta^\ast \mid \bx)\right)
    \;.
    \label{eqn:HPDI_ranks}
\end{equation}
%
The main difference w.r.t.~SBC above is that we here use the learned posterior density itself as ordering principle.  Importantly, Eq.~\eqref{eqn:HPDI_ranks} provides a Monte Carlo estimate for the statistic in Eq.~\eqref{eqn:F_statistic_q}.  

Together with Eq.~\eqref{eqn:F_coverage_test}, the average expected coverage for HPDR with nominal coverage $1-\alpha$ can be estimated as
%
\begin{equation}
    1-\tilde \alpha = 
    \mathbb{E}_{\substack{
        \btheta^\ast \sim p(\btheta),\; 
        \bx \sim p(\bx \mid \btheta^\ast),\\[2pt]
        \btheta_{1:L} \sim \prod_{i=1}^L q_\phi(\btheta_i \mid \bx)
    }}\!\left[
    \mathbb{1}\!\left\{ \frac1L\sum_{i=1}^L \mathbb{1}(
    q_\phi(\btheta_i\mid \bx) \geq q_\phi(\btheta^\ast\mid \bx)) \leq 1-\alpha \right\}
    \right]\;.
\end{equation}
%
This expression looks somewhat cumbersome, but provides a precise and straightforward to implement prescription for Monte Carlo estimation.


\begin{figure}[t]
    \centering
    \includegraphics[width=0.49\linewidth]{figures/cov_shapes.png}
    \includegraphics[width=0.49\linewidth]{figures/cov_pp.png}
    \caption{Expected HPDI coverage for inferred posteriors with different systematics. \emph{Left:} Inferred posteriors \( q_\phi(\theta \mid \bx) \) compared to the ground truth \( p(\theta \mid \bx) \), assuming deviations are independent of \( \bx \). \emph{Right:} \( pp \)-plot of realized vs nominal coverage. Over- and underdispersed posteriors lead to systematic over- or undercoverage. Compared to SBC, the test is less sensitive to posterior bias.}
    \label{fig:hpdi_cov}
\end{figure}


\subsubsection{Interpreting Coverage Diagnostics}

Figure~\ref{fig:hpdi_cov} illustrates how biased, overdispersed, and underdispersed posteriors manifest in a coverage plot (or \emph{pp-plot}) comparing realized with nominal coverage. A few typical observations can be made:
\begin{itemize}
    \item \textbf{Bias} (shifted posteriors) lowers realized coverage with respect to nominal coverage, and leads typically to overconfident regions.
    \item \textbf{Overdispersion} (too wide posteriors) leads to underconfident regions, i.e., higher realized coverage than nominally expected.
    \item \textbf{Underdispersion} (too narrow posteriors) lowers realized coverage and leads to overconfident regions.
\end{itemize}
Notably, even strongly biased posteriors produce only mild deviations---whereas the same bias would yield a pronounced signal in SBC diagnostics (see Fig.~\ref{fig:SBC_basics} for comparison).\cw{TODO: Check if the plot is correct?]}

\medskip

Note that standard coverage plots emphasize the behaviour of regions with credibility in the range \(0.2 \lesssim 1-\alpha \lesssim 0.8\), whereas behaviour for higher credibility is compressed to the upper right corner of the plot. Yet, the most practically relevant values of \( \alpha \) are typically small—for instance, \( \alpha = 0.05 \) for a 95\% credible region. To better visualize miscalibration in these regimes, it is helpful to transform the credibility values \(1-\alpha\) to the corresponding \emph{z-scores}, \(z_{1-\alpha}\).\footnote{We define \(z_{1-\alpha}\) as the standard normal quantile such that the central interval \([-z_{1-\alpha}, z_{1-\alpha}]\) contains probability mass \(1-\alpha\). Equivalently, \(z_{1-\alpha} = \Phi^{-1}(1-\tfrac{\alpha}{2})\), where \(\Phi^{-1}(\cdot)\) is the inverse cumulative distribution function of the standard normal distribution.}
For example, a 95\% credible interval corresponds to \( z \approx 1.96 \), and a 99.7\% interval to \( z \approx 3 \). Plotting realized \( \tilde z \) against nominal \( z \) better highlights tail miscalibration, where under- or overcoverage may be particularly consequential.



\begin{figure}[t]
    \centering
    \includegraphics[width=0.49\linewidth]{figures/cov_shape_blind.png}
    \includegraphics[width=0.49\linewidth]{figures/cov_pp_blind.png}
    \caption{Examples of inferred posteriors \( q(\theta \mid \bx) \) that are specifically constructed to match the HPDI coverage of the true posterior \( p(\theta \mid \bx) \), despite visibly differing in shape and location. This illustrates a blind spot of HPDI-based diagnostics: they test only the posterior mass within super-level sets of \( q \), and are thus insensitive to mass redistribution along contour surfaces. The deviations shown here would be readily detected by SBC.}
    \label{fig:blind_cov}
\end{figure}


\subsubsection{Insensitivity to Posterior Shape Distortions}

The insensitivities of SBC apply to coverage diagnostics as well: HPDR coverage plots are completely insensitive to type~B (information loss) uncertainties; type~C (inference error) uncertainties can potentially cancel in the coverage plot due to averaging across the data space (for instance when some regions feature overdispersed and others underdispersed learned posteriors). However, HPDR-based coverage diagnostics also introduce a \emph{new}, arguably more subtle failure mode.

\paragraph{A new failure mode: shape distortions}

HPDR coverage tests evaluate whether intervals \(\mathcal{C}_{1-\alpha}(\bx)\) derived from the learned posterior contain the correct amount of mass under the true posterior \(p(\btheta \mid \bx)\). This requirement, however, does not automatically imply that the \emph{shape} of the interval corresponds correctly to the shape implied by the true posterior. In fact, a wide range of intervals can be constructed that satisfy the integral constraints while featuring incorrect shapes.

Figure~\ref{fig:blind_cov} illustrates this limitation with three pathological approximate posteriors that deviate markedly from the ground truth. Despite being obviously misshapen, each one yields a perfectly diagonal coverage plot. This is not accidental: these posteriors are constructed such that the posterior mass within every symmetric HPDR interval matches that of the true posterior, ensuring perfect coverage.\footnote{For the example, we choose a biased center \( \theta_0 \) and define \( s(\theta) = -|\theta - \theta_0| \) as the ordering statistic. The super-level sets \( I(w) = \{ \theta : |\theta - \theta_0| \leq w \} \) define symmetric intervals. Let \( M(w) = \int_{I(w)} p(\theta)\, d\theta \) be the true posterior mass within the interval. We then construct a density \( q(\theta) = g(|\theta - \theta_0|) \) such that \( g(w) = \tfrac{1}{2} M'(w) \), and normalize. This procedure ensures that all HPDR intervals under \( q \) match those under \( p \) by construction.}

\paragraph{Mitigation strategies} 

This example underscores an additional insensitivity of credible region coverage tests that is not present in SBC: coverage tests based on credible regions are invariant to a wide range of shape mismatches. Despite visual mismatches or implausible shapes, such distortions may remain entirely undetected by standard coverage diagnostics. Importantly, the blind directions depend on how credible regions are constructed. A mitigation strategy is therefore to combine multiple approaches—HPDR, central regions, upper and lower limits, as well as SBC tests—to achieve more comprehensive diagnostic coverage.


\subsection{Classifier Two-Sample Test (C2ST)}
\label{sec:diag:forward_back:c2st}

A natural strategy for forward-backward diagnostic tests is to train a neural classifier to distinguish samples from the forward model and samples from the backward model. If the classifier performs better than random, this provides clear evidence for a mismatch between the learned posterior and the true posterior.

The strategy is to train a classifier to distinguish between samples drawn from the inverse model \( q_\phi(\btheta \mid \bx)p(\bx) \) (labeled 1) and those drawn from the generative model \( p(\bx \mid \btheta)p(\btheta) \) (labeled 0). Classifier training is done analogously to NRE, see Sec.~\ref{sec:methods:ratios:nre}. The trained classifier, \(d_\phi(\btheta, \bx)\), provides the probability that a parameter-data pair \((\btheta, \bx)\) has been drawn from the backward model. A typical strategy is to define it in terms of a scalar network \(f_\phi\), as \(d_\phi(\btheta, \bx) = \sigma(f_\phi(\btheta, \bx))\). If the classifier is Bayes optimal, the scalar function approximates the log posterior ratio,
%
\begin{equation}
f_\phi(\btheta, \bx) \approx \log 
\frac{q_\phi(\btheta \mid \bx)p(\bx)}{p(\btheta, \bx)}
= \log\frac{q_\phi(\btheta \mid \bx)}{p(\btheta \mid \bx)}
\;.
\end{equation}
%
In situations where the learned posterior agrees exactly with the true one, the expected network output is zero, \(f_\phi(\btheta, \bx) \simeq 0\), and hence \(d_\phi(\btheta, \bx) \simeq \tfrac{1}{2}\). Deviations from these values could indicate posterior mismatches. In practice, because of finite training data and network capacity, deviations are always expected, and one has to carefully assess the their significance. We will here discuss different common strategies to quantify observed mismatches.


\subsubsection{ROC-Based Evaluation}

A common diagnostic approach in this context is the \emph{receiver operating characteristic} (ROC) curve. It is defined by plotting the true positive rate (TPR) against the false positive rate (FPR) for varying thresholds \( t \) on the classifier output:
\begin{align*}
\text{TPR}(t) &= \mathbb{E}_{\bx, \btheta \sim q_\phi(\btheta \mid \bx)p(\bx)} 
[\mathbb{1}(d_\phi(\btheta, \bx) > t)], \\
\text{FPR}(t) &= \mathbb{E}_{\bx, \btheta \sim p(\bx, \btheta)} 
[\mathbb{1}(d_\phi(\btheta, \bx) > t)], \\
\text{ROC} &= \{ (\text{FPR}(t), \text{TPR}(t)) : t \in \mathbb{R}\},
\end{align*}
where \( d_\phi(\btheta, \bx) = \sigma(f_\phi(\btheta, \bx)) \) denotes the classifier output.
If the two distributions are indistinguishable, the ROC curve lies along the diagonal, i.e.\ \( \text{TPR}(t) \approx \text{FPR}(t) \) for all \( t \). Deviations from the diagonal indicate that the classifier can distinguish between the two sample sets, and hence that the learned posterior \( q_\phi \) deviates from the true posterior \( p \).

A commonly used summary of the ROC curve is the \emph{area under the curve} (AUC). The AUC can be interpreted as the probability that the classifier assigns a higher score to a sample from \( q_\phi(\btheta \mid \bx)p(\bx) \) than to one from \( p(\bx, \btheta) \): 
\begin{equation}
\label{eqn:AUC}
\text{AUC} = \mathbb{E}_{q_\phi(\btheta\mid \bx)p(\bx, \btheta^\ast)}
\left[\mathbb{1}(f_\phi(\btheta^\ast, \bx) < f_\phi(\btheta, \bx))\right]\;.
\end{equation}
When \(q_\phi(\btheta \mid \bx) = p(\btheta \mid \bx)\), the expected value of the AUC is \emph{exactly} \( \tfrac{1}{2} \). It increases toward 1 as the distributions become more separable. Any statistically significant deviation from \(\tfrac{1}{2}\) (with respect to Monte Carlo noise) indicates a mismatch between the learned and true posterior.

For Bayes optimal classifiers, one can show that the AUC is directly related to the total variation (TV) distance that we discussed in Sec.~\ref{sec:diag:reference:divergences},
\begin{equation}
\text{AUC} 
\xrightarrow{\text{Bayes opt}}
\frac{1}{2} + \frac{1}{2} \, \mathbb{E}_{p(\bx)} \left[D_\text{TV}(q_\phi(\btheta\mid \bx) \| p(\btheta \mid \bx))\right]\;.
\end{equation}
The AUC is connected to the \emph{data-averaged} total variation divergence between the true and the learned posterior. Since the integrand is strictly non-negative, \(\text{TV} = 0\) (or equivalently \(\text{AUC} = \tfrac{1}{2}\)) would imply that \(q_\phi(\btheta \mid \bx) = p(\btheta \mid \bx)\) exactly.


\subsubsection{Connection to Rank-Based Tests}

Although the C2ST is usually not conceived as a rank-based test, there are clear connections. In fact, the AUC in Eq.~\eqref{eqn:AUC} can be interpreted as a rank-based test: we compare samples from the forward model \((\btheta^\ast, \bx)\) with a single sample from the learned posterior \(\btheta'\), effectively using \(L=1\) with \(f_\phi(\btheta, \bx)\) as the ordering function, similarly to Eq.~\eqref{eqn:HPDI_ranks} for HPDR-based coverage estimation.

Based on this logic, and on the rank-test samples from Eq.~\eqref{eqn:rank_test_samples_multi}, one can define an \(L>1\) generalization of the AUC, which gives the normalized rank statistic
%
\begin{equation}
    F_\bx(\btheta^\ast) = \frac{1}{L} \sum_{i=1}^L \mathbb{1}
    \left(f_\phi(\btheta_i, \bx) \geq f_\phi(\btheta^\ast, \bx)\right)
    \;.
    \label{eqn:C2ST_ranks}
\end{equation}
Any statistically significant deviation from uniformity would indicate a mismatch between learned and true posterior. Note that this test is robust against potential approximation errors in \(f_\phi\) itself, since \(f_\phi\) only acts here as an ordering function for comparing forward with backward samples, and does not quantify the difference by itself.

Put differently, this approach allows us to first optimize an ordering function \(f_\phi \) to pick up critical regions of learned vs.\ true posterior mismatch, and then quantify that mismatch through a rank-based test.


\paragraph{Insensitivity to information loss}

Just like SBC in Sec.~\ref{sec:diag:forward_back:sbc} and coverage diagnostics in Sec.~\ref{sec:diag:forward_back:coverage}, C2ST diagnostics may be insensitive to information loss in the learned posterior.  Whether or not this happens depends on how much information the learned classifier is able to extract from the data.

Consider the case where the learned posterior is based on a lossy summary, \(q_\phi(\btheta \mid \bx) = p(\btheta \mid T(\bx))\), and the classifier similarly extracts only limited information from \(\bx\) through some implicit summary \(U(\bx)\), such that \(f_\phi(\btheta, \bx) = g(\btheta, U(\bx))\). If the summaries form a Markov chain \(\bx \to T(\bx) \to U(\bx)\), meaning \(U\) discards at least as much information as \(T\), then the C2ST will fail to detect the information loss in \(q_\phi\). In short, whether C2ST detects lossy summaries depends on whether the classifier \(f_\phi\) can extract more information from \(\bx\) than the approximate posterior \(q_\phi\) does.


\subsubsection{Beyond Forward-Backward: Local Divergence Estimates}

Besides ROC-based and rank-based tests, the trained classifier $f_\phi(\btheta, \bx)$ can also serve as input to divergence estimates, as they were discussed above in Sec.~\ref{sec:diag:reference:divergences}.  This approach does not rely on rank test samples, but it is sufficient to have samples from the learned posterior only.  For instance, the $\chi^2$-divergence can be estimated for a given observation $\bxobs$~\citep{linhart_l-c2st_2023}
%
\begin{equation}
    D_{\chi^2}[q_\phi(\btheta \mid \bxobs) \mid\mid p(\btheta \mid \bxobs)] = \mathbb E_{q_\phi(\btheta \mid \bxobs)}
    \left[(1-e^{-f_\phi(\btheta; \bxobs)})^2\right]
\end{equation}
%
Similar expressions can be readily derived from the divergences in Tab.~\ref{tab:divergences}.  An advantage of the test is that it is local for a specific value of $\bxobs$.  A challenge is that even if $q_\phi(\btheta \mid \bx) = p(\btheta \mid \bx)$, approximation errors in the trained classifier can lead to deviations from zero.  In order to estimate the significance of non-zero values, one hence resort to Monte Carlo methods to estimate the null distribution of $D_\chi^2$, for instance by using training data with scrambled 0, 1 labels as suggested in~\cite{linhart_l-c2st_2023}.


\subsection{Model-Based Rank Diagnostics}
\label{sec:diag:forward_back:model_based}

So far we have discussed rank-based diagnostics using model parameters or learned quantities as ordering functions.  However, in some situations we might be able to use components of the simulation model itself. In fact, often simulation codes are implemented such that they allow to efficiently evaluate the likelihood density \( p(\bx \mid \btheta) \), or---through auto-differentiation---the score function \( \bs(\btheta; \bx) \equiv \nabla_{\btheta} \log p(\bx \mid \btheta) \). In situations where the true likelihood function is not tractable, even simplified analytic likelihoods would be a valid starting point for what we discuss here.

Since the likelihood density function encodes all information about how data and parameters are connected, it is natural that it should also serve as a useful ingredient for calibration diagnostics.  Moreover, since the likelihood depends directly on the full data \(\bx\) rather than on any learned summary, this approach offers a strategy for detecting type~B (information loss) uncertainties~\citep{modrak_simulation-based_2025}, which is a failure mode most other forward-backward tests remain insensitive to. 


\subsubsection{Rank Tests via Explicit Model Evaluation}

Once we have access to the likelihood density or score function of the simulation model, several powerful tests rank-based tests become possible, which we summarize briefly. All tests follow the same logic as SBC or coverage tests, they only differ in the adopted ordering function.

\paragraph{Likelihood ranks}

Using the (true) likelihood density $p(\bx \mid \btheta)$ as ordering function is straightforward---assuming it can be efficiently evaluated.  We generate rank test samples from Eq.~\eqref{eqn:rank_test_samples_multi} and evaluate for each sample the following normalized rank statistic: %
%
\begin{equation}
    F_\bx(\btheta^\ast) = \frac{1}{L} \sum_{i=1}^L \mathbb{1} \left(p(\bx \mid \btheta_i) \geq p(\bx \mid \btheta^\ast)\right)\;.
\end{equation}
%
If the learned and true posteriors agree, we expect a uniform distribution of $F_\bx(\btheta^\ast)$, while deviations indicate a mismatch.  Note that, as for the coverage tests discussed in Sec.~\ref{sec:diag:forward_back:coverage}, the ordering function is observation dependent.  We will refer to this particular diagnostic as \emph{likelihood rank diagnostics}.  This diagnostic was introduced by~\cite{modrak_simulation-based_2025}.  


\paragraph{True HPDR coverage} 

Alternatively, if both the model likelihood density $p(\bx \mid \btheta)$ and prior $p(\btheta)$ can be evaluated, it is straightforward to use as ordering function the unnormalised posterior, $\pi(\btheta \mid \bx) \equiv p(\bx \mid \btheta) p(\btheta)$,
%
\begin{equation}
    F_\bx(\btheta^\ast) = \frac{1}{L} \sum_{i=1}^L \mathbb{1} \left(\pi(\btheta_i\mid \bx) \geq \pi(\btheta^\ast\mid \bx)\right)\;.
\end{equation}
Importantly, this is \emph{equivalent to ordering by the \emph{true} posterior, $p(\btheta \mid \bx)$}, since both ordering functions only differ by a data dependent constant (which leaves the ordering unchanged).  These definitions closely parallel the one in Eq.~\eqref{eqn:HPDI_ranks}, but with a crucial difference: Instead of relying on intervals defined through an learned posterior \( q_\phi(\btheta \mid \bx) \), the intervals here are based on contours of the true posterior \( p(\btheta \mid \bx) \).  
\medskip

Notably, \emph{this test actually probes the coverage of the \emph{true} highest posterior density regions}, in contrast to traditional coverage tests discussed in Sec.~\ref{sec:diag:forward_back:coverage}, which only test coverage of the learned posteriors.  This will be discussed further in Sec.~\ref{sec:diag:forward_back:general}.


\paragraph{Score ranks}

Lastly, if the (true) score function $\bs(\btheta; \bx) \equiv \nabla_{\btheta} \log p(\bx \mid \btheta)$ of the model can be evaluated, this enables a large number of additional tests.  For each components $j$ of the score function, a rank statistic can be defined
%
\begin{equation}
    F^{(j)}_\bx(\btheta^\ast) = \frac{1}{L} \sum_{i=1}^L \mathbb{1} \left(
    s_j(\bx \mid \btheta_i) \geq 
    s_j(\bx \mid \btheta^\ast)\right)\;,
\end{equation}
%
leading to as many independent tests as there are parameters. This mirrors the structure of SBC, see Sec.~\ref{sec:diag:forward_back:sbc}.  In fact, in simple situations like linear regression, where the score function becomes linear in $\btheta$, both tests are closely related.  

\medskip

A key advantage of the score function as an ordering function, with respect to likelihood or posterior density, is that it generally enables (similar to SBC) \emph{parameter specific tests}, whereas likelihood or true posterior rank diagnostics only provide aggregated information.


\subsubsection{Sensitivity to Information Loss}

As shown by~\cite{modrak_simulation-based_2025}, the true likelihood rank diagnostics is sensitive to type B (information loss) uncertainties. In contrast to other rank tests that we discussed above, see Sec.~\ref{sec:diag:forward_back:sbc:information}, this test can reveal inference pathologies due to lossy summaries that traditional highest-posterior density interval (HPDI) tests may completely overlook.

In Fig.~\ref{fig:SBC_insufficiency_solved}, we illustrate a case in which the use of a lossy summary statistic leads to type~B uncertainty. Explicit posterior-based calibration reveals the deficiency, whereas traditional HPDI-based SBC tests remain blind to it. \cw{TODO: Update text once figure is updated}

Model-based rank diagnostics is a sensitive method for testing the fidelity of learned posteriors.  However, we still expect that the above model-based rank diagnostics suffer from similar insensitiveness to spatially varying bias and shape distortions as we discussed earlier in this section. Only a combination of multiple different rank diagnostics can provide increasing certainty.

\begin{figure}[th]
    \centering
    \includegraphics[width=0.49\linewidth]{figures/SBC_shapes_suboptimal.png}
    \includegraphics[width=0.49\linewidth]{figures/ELC_vs_HPDI_comparison.png}
    \caption{HPDR coverage test compared to likelihood-based rank diagnostic. \cw{TODO: Update caption and figure}}
    \label{fig:SBC_insufficiency_solved}
\end{figure}



% TODO: Explore other options
%\subsubsection{Other methods}

%\cw{TODO: Add discussion about other methods (TARP + Score) or remove}

%\paragraph{Reference points}
%Beyond posterior-based diagnostics, one can also construct calibration tests based on the likelihood or the score function. For example, one may define an ordering function as
%\begin{equation}
%\rho_\bx(\btheta) = s_i(\btheta; \bx)^k,
%\end{equation}
%where \( s_i(\btheta; \bx) \) is a component of the score vector, and \( k = 1, 2, 3, \dots \) is a power index that determines the sensitivity of the test. These score-based rank diagnostics offer a flexible family of calibration tools that can be tailored to specific scientific questions or model structures.


\subsection{Generalized Rank Diagnostics}
\label{sec:diag:forward_back:general}

\begin{table}[t]
    \centering
    \begin{tabular}{llc>{\raggedright\arraybackslash}p{4.5cm}}
    \toprule
    Method & Ranks by & Type B & Ordering function $f(\btheta, \bx)$ \\
    \midrule
    \addlinespace[0.5em]
    \multicolumn{4}{l}{\textit{Common rank diagnostics}} \\
    SBC & Parameter ($d$ tests) & No & $\theta_i$ for $i=1, \dots, d$ \\
    HPDR coverage & Learned posterior & No & $q_\phi(\btheta \mid \bx)$ \\
    \addlinespace[1.0em]
    \multicolumn{4}{l}{\textit{Distance and classifier-based rank diagnostics}} \\
    TARP (x-indep) & Distance to reference & No & $\|\btheta - \btheta'\|^2$ where $\btheta' \sim p(\btheta)$ \\
    TARP (x-dep) & Distance to reference & Maybe & $\|\btheta - \btheta_r(\bx)\|^2$\\
    C2ST & Posterior ratio & Maybe & $f_\phi(\btheta; \bx) \approx \log  \frac{q_\phi(\btheta \mid \bx)}{p(\btheta \mid \bx)}$ \\
    \addlinespace[1.0em]
    \multicolumn{4}{l}{\textit{Model-based rank diagnostics --- requires tractable model}} \\
    True HPDR & True posterior & Yes & $\pi(\btheta \mid \bx) \propto p(\bx\mid\btheta) p(\btheta)$\\
    Likelihood ranks & Likelihood & Yes & $p(\bx \mid \btheta)$ \\
    Score ranks & Score ($d$ tests) & Yes & $\nabla_{\theta_i}\log p(\bx \mid \btheta)$ \\
    \bottomrule
    \end{tabular}
    \caption{Generalized rank diagnostics with their ordering functions $f(\btheta, \bx)$ and Type B (information loss) sensitivity.}
    \label{tab:rank_based_test}
\end{table}

The various tests discussed in the previous sections are all examples of forward-backward diagnostics, and have as such in common that they all rely on comparing samples from the inverse mode $q_\phi(\btheta \mid \bx) p(\bx)$ with samples from the forward model $p(\bx, \btheta)$.   Importantly, most forward-backward tests can be interpreted as variants of the same underlying rank-based analysis framework.  This generalized view is starting to receive increasing attention in the literature~\citep{lemos_sampling-based_2023, modrak_simulation-based_2025}, which make it possible to better understand the role of individual rank-based test and provides a framework for comprehensive comparison of learned and true posteriors.


\subsubsection{Framework and Methodology}

General rank diagnostics follow the same structural logic as, \fex\ SBC and HPDI diagnostics, but generalize the ranking rule by supporting arbitrary ordering functions, \( f(\btheta, \bx) \).  The role of the ordering function is to provide a way to order values of $\btheta$ for rank calculation. In general, ordering functions can depend on the observation $\bx$.

As before, we generate rank-test samples in Eq.~\eqref{eqn:rank_test_samples_multi}.  Then, given the ordering function \( f(\btheta, \bx) \), we compute the rank statistic
%
\begin{equation}
    F_{\bx}(\btheta^\ast) = \frac{1}{L} \sum_{i=1}^L 
    \mathbb{1} \left( f(\btheta_i, \bx) \geq f(\btheta^\ast, \bx) \right)\;.
    \label{eqn:general_rank_statistic}
\end{equation}
%
This quantifies the position of the true parameter \( \btheta^\ast \) among the posterior samples, under the ordering imposed by \( f(\btheta, \bx) \).
%
Under the null hypothesis that $p(\btheta \mid \bx) = q_\phi(\btheta \mid \bx)$, we expect that the rank statistic is uniformly distributed.  Significant deviations from uniformity imply a mismatch between learned and true posterior. 

\paragraph{Examples} In Tab.~\ref{tab:rank_based_test}, we list the test that we discussed so far in this section, and the corresponding ordering function.  For instance, in the case of simulation-based calibration (SBC, see Sec.~\ref{sec:diag:forward_back:sbc}), the ordering function does not depend on $\bx$ and is simply given by the parameter of interest, $\theta_i$.  On the other hand, in the case of HPDI coverage tests (see Sec.~\ref{sec:diag:forward_back:coverage}), the ordering function is $\bx$ dependent and given by the learned posterior density, $q_\phi(\btheta \mid \bx)$.

For completeness, we note that the ordering function supports also a stochastic component.  This was exploited for the TARP algorithm (Test of Accuracy with Random Points, \citep{lemos_sampling-based_2023}), where the ordering is given by $f(\btheta, \bx) = \| \btheta - \btheta'\|^2$, and $\btheta' \sim p(\btheta)$ is an additional contrastive parameter randomly drawn from the prior each time when sampling $\bx \sim p(\bx\mid \btheta^\ast)$.


\subsubsection{Implications of Passing the Test}

General rank diagnostics define a broad class of tests that assess whether an learned posterior $q_\phi(\btheta \mid \bx)$ assigns appropriate probability mass, as given by $p(\btheta \mid \bx)$, to a family of regions in parameter space $\Theta$ that are defined as superlevel sets of an ordering function, $f(\btheta, \bx)$.  For a given level $t\in \mathbb{R}$, the region is given by
%
\begin{equation}
    \Theta_{t}(\bx) = \{ \btheta : f(\btheta, \bx) \geq t\}\;.
\end{equation}
%
By construction, these regions are nested, $\Theta_t(\bx) \subseteq \Theta_{t'}(\bx)$ for $t' < t$.

It is instructive to consider the large samples limit, where $L \gg 1$ and the number of rank test samples is large.  In that case, passing a given rank diagnostic test implies that
%
\begin{equation}
    \bbE_{p(\bx)} 
    \left[\int_{f(\btheta, \bx) \geq t} d\btheta\, 
    [q_\phi(\btheta \mid \bx)  - p(\btheta \mid \bx)]
    \right] = 0 \quad \text{for all} \quad t \in \mathbb R
    \label{eqn:general_rank_diagnostic_result}
\end{equation}
%
Different ordering functions give rise to the different tests. Examples include $f(\btheta, \bx) = q_\phi(\btheta \mid \bx)$, which tests the average mass in the HPDRs defined by the learned posterior, or $f(\btheta, \bx) = \pi(\btheta \mid \bx)$, which tests the average mass in the HPDRs of the true posterior. But the intuition provided by Eq.~\eqref{eqn:general_rank_diagnostic_result} also opens up the possibility to define new targeted tests for specific inference tasks.


\paragraph{Completeness Guarantees}

As shown in Eq.~\eqref{eqn:general_rank_diagnostic_result}, each choice of ordering function probes a specific one-dimensional projection of the learned and true posterior functions. 
It turns out that consistency across \emph{all} possible functions $f(\btheta, \bx)$ implies global consistency.
Indeed, \cite{lemos_sampling-based_2023} demonstrated that if \( F_{\bx}(\btheta^\ast) \) is uniformly distributed for \emph{every} choice of \( f(\btheta, \bx) \), then it implies that \( q_\phi(\btheta \mid \bx) = p(\btheta \mid \bx) \) almost everywhere. 

While full coverage over all possible ordering functions is practically impossible, this highlights the fact that rank-based tests are, in principle, complete. The framework thus motivates the use of diverse, problem-specific ordering functions to uncover inference errors that might escape standard diagnostics like SBC or HPDI. 


%%%%%%%%%%%%%%%%%%%%%%%%%%%%%%%%%%%%%%%%%%%%%%%%%%%%%%%%%%%%%%%%%%%%%%%
%%%%%%%%%%%%%%%%%%%%%%%%%%%%%%%%%%%%%%%%%%%%%%%%%%%%%%%%%%%%%%%%%%%%%%%
\section{Model Misspecification Diagnostics}
\label{sec:diag:misspec}
%%%%%%%%%%%%%%%%%%%%%%%%%%%%%%%%%%%%%%%%%%%%%%%%%%%%%%%%%%%%%%%%%%%%%%%
%%%%%%%%%%%%%%%%%%%%%%%%%%%%%%%%%%%%%%%%%%%%%%%%%%%%%%%%%%%%%%%%%%%%%%%


%\begin{quotation}
%\noindent
%\emph{All models are wrong, but some are useful.} — George Box
%\cw{TODO: Check quote}
%\end{quotation}

\begin{quotation}
    \textit{``It all looked so easy when you did it on paper — where valves never froze, gyros never drifted, and rocket motors did not blow up in your face.''}
    
    \hfill --- Mary F. Shafer, rocket engineer \cw{TODO: confirm}
    
\end{quotation}

\noindent
Simulation-based inference is built on simulation models that inevitably reflect assumptions, deliberate simplifications, limited knowledge, and inductive biases. Determining whether a model adequately describes the data, and whether scientific conclusions can be trusted or might be affected by the model's shortcomings, is central to any rigorous analysis. 

Here we focus on methods for model criticism and misspecification detection, which test whether the assumed generative model is consistent with observations:
\begin{equation}
\nonumber
p(\bx \mid \btheta)p(\btheta)
\;\;\xleftrightarrow{\text{consistent with?}}\;\;
\bxobs
\;.
\end{equation}
What precisely consistency entails depends on the specific analysis goals. No single universal method can confirm consistency for all use cases. The techniques presented here provide a toolkit for careful, self-critical validation.

All approaches in this section address \textbf{Type A} (misspecified model) epistemic uncertainty from Sec.~\ref{sec:diag:taxonomy}. While \textbf{Type B} (lossy summary) and \textbf{Type C} (inexact inference) uncertainties can be diagnosed through reference posterior comparisons and calibration tests (Secs.~\ref{sec:diag:reference} and~\ref{sec:diag:forward_back}), controlled lossy summaries can also \emph{mitigate} model misspecification consequences, as explored in Sec.~\ref{sec:adv:uncertainty:robust}.

\subsection{Robustness Diagnostics}
\label{sec:diag:misspec:robustness}

\begin{quotation}
    \textit{``Does it make sense?''}
    
    \hfill --- Your inner supervisor
\end{quotation}

\noindent
Robustness diagnostics assess whether learned posteriors remain stable under controlled perturbations to the data or the inference pipeline. In a well-specified model, the posterior should reflect only aleatoric uncertainty. Small, structured modifications to the input or modeling assumptions should alter the outcome in a limited and often predictable way. If inference is unusually sensitive to such changes, this indicates deeper failure modes, including type~A uncertainty from model mismatch.

Although a wide range of robust tests can be conceived, we focus here on two common scenarios: (i)~\textit{masking parts of the observed data}, and (ii)~\textit{modifying the summary network used for feature extraction}. These perturbations help reveal over-reliance on specific input features or sensitivity to learned representations—common failure modes in practice. 
\cw{TODO: Update once structure is settled}


\subsubsection{Inference Consistency under Data Masking} 

\cw{TODO: Add references}

A common robustness check is to repeat the analysis on systematically reduced or filtered versions of the data. Such reductions may correspond to excluding certain spatial regions, frequency bands, detector channels, or other domain-specific structures. Formally, let \(\{M_k\}\) be a collection of mask operators acting on the original data \(\bx\), producing \(\bx_k = M_k(\bx)\). Each analysis then trains the inference model on simulations processed by the same operator and applies the resulting posterior to the correspondingly transformed observation, \(\bxobs_{,k} = M_k(\bxobs)\).\footnote{In a likelihood-based analysis this would be considerably more involved, since one would need a closed-form likelihood for the reduced data, \(p(\bx_k \mid \btheta)\), which is often intractable even when the full-data likelihood is known.} If the simulation model matches the true data-generating process, results based on different masked versions should agree within expected posterior variability, typically yielding overlapping high-probability regions. Conversely, if the simulator is misspecified, biases may vary across masks and appear as discrepancies between the corresponding inference results.


\paragraph{A hypothesis-testing view} 

From each masked version of the data \( \bx_i = M_i(\bx) \), we obtain a learned posterior \( q_\phi^{(i)}(\btheta \mid \bx_i) \). How can we compare these posteriors meaningfully? Since different masks reveal different aspects of the data, some variation across posteriors is expected; therefore simple discrepancy measures such as the KL divergence, which penalize any difference, are too strict for this purpose.

Drawing on ideas from \emph{posterior conflict diagnostics} in hierarchical models~\cite{xxx}, it is natural to frame the comparison as a hypothesis test. The question is whether the observed masked datasets, \(\bxobs_{,i}\) and \(\bxobs_{,j}\), are compatible with having arisen from the same underlying generative process (\(\mathcal{H}_0\)), or whether they behave as if they were generated independently after masking (\(\mathcal{H}_1\)). That is,
\[
\mathcal{H}_0:\; \bx_i,\, \bx_j \sim p(\bx_i, \bx_j)
\qquad\text{vs.}\qquad
\mathcal{H}_1:\; \bx_i,\, \bx_j \sim p(\bx_i)\, p(\bx_j)\;,
\]
where under \(\mathcal{H}_0\) the joint distribution arises by masking draws from the simulation model,
\[
p(\bx_i, \bx_j) = \int d\bx\, d\btheta\, \delta(\bx_i - M_i(\bx))\, \delta(\bx_j - M_j(\bx))\, p(\bx,\btheta)\;.
\]
Under \(\mathcal{H}_1\), the two masked views behave as though they contain incompatible information about \(\btheta\).

\paragraph{Posterior-overlap statistic} The optimal test statistic for discriminating \(\mathcal{H}_0\) and \(\mathcal{H}_1\) is the log-likelihood ratio
\begin{equation}
    \log R(\bx_i, \bx_j) = \log \frac{p(\bx_i, \bx_j)}{p(\bx_i)\, p(\bx_j)} \simeq \log \int d\btheta\, \frac{q_\phi(\btheta \mid \bx_i)\, q_\phi(\btheta \mid \bx_j)}{p(\btheta)}\;.
    \label{eqn:posterior_overlap_integral}
\end{equation}
On the right-hand side we introduced the \emph{posterior overlap integral}, which measures agreement between posteriors by comparing them to the prior. The integral is symmetric in \(i\) and \(j\), easy to evaluate by Monte Carlo sampling from the learned posteriors (assuming their densities are tractable), and becomes large when the posteriors concentrate on similar regions of parameter space.

The approximation in Eq.~\eqref{eqn:posterior_overlap_integral} becomes exact when the learned posteriors equal the true ones, \(q_\phi(\btheta \mid \bx_k) = p(\btheta \mid \bx_k)\), and when the masked datasets are conditionally independent given \(\btheta\), \(p(\bx_i,\bx_j \mid \btheta) = p(\bx_i \mid \btheta)\, p(\bx_j \mid \btheta)\). The latter holds when the two masks select disjoint or only weakly overlapping regions of the data. Violations of this condition reduce the power of the test but do not invalidate it.

\paragraph{Practical evaluation} Using Eq.~\eqref{eqn:posterior_overlap_integral}, one may directly evaluate the test statistic $\log R(\bxobs_{,i}, \bxobs_{,j})$ for the observed data. Small values indicate inconsistencies between masked views; large values indicate agreement. To quantify statistical significance, one generates Monte Carlo pairs \((\bx,\btheta)\sim p(\bx\mid\btheta)\,p(\btheta)\), applies masks \(M_i\), \(M_j\), and computes \(\log R(\bx_i,\bx_j)\) on these synthetic datasets to obtain \(p\)-values. For more than two masks, the collection of pairwise statistics \(\log R(\bxobs_{,i},\bxobs_{,j})\) provides a convenient diagnostic summary.

\paragraph{Example}

\cw{TODO: Remove example and figure, make text consistent with missing example}

%A global consistency measure compares all \( K \) masked posteriors via
%\[
%\log R_{\text{global}} = \log \int d\btheta\, \frac{\prod_{i=1}^K q_\phi(\btheta \mid \bx_i)}{p(\btheta)^{K-1}} \,,
%\]
%which tests whether all masked subsets agree on a common parameter value. While this global test may be more powerful, it is also more sensitive to numerical instability and typically has higher variance in high-dimensional problems.

\begin{figure}
        \centering
        \includegraphics[width=0.49\linewidth]{figures/simulated_data.png}
        \includegraphics[width=0.49\linewidth]{figures/posterior_contours.png}
        \caption{Enter Caption, R=9.6, R=3.3, p=0.57, p=0.014}
        \label{fig:enter-label}
\end{figure}


\subsection{Posterior Predictive Diagnostics}
\label{sec:diag:misspec:ppc}

A basic principle in model criticism is that the observed data \( \bxobs \) should be typical under the assumed simulation model \( p(\bx \mid \btheta) \). 
\emph{Posterior predictive checks} (PPCs) aim to detect systematic discrepancies between the observation and the simulator by comparing \( \bxobs \) to draws from the \emph{posterior predictive distribution}~\citep{xxx, GelmanMengStern1996},
\[
p(\tilde\bx\mid \bxobs) = \int  d\btheta \;
p(\tilde \bx \mid \btheta)\, p(\btheta \mid \bxobs)\, 
\; .
\]
This distribution represents the expected variability of new data, conditional on having generated the observation in the first place. 
In SBI, the exact posterior is replaced by a learned approximation \(q_\phi(\btheta \mid \bxobs)\), but the conceptual structure is unchanged.

In contrast to prior predictive checks---which test whether \( \bxobs \) is typical under the marginal distribution \(p(\bx)\)---PPCs focus on plausible parameter ranges given $\bxobs$ and are therefore more directly aligned with the observed data.  As such, PPCs can also be viewed as a posterior-predictive analogue of anomaly detection, checking whether \(\bxobs\) falls within the typical region of the predictive distribution. 


\subsubsection{Posterior Predictive Checks}
\label{sec:diag:misspec:ppc:checks}

\cw{TODO: Add references}

PPCs in the classic sense of~\citep{GelmanMengStern1996} do not compare the raw data $\bx$ directly. Instead, one evaluates a \emph{discrepancy function} $D(\bx, \btheta)$ that targets specific features of the data predicted by the model~\cite{xxx}. In many applications $D$ is simply a function of \(\bx\), but allowing a dependence on \(\btheta\) enables conditional or residual-style checks that may be more powerful for some types of mismatch.

Given posterior samples \(\btheta_i \sim q_\phi(\btheta \mid \bxobs)\), one computes
\[
D_{\mathrm{obs}}^{(i)} = D(\bxobs,\btheta_i),
\qquad
D_{\mathrm{sim}}^{(i)} = D(\tilde\bx_i,\btheta_i),
\quad
\tilde\bx_i \sim p(\bx \mid \btheta_i).
\]
Under a well-specified model, the values \(D_{\mathrm{obs}}^{(s)}\) should be \emph{typical} relative to the replicated values \(D_{\mathrm{sim}}^{(s)}\): they should not systematically lie in the extreme tails of the posterior predictive distribution. Posterior predictive \(p\)-values,
\[
p_{\mathrm{PPC}} \approx \frac{1}{N} \sum_{i=1}^N\mathbb{1}\!\left( D_{\mathrm{sim}}^{(i)} \ge D_{\mathrm{obs}}^{(i)} \right),
\]
are typically conservative~\cite{GelmanMengStern1996}: even under a correct model they are not uniformly distributed but squeezed around 0.5.  However, values near 0 or 1 still indicate clear mismatches, although the precise values are only of indicative value.

\paragraph{Choosing \(D\): structured discrepancies for SBI}

The usefulness of a PPC depends critically on the choice of summary \(D\). This choice is often guided by domain knowledge and by the structure of the simulator. Good discrepancy functions capture stable, low-variance aspects of the data that the model is expected to reproduce. Examples include amplitudes, variances, residual structure, frequency content, or more domain-specific descriptors.

A common and effective choice in high-dimensional settings is to construct \emph{residual maps} from posterior predictive draws,
\[
R^{(i)} = \bxobs - \tilde\bx_i,
\]
and base \(D\) on simple functions of these maps (\fex\ local values, power spectra, or other spatial statistics). Structured residual patterns can reveal systematic mismatches that would remain invisible in other more aggregate scalar summaries.

In settings with learned summaries, one may also define \(D\) in the latent space of the embedding network itself. For example
\[
D(\bx) = \| f_\phi(\bx) - f_\phi(\bxobs) \|^2,
\]
which checks whether the observed data are typical with respect to the posterior predictive distribution after projection into the learned representation. The PPC thus evaluates model fit in latent space, assessing whether the embedding of the observed data lies in the same typical region as embeddings of posterior predictive samples.

\medskip

To summarise, a typical PPC proceeds by (i) sampling \(\btheta_i\sim q_\phi(\btheta\mid\bxobs)\), (ii) drawing posterior predictive simulations \(\tilde\bx_i\sim p(\bx\mid\btheta_i)\), (iii) evaluating \(D_{\mathrm{obs}}^{(i)}\) and \(D_{\mathrm{sim}}^{(i)}\), and (iv) assessing whether the observed discrepancy is typical relative to the simulated ones, either just visually or via tail probabilities. If the model is well-specified, the observed summary should fall comfortably within the predictive distribution; if not, systematic deviations indicate potential simulator mis-specification.


\subsubsection{Log Predictive Density-Based Checks}

\cw{TODO: Add references}

If the exact likelihood function $p(\bx\mid\btheta)$, or a accurate likelihood surrogate \(q_\phi(\bx \mid \btheta)\) is available, one can also consider a density-based discrepancy such as the \emph{log predictive density} (LPD),
\[
\mathrm{LPD}(\bxobs) \simeq \log \frac{1}{N} \sum_{i=1}^N q_\phi(\bxobs \mid \btheta_i),
\qquad \btheta_i \sim q_\phi(\btheta \mid \bxobs).
\]
This yields a \(p\)-value by comparing \(\mathrm{LPD}(\bxobs)\) to its distribution under simulated data~\cite{xxx}. While conceptually valid, LPD-based checks are often weak in high-dimensional settings and sensitive to errors in the likelihood approximation. In practice, summary-based PPCs tend to provide more interpretable and sensitive sensitive diagnostics.


\subsection{Comparative Model Diagnostics}
\label{sec:diag:misspec:comparative}

\begin{quotation}
    \textit{``[...] the first step in the analysis of any decision problem is necessarily a purely intuitive selection by the decision maker of those courses of action which seem to him worthy of further consideration. Only after a set of “reasonable contenders” has thus been defined does it become \emph{possible} to apply formal procedures for choice among them [...]''}

    \hfill --- \cite{raiffa_applied_1961}
\end{quotation}


\noindent
A central challenge in model criticism is to decide whether the observed data \( \bxobs \) is typical under the assumed generative model. But this task is subtle: what deviations from the model should we test for, and how should we prioritize them? There is no general answer; without structure or assumptions, no diagnostic can be broadly powerful to \emph{all} deviations. This is a version of the \emph{“no free lunch”} principle: just as optimization requires problem-specific guidance, model criticism requires a notion of what kinds of discrepancies are meaningful.

\cw{TODO: Add references throughout}

These questions already appeared when choosing discrepancy functions for PPCs in Sec.~\ref{sec:diag:misspec:ppc}. The strength of PPCs is that they are easy to apply as soon as samples from the posterior predictive distribution are available. However, PPCs express mismatch only indirectly---via tail probabilities or visual comparisons---and their power depends entirely on the chosen summary \(D\).

\medskip

The diagnostics developed in this section follow a similar logic but frame the problem directly as hypothesis testing. This requires additional infrastructure, such as training discriminative networks, but enables more powerful and more targeted tests as well as principled significance calculations.


\subsubsection{Structured Hypothesis Tests for Model Criticism}

The basic idea is to define a structured family of augmented simulators, each
encoding a specific deviation from the baseline model, following ideas from~\cite{xxx}.
By comparing the baseline model to these structured alternatives, we can cast
model criticism into an interpretable battery of hypothesis tests---each targeting
a different type of possible misspecification.

\paragraph{Localized and aggregated deviation tests}

Let \( \mathcal{H}_0 \) denote the baseline simulator, and let each \( \mathcal{H}_i \) for
\( i \geq 1 \) correspond to an augmented simulator encoding a specific distortion.
These may be local (e.g., additive bumps, regional biases) or global (e.g., changes
in overall variance or shape). Each hypothesis defines a data distribution \( p_i(\bx) \),
with \( p_0(\bx) \equiv p(\bx) \) under the baseline.

For each augmented hypothesis, the optimal test statistic (in the sense of the
Neyman--Pearson lemma) is the log-likelihood ratio
\begin{equation}
\label{eqn:aug_ti}
t_i(\bxobs) = \log \frac{p_i(\bxobs)}{p_0(\bxobs)}
\approx f_\phi^{(i)}(\bxobs)\;.
\end{equation}
Here, \( t_i(\bx) \) can be approximated by training a binary classifier to distinguish
samples from \(p_i\) and \(p_0\). Above we assumed that the classifier
\(f_\phi^{(i)}(\bx)\) is trained to approximate the logit score separating the two classes.
In practice, when distortions correspond to structured changes in space
(e.g., image domains), classifiers can be trained jointly using shared neural
architectures such as U-Nets, with each output channel targeting a different
hypothesis \( i \).

The test statistic in Eq.~\eqref{eqn:aug_ti} focuses on targeted, isolated distortions
in the data. Searches based on this statistic assume that only one or a few distinct
distortions are present. If instead it is plausible that many distortions contribute
at similar levels, it is natural to consider an aggregate statistic, the simplest of
which is the sum
\begin{equation}
    t_{\mathrm{sum}}(\bxobs) = \sum_{i=1}^N t_i(\bxobs)\;.
\end{equation}
In fact,~\cite{xxx} showed that this construction generalizes the classical
\( \chi^2 \) goodness-of-fit test in likelihood-based analyses. We here implicitly
assume a uniform prior over distortion types. If the test battery is unbalanced or
dominated by uninformative directions, the global statistic may be biased or diluted;
careful design of the distortion set is essential.

More generally, these approaches unify several classical ideas~\cite{xxx}. Localized
distortions and their associated tests are closely related to matched filters and
signal-to-noise ratio statistics. Aggregated discrepancy statistics reduce to
\( \chi^2 \) tests when distortions are orthogonal and noise is Gaussian. In this
sense, simulator augmentation extends classical residual analysis and goodness-of-fit
testing to the flexible settings of simulation-based inference.

\paragraph{Local and global significance}

It is common in model criticism to adopt a frequentist perspective when evaluating
the significance of discrepancies, since priors over different model deviations are
typically unknown. Each test statistic \( t_i \) gives rise to a local significance level,
\[
p_i = \mathbb{E}_{\bx \sim p_0}\!\left[\, \mathbb{1}\!\left(t_i(\bx) \geq t_i(\bxobs)\right)\right],
\]
which quantifies the evidence against \( \mathcal{H}_0 \) in favor of the structured
deviation \( \mathcal{H}_i \). These \emph{local \(p\)-values} target specific, localized
mismatches and resemble matched filtering or bump hunting in high-energy physics and
signal processing~\cite{xxx}. An analogous definition applies to the aggregate statistic
\( t_{\mathrm{sum}} \).

\medskip

Since many hypotheses are tested simultaneously, the chance of picking up a random
fluctuation increases. To assess the global (overall) significance of the observed
deviations, one performs a trial correction. A common approach is to compute a
\emph{global \(p\)-value} based on the most extreme local result (including any
aggregated tests),
\[
p_{\mathrm{global}}
= \mathbb{E}_{\bx\sim p_0}\!\left[\,
   \mathbb{1}\!\left( \min_i p_i \leq \min_i p_i(\bxobs) \right)
 \right]\;.
\]
In practice, this global \(p\)-value helps assess whether observed discrepancies are substantial enough to warrant further investigation.

\chapter{Advanced Topics}
\label{chap:adv}

\section{Inference Under Simulator Uncertainty}
\label{sec:adv:uncertainty}

Model uncertainty is a central concern in any form of model-driven inference, and simulation-based inference is no exception. In practice, simulators are inevitably approximations: they incorporate unknown biases, omit physical effects, and make simplifying assumptions. Using an incorrect model introduces type~A epistemic uncertainties, arising from mismatches between the assumed and the true data-generating process. A critical step toward increasing the practical applicability of SBI is therefore to develop methods that make inference resilient to such mismodeling---ideally, procedures that remain reliable even when the simulator is inaccurate in certain respects. In what follows, we formalize the problem and identify principled strategies for achieving robustness.


\subsection{Defining Robustness in Simulation-Based Inference}
\label{sec:adv:uncertainty:defining}


\subsubsection{Model Misspecification in SBI}

The starting point for discussions of model misspecification and Bayesian inference is the (hypothetical) true data-generating process \(p_0(\bx)\)~\citep[\fex][]{kleijn_misspecification_2006, shalizi_dynamics_2009}. It represents the distribution of observations produced by nature, including all physical, instrumental, and environmental effects. In practice, $p_0$ is neither known nor can it be directly simulated; it serves only as a conceptual reference for reasoning about correctness and bias.

In context of the machine-learning literature, a model \(p(\bx\mid\bgamma)\) is often considered well-specified if there exists a configuration \(\bgamma^\ast\) such that \(p(\bx\mid\bgamma^\ast)=p_0(\bx)\)~\citep[\fex][]{cannon_investigating_2022}. However, in the physical sciences we are not primarily interested in matching the overall distribution of observed data.  What matters is whether the simulator captures the \emph{conditional} mechanism linking the physical parameters of interest, \(\btheta\), to observations. 

To formalise this, we write the simulator-based generative model as
\begin{equation}
    p(\bx \mid \btheta, \bgamma)\;,
    \label{eq:sbi-generative-model}
\end{equation}
where $\btheta$ denotes the physical parameters of interest and $\bgamma$ 
indexes the simulator configuration, implementation choices, approximations, 
or unmodelled effects. Adequacy of the theoretical model requires that, for some simulator configuration setting \(\bgamma^\ast\),
%
\begin{equation}
p(\bx\mid\btheta,\bgamma^\ast)\approx p_0(\bx\mid\btheta)
\quad\text{for all relevant }\btheta\;,
\end{equation}
%
where \(p_0(\bx\mid\btheta)\) denotes the (hypothetical) \emph{true conditional data-generating mechanism}. This distinction between matching the data distribution and capturing the conditional mechanism is essential for defining robustness in SBI.

The role of \(\bgamma\) remains here intentionally broad. It may index different simulator versions, parametrise uncertain or imperfectly known components, encode missing physical effects, or simply stand in for aspects of the generative mechanism that the simulator does not reliably capture. In practice, the corresponding “true’’ configuration \(\bgamma^\ast\) is unknown and often not even meaningfully parameterisable; it serves only as the conceptual limit in which the simulator reproduces the correct conditional mechanism. Robustness in SBI therefore means designing inference procedures that remain reliable even when \(\bgamma^\ast\) is unknown and only approximated by the available simulator configurations.


\subsubsection{The Asymptotic Meaning of Robustness}

If we knew the correct simulator configuration $\bgamma^\ast$,  Bayesian inference would proceed in the usual way,
%
\begin{equation}
    p(\btheta \mid \bxobs, \bgamma^\ast)
    \propto p(\bxobs \mid \btheta, \bgamma^\ast)\, p(\btheta)\;.
\end{equation}
%
In this idealised setting, the physical parameters take some true value  $\btheta^\ast$, so that the true data-generating distribution satisfies
%
\begin{equation}
    p_0(\bx) = p_0(\bx \mid \btheta^\ast)\;.
\end{equation}
%
If the simulator contains a configuration $\bgamma^\ast$ that reproduces  the correct conditional mechanism, then inference based on $p(\bx\mid\btheta,\bgamma^\ast)$ will concentrate around $\btheta^\ast$ as more data become available.

\medskip
In practice, however, the correct configuration $\bgamma^\ast$ is unknown,  and using an incorrect value $\bgamma$ alters the posterior  $p(\btheta \mid \bxobs, \bgamma)$.   To understand what “wrong’’ means in a precise sense, it is useful to  consider the asymptotic regime in which many independent observations $\bx_{1:n}$ are drawn from $p_0(\bx)$ and analysed with the  misspecified model $p(\bx \mid \btheta, \bgamma)$.   Classical results on Bayesian asymptotics under misspecification \citep[\fex][]{white_maximum_1982, kleijn_misspecification_2006} show that the posterior then concentrates around the \emph{pseudo-true parameter}
%
\begin{equation}
    \btheta^\dagger(\bgamma)
    = \arg\min_{\btheta}
    D_{\mathrm{KL}}\!\left(
        p_0(\bx) \,\middle\|\, 
        p(\bx \mid \btheta , \bgamma)
    \right)\!,
\end{equation}
that is, the value of $\btheta$ whose likelihood is closest in the  Kullback--Leibler sense to the true distribution under the assumed  configuration~$\bgamma$.   For the correct configuration one recovers  $\btheta^\dagger(\bgamma^\ast)=\btheta^\ast$, while for any other value  of $\bgamma$ the limiting parameter will in general be biased.

Moreover, in the large-$n$ limit the posterior becomes increasingly concentrated around $\theta^{\dagger}(\gamma)$, shrinking at the usual $1/\sqrt{n}$ rate (a Bernstein-von Mises-type behaviour; see \cite{belomestny_bernsteinvon_2023} for a recent review). This means that even small biases induced by choosing an incorrect configuration $\gamma$ will be amplified across many observations, leading to combined inferences that do not converge to the true value $\theta^\ast$.

\paragraph{Why asymptotics matter} Although many experiments in the physical sciences are not repeatable in the strict sense, we routinely combine information from multiple independent observations. Ensuring that each single-event posterior behaves correctly in the asymptotic sense is therefore essential for the consistency of joint or hierarchical analyses~\citep{koers_misspecified_2023}.  Robustness in SBI thus ideally aims at designing inference procedures whose posteriors remain well behaved---in particular, concentrate near  $\btheta^\ast$ with appropriate uncertainty scaling---even when the  unknown configuration $\bgamma^\ast$ is only imperfectly approximated by  the available simulator family.


\subsubsection{Three Strategies for Robust SBI}

The asymptotic considerations above suggest that robustness requires controlling how inference depends on the unknown simulator configuration~$\bgamma^\ast$. In practice, one may distinguish three very broad strategies for addressing such misspecification, which align with the three components of SBI in Fig.~\ref{fig:sbi_overview}:\footnote{In actual scientific applications, approaches are mixed, calibration observations play a role, not everything in the literature strictly maps onto the presented categories. But we find it useful to draw here in broad strokes to frame the discussion and highlight main differences.} \emph{model augmentation}, \emph{robust summary learning}, and \emph{general Bayesian updating}. Each modifies a different component of the inference pipeline---the simulation model, the compressed data representation, or the Bayesian update rule---and each tends to exhibit different asymptotic behaviour.


\paragraph{Model augmentation}

A first approach treats $\bgamma$ as a latent nuisance variable and integrates it out,
%
\begin{equation}
    p(\btheta \mid \bxobs)
    \;\longrightarrow\;
    \frac{1}{Z}\!\int p(\bxobs \mid \btheta,\bgamma)\,p(\bgamma)\,p(\btheta)\,d\bgamma\;.
\end{equation}
%
Examples include modeling systematic uncertainties (parametrized though $\bgamma$) with uninformed priors, or the use of Gaussian processes (represented by $\bgamma$) to account for unmodeled components.  This strategy is expected to be asymptotically reliable if each observation has an independent configuration $\bgamma_i \sim p(\bgamma)$, and if its prior is correctly specified, $p(\bgamma)=p_0(\bgamma)$. Otherwise, the $\bgamma$–averaged simulator may itself become misspecified and the posterior concentrates at a pseudo-true parameter $\btheta^\dagger$.  
If $\bgamma$ is instead a global calibration parameter or systematic effect shared across all observations, treating it as i.i.d.\ noise is invalid, and it should be be jointly estimated as global parameter with $\btheta$. 

\paragraph{Robust summary learning}
A second strategy constructs summaries $T(\bx)$ such that inference depends on
$p(T(\bx)\mid\btheta,\bgamma)$ rather than on $p(\bx\mid\btheta,\bgamma)$,
\begin{equation}
    p(\btheta \mid \bxobs)
    \;\longrightarrow\;
    \frac{1}{Z}\,
    p\!\big(T(\bxobs)\mid\btheta,\bgamma\big)\,p(\btheta)\,,
    \qquad\text{for any }\bgamma\;.
\end{equation}
This essentially corresponds to masking those aspects of the data for which the simulator is misspecified.  Robustness is achieved when $T$ successfully removes the dependence on $\bgamma$, i.e.\ when $p(T(\bx)\mid\btheta,\bgamma)\approx p(T(\bx)\mid\btheta)$ for all relevant $\bgamma$, including the unknown $\bgamma^\ast$.   In this ideal situation the summary likelihood becomes well specified, and the posterior  is expected to asymptotically focus on $\btheta^\ast$.  In practice, however, removing all $\bgamma$-dependence maybe be difficult to achieve.  Robust summary learning therefore introduces a bias-variance trade-off: reducing sensitivity to $\bgamma$ inevitably discards some information about~$\btheta$.  

\paragraph{General Bayes}
A third strategy encompasses everything where the Bayesian update rule itself is modified with the intend to increase robustness. In a general form, it may be formalized as replacing the likelihood by an appropriate loss function~(see \cite{bissiri_general_2016}, and 
\cite{guedj_primer_2019} for connections to PAC Bayesian learning)
%
\begin{equation}
    p(\btheta \mid \bxobs)
    \;\longrightarrow\;
    \frac{1}{Z}\,\exp\!\big(-\ell(\btheta,\bxobs)\big)\,p(\btheta)\;,
\end{equation}
%
where $\ell(\btheta, \bx)$ is chosen to reduce the impact of misspecification.  Examples include tempered likelihoods, but also (in the broad sense that we adopt here) the analysis of denoised data.  In general, however, such updates would asymptotically concentrate at the loss-risk minimiser $\btheta_\ell^\dagger=\arg\min_\theta \mathbb{E}_{p_0}[\ell(\theta,X)]$, which typically differs from both $\btheta^\ast$ and the KL pseudo-true value.   Posterior uncertainty is not automatically calibrated unless $\ell$ is correctly scaled, and only special choices yield desirable asymptotic behaviour.

\medskip
These three approaches respectively correspond to modifying the simulator, the representation, or the posterior update rule.  In the remainder of this section we focus on \emph{robust summary learning}, which may offer the most principled path to asymptotic correctness.


\subsection{Robust Summary Learning}
\label{sec:adv:uncertainty:robust}

As motivated in Sec.~\ref{sec:adv:uncertainty:defining}, we will consider a generative model that includes both the parameters of interest \(\btheta\) and the latent model configuration \(\bgamma\), with the full joint distribution
%
\begin{equation}
    p(\bx, \btheta, \bgamma) = p(\bx \mid \btheta, \bgamma) p(\btheta \mid \bgamma) p(\bgamma)\;.
    \label{eqn:joint_model}
\end{equation}
%
Since we focus on situations where \(\bgamma\) captures uncertainties in the data-generating process rather than in our prior beliefs about \(\btheta\), we subsequently assume \(p(\btheta \mid \bgamma) = p(\btheta)\).   Furthermore, we are interested in the summary-induced generative model, where all dependence on \(\bx\) enters only through the learned summary \(T(\bx)\),
%
\begin{equation}
    p(T(\bx), \btheta, \bgamma) = p(T(\bx) \mid \btheta, \bgamma) p(\btheta) p(\bgamma)\;.
    \label{eqn:joint_model}
\end{equation}
%
For this specific but very generally applicable setting, we will explore how variations in the simulator configuration, $\bgamma$, biases inferential tasks, using concepts of information theory.  This will provide the necessary basis for defining optimization strategies and criteria for robust inference.


\subsubsection{Three Faces of Configuration Bias}
\label{sec:adv:uncertainty:robust:three}

Above, we assumed that the model parameters \(\btheta\) and simulator configurations \(\bgamma\) are statistically independent. (Even if they were not, modifying the data summaries \(T(\bx)\) would not alter that dependence.) This leaves three channels through which the simulator configuration can enter the inference procedure: the posterior, the data summary, and the likelihood.


\paragraph{Posterior bias: parameter estimates that depend on $\bgamma$}

When model configurations \(\bgamma\) vary, inference based on a given summary statistics \(T_\phi(\bx)\) can become biased or unstable.  Our goal is therefore to learn a summary mapping \(T_\phi(\bx)\) such that the resulting posterior becomes (approximately) independent of the simulator configuration. That is, we seek summaries for which
\[
p(\btheta \mid T_\phi(\bx), \bgamma) \simeq p(\btheta \mid T_\phi(\bx)) \quad \text{for all plausible } \bgamma\;.
\]
%
This expresses the requirement that, once the data are summarized by \(T_\phi(\bx)\), changing the simulator configuration \(\bgamma\) no longer affects inference over \(\btheta\).\footnote{Note that here $p(\btheta \mid T_\phi(\bx))$ is technically marginalized over $\bgamma$, but its main role is to simply provide a constant reference point. The specifically adopted prior is not relevant.} 

A natural way to quantify deviations from this ideal is via the mutual information between \(\btheta\) and \(\bgamma\), conditioned on \(T_\phi(\bx)\):
\begin{equation}
\mathbb{E}_{p(\bx \mid \bgamma)p(\bgamma)}
\left[D_{\mathrm{KL}}\left(p(\btheta \mid T_\phi(\bx), \bgamma) \mid\mid p(\btheta \mid T_\phi(\bx))\right)\right] 
\equiv
\mathcal{I}(\btheta; \bgamma \mid T_\phi(\bx)) 
\leq \varepsilon\;.
\label{eqn:posterior_bias}
\end{equation}
%
This quantity measures how much knowing \(\bgamma\) improves inference about \(\btheta\), given the summary. We use this mutual information as a proxy for posterior bias throughout. When \(\varepsilon = 0\), the posterior is completely robust to simulator configurations, independent of the prior over \(\bgamma\) (as long as it has sufficient support). 

\smallskip

Small values of \(\varepsilon\) indicate that inference is stable across different simulator configurations and approximates the result we would obtain under the correct (but unknown) \(\bgamma^\ast\). At the same time, this invariance must be balanced against the requirement that \(T_\phi(\bx)\) remains informative about \(\btheta\); overly aggressive bias suppression may lead to summaries that are too coarse for precise inference.

\paragraph{Summary bias: representation drift across $\bgamma$}

Even if posterior inference is stable across \(\bgamma\), the learned summary \(T_\phi(\bx)\) itself may retain residual dependence on the simulator configuration. This can occur when \(T_\phi(\bx)\) encodes aspects of the data that are irrelevant for inference over \(\btheta\) but still vary with \(\bgamma\). Such dependence is typically undesirable, as it makes the influence of \(\bgamma\) on the representation---and therefore on the overall inference pipeline---harder to diagnose and control.

It is therefore natural to require that the distribution of \(T_\phi(\bx)\) be invariant to changes of the simulator configuration \(\bgamma\), i.e.,
\[
p(T_\phi(\bx) \mid \bgamma) \simeq p(T_\phi(\bx)) \quad \text{for all plausible } \bgamma.
\]
This condition ensures that \(T_\phi(\bx)\) and \(\bgamma\) are statistically independent, meaning that observing the summary provides little or no information about the underlying simulator configuration. 

As before, this requirement can be formalized using mutual information:
\begin{equation}
\mathbb{E}_{p(\bgamma)}\left[D_{\mathrm{KL}}\left(p(T_\phi(\bx) \mid \bgamma) \,\|\, p(T_\phi(\bx))\right)\right]
\equiv
\mathcal{I}(T_\phi(\bx); \bgamma)  
\leq \varepsilon.
\label{eqn:summary_bias}
\end{equation}
%
In the limit \(\varepsilon \to 0\), this enforces strict simulator invariance of the summary, independent of the specific prior over \(\bgamma\). Note that while this promotes interpretability, it does \emph{not} by itself guarantee robustness of the posterior as in Eq.~\eqref{eqn:posterior_bias}: \(T_\phi(\bx)\) and \(\bgamma\) may still be dependent once \(\btheta\) is fixed.
\footnote{Marginal independence does not imply conditional independence. 
Let $\theta,\gamma\in\{0,1\}$ be independent and uniform. Define  
$T = \gamma$ when $\theta=0$ and $T = 1-\gamma$ when $\theta=1$.  
Then $T$ and $\gamma$ are deterministically dependent given $\theta$,
but marginally $p(T,\gamma)=p(T)p(\gamma)=\frac14$ for all $T$ and $\gamma$.}


\paragraph{Likelihood bias: $\bgamma$-dependence at fixed parameters}

The most direct impact on the inference process comes through the likelihood function.   Following the same logic as above, we can write the condition for a summary likelihood that is independent of the simulator configuration $\bgamma$,
\[
p(T_\phi(\bx) \mid \btheta, \bgamma) \simeq p(T_\phi(\bx) \mid \btheta) \quad \text{for all plausible } \bgamma\;.
\]

Again, this condition can be formalised and written in terms of mutual information  between the summary \(T_\phi(\bx)\) and the simulator configuration \(\bgamma\), conditioned on the parameter \(\btheta\):
\begin{equation}
\mathbb{E}_{p(\bx, \btheta, \bgamma)} 
\left[ 
\log \frac{p(T(\bx), \bgamma \mid \btheta)}{p(T(\bx) \mid \btheta)\, p(\bgamma \mid \btheta)} 
\right]
 \equiv \mathcal{I}(T(\bx); \bgamma \mid \btheta) 
    \leq \varepsilon \,.
    \label{eqn:likelihood_bias}
\end{equation}
%
This quantity captures how much additional information about the summary \(T(\bx)\) is provided by \(\bgamma\), beyond what is already explained by \(\btheta\). It therefore quantifies the stability of the summary likelihood function across simulator configurations.


\subsubsection{The Simulator Bias Identity}

\begin{figure}[t]
    \includegraphics[width=\linewidth]{figures/Venn2.pdf}
    \caption{Different contributions to inference bias visualised.}
    \label{fig:venn_simulator_bias_identity}
\end{figure}

All of the mutual-information–based bias measures introduced above can be unified in a single quantity that captures the total dependence of the inference process on the simulator configuration \(\bgamma\). This is the mutual information between the summarized simulator outputs \((\btheta, T_\phi(\bx))\) and the configuration \(\bgamma\), given by
%
\begin{equation}
\mathcal{I}(\btheta, T_\phi(\bx); \bgamma)
\equiv
\int d\btheta \, d\bx \, d\bgamma\;
p(\btheta, \bx, \bgamma) \log \left(
\frac{p(\btheta, T_\phi(\bx) \mid \bgamma)}{p(\btheta, T_\phi(\bx))} 
\right) \;,
\label{eqn:total_bias}
\end{equation}
%
and it quantifies the overall bias induced by varying simulator configurations. If this quantity vanishes, then the joint distribution of \(\btheta\) and \(T_\phi(\bx)\) is completely independent of \(\bgamma\), implying fully robust inference across simulator variants.


\paragraph{Simulator bias decomposition identity}  The various bias terms introduced above are linked by a simple and illuminating relation,
%
\begin{equation}
    \underbrace{\mathcal{I}(T_\phi(\bx); \gamma)}_{\text{Summary bias}}
    +
    \underbrace{\mathcal{I}(\btheta; \gamma \mid T_\phi(\bx))}_{\text{Posterior bias}}
    \;=\;
    \underbrace{\mathcal{I}(\btheta, T_\phi(\bx); \gamma)}_{\text{Simulator bias}}
    \;=\;
    \underbrace{\mathcal{I}(\btheta; \gamma)}_{\text{Prior bias}}
    +
    \underbrace{\mathcal{I}(\gamma ; T_\phi(\bx) \mid \btheta)}_{\text{Likelihood bias}}
    \label{eqn:bias_identity}
\end{equation}
which follows directly from the chain rule of mutual information. For a fixed summary map \(T_\phi(\bx)\), each term quantifies a different way in which the simulator configuration \(\bgamma\) can influence the inference problem. The identity therefore shows how summary, posterior, prior, and likelihood biases decompose the \emph{simulator bias}—the total dependence of the inference pipeline on \(\bgamma\).

A key consequence is that an upper bound on the likelihood bias automatically bounds both the summary and posterior bias. Since we assume that \(\btheta\) and \(\bgamma\) are independent a priori, the prior bias term vanishes. Thus, enforcing a sufficiently small likelihood bias is enough to suppress all remaining forms of configuration-induced bias.

\cw{TODO: Explain figure and add caption} This is shown in Fig.~\ref{fig:venn_simulator_bias_identity}.


\subsubsection{Constrained Optimisation for Robust Minimal Summaries}

As discussed in Sec.~\ref{sec:adv:uncertainty:robust}, robustness requires data summaries  $T(\bx)$ that remove sensitivity to simulator configurations~$\bgamma$ while  retaining predictive information about~$\btheta$.   Two complementary information-theoretic principles govern this trade-off:  (i) \emph{invariance} to different simulator configurations, and  (ii) controlled \emph{compression} to suppress spurious or unstable features.  We now formalize both ideas.

The simulator bias identity, Eq.~\eqref{eqn:bias_identity}, shows that robustness to simulator configurations~$\bgamma$ is achieved most directly by suppressing the posterior and summary bias, or equivalently the likelihood bias $\mathcal{I}(T(\bx);\bgamma\mid\btheta)$.  Minimizing this term with respect to the summary $T_\phi(\bx)$ alone, however, admits the trivial solution of a summary that is uninformative about $\btheta$. To avoid such collapse, one must simultaneously \emph{retain} parameter-relevant information, as given by $\mathcal{I}(T_\phi(\bx); \btheta)$, while \emph{removing} configuration-relevant information.  

\paragraph{Lagrangian objective for robust minimal summaries}

It is standard to express the trade-off discussed above through a Lagrangian objective. With the goal in mind that $T_\phi(\bx)$ 
should be \emph{informative}, \emph{robust} and \emph{minimal}, we can write as optimisation objective:
\begin{equation}
\label{eqn:IIB_objective}
\max_{T_\phi}\;
\underbrace{\mathcal{I}(T_\phi(\bx);\btheta)}_{
\substack{\text{Parameter}\\ \text{information}}}
-\lambda
\underbrace{\mathcal{I}(\btheta; \bgamma \mid T_\phi(\bx))}_{
\substack{\text{Posterior bias}}}
- \eta
\underbrace{\mathcal{I}(T_\phi(\bx);\bgamma)}_{
\substack{\text{Summary bias}}}
- \beta
\underbrace{\mathcal{I}(T_\phi(\bx);\bx)}_{
\substack{\text{Information}\\ \text{bottleneck}}}
\;.
\label{eqn:robust_minimal_summary_objective}
\end{equation}
%
Here, $\lambda, \eta, \beta \geq0$ are tunable hyperparameters that control the trade-off balance. We will discuss each of the four components of the optimisation objective in Eq.~\eqref{eqn:robust_minimal_summary_objective} separately.
\begin{itemize}
    \item \textbf{Parameter information}: Maximizing the first term in Eq.~\eqref{eqn:robust_minimal_summary_objective} is necessary to retain information about the parameters of interest, $\btheta$. This is often a automatic consequence of end-to-end learning of data summaries, see \fex\ the NPE loss in Eq.~\eqref{eqn:NPE_MI}.
    
    \item \textbf{Posterior bias}: As discussed in Sec.~\ref{sec:adv:uncertainty:robust:three}, this term controls how much the posterior $p(\btheta \mid T_\phi(\bx), \bgamma)$ depends on the simulator configuration $\bgamma$.  It appears in the literature in context of invariant risk minimization  \citep{arjovsky_invariant_2020} and invariant information bottleneck \citep{li_invariant_2022}.
    
    \item \textbf{Summary bias}: As discussed in Sec.~\ref{sec:adv:uncertainty:robust:three}, this term controls how much the data summary distribution, $p(T_\phi(\bx) \mid \bgamma)$, varies with $\bgamma$.  It appears in the literature in context of fair and invariant representation learning,  \citep{zemel_learning_2013, zhao_fundamental_2020}.  If $\lambda = \eta$, the term and the previous one can be replaced equivalently by the likelihood bias, but we kept the contributions here separate for clarity.
    
    \item \textbf{Information bottleneck}: This term penalises how much information the summary $T_\phi(\bx)$ retains of the full data $\bx$.  It suppresses unnecessary information in the summary $T_\phi(\bx)$, promoting a \emph{minimal} summary. In the literature it appears in context of information bottleneck ideas \citep{tishby_information_2000}.
\end{itemize}

The introduction of the information bottleneck term does not only promote minimalism, but can also enhance robustness. Simulator configuration invariance alone does not guarantee robustness of inference to \emph{unmodeled forms of mismodeling}, i.e.\ deviations not captured by~$\bgamma$. In particular high-dimensional complex summaries may still encode aspects of $\bx$ that are sensitive to unmodeled imperfections of the simulator.  The information bottleneck term can help to further reduce such pathways for mismodeling to propagate, into the inference results. 

\cw{MAYBE: Individual components visualised in Fig.~\ref{fig:Venn2}}

\cw{MAYBE: Technical implementation }

%\subsubsection{Technical Implementation Challenges}

The conditional mutual information in Eq.~\eqref{eqn:IIB_objective} can be, \fex\ estimated through an approximate variational upper bound based on the CLUB estimator~\citep{cheng_club_2020},
%
\begin{equation}
\mathcal{I}(T_\phi(\bx); \bgamma \mid \btheta) \lesssim
\mathbb{E}_{p(\bx, \btheta, \bgamma)} 
\left[ \log q_\psi(T_\phi(\bx) \mid \bgamma, \btheta) \right]
- 
\mathbb{E}_{p(\btheta, \bgamma)p(\bx \mid \btheta)}
\left[ \log q_\psi(T_\phi(\bx) \mid \bgamma, \btheta) \right] \;
\label{eqn:likelihood_bias_bound}
\end{equation}
Here, we introduced an auxiliary network $q_\psi(T_\phi(\bx) \mid \gamma, \theta)$ that approximates the conditional distribution of summaries.  Alternatively to CLUB, for instance adversarial estimates of the conditional mutual information are possible~\citep{xxx}.

For deterministic and continuous summaries, the term 
$\mathcal{I}(T_\phi(\bx);\bx)$ is actually formally ill-defined.\footnote{Under common 
regularization schemes, 
$\mathcal{I}(T_\phi(\bx);\bx)$ reduces to a differential entropy term 
$\mathcal{H}(T_\phi(\bx))$ plus a regularization-dependent constant.  
However, differential entropy is scale-dependent, and the penalty can always be reduced  simply by shrinking the scale of $T_\phi(\bx)$, rendering it unsuitable as a direct  optimization target.} 
A standard remedy—used, for example, in the variational information bottleneck (VIB)~\citep{xxx}—is  to inject a small, fixed amount of noise into the representation, which fixes the scale  and renders the mutual information well-defined. In practice on often uses structured proxies for compression, such as dimensionality reduction, sparsity  constraints, or gating mechanisms~\citep{xxx}.   These serve as practical surrogates for the idealized SIB objective.   

\paragraph{Variational Bound for Invariant Information Bottleneck}

We illustrate here, for a simple example, how the IIB and SIB objectives, Eqs.~\eqref{eqn:IIB_objective} and~\eqref{eqn:SIB_objective}, can be implemented in practice. Our aim is to demonstrate how type~A epistemic uncertainty---stemming from mismatches between the simulator and the true data-generating process---can be transformed into type~B uncertainty, arising from information loss. This trade-off enables more robust inference by reducing the influence of unreliable or ambiguous aspects of the simulator.

\paragraph{Variational Bounds for Mutual Information and Entropy}

In order estimate the conditional mutual information in Eq.~\eqref{eqn:IIB_objective} in a tractable way, we use an approximate variational upper bound inspired by the CLUB estimator~\citep{cheng_club_2020}
%
\begin{equation}
\mathcal{I}(T_\phi(\bx); \bgamma \mid \btheta) \lesssim
\mathbb{E}_{p(\bx, \btheta, \bgamma)} 
\left[ \log q_\phi(T_\phi(\bx) \mid \bgamma, \btheta) \right]
- 
\mathbb{E}_{p(\btheta, \bgamma)p(\bx \mid \btheta)}
\left[ \log q_\phi(T_\phi(\bx) \mid \bgamma, \btheta) \right] \;.
\label{eqn:likelihood_bias_bound}
\end{equation}
%
In practice, we can then minimise the upper bound as surrogate for minimising the mutual information directly. The second term is here estimated using mismatched pairs: summaries generated from samples \( \bx \sim p(\bx \mid \btheta) \), but evaluated under different configurations \( \bgamma \).  Note that the variational upper bound is only approximate (it becomes strict for exact $q_\phi$), but it features a useful structural property: the gradient with respect to \( T_\phi(\bx) \) vanishes when \( q_\phi \) becomes independent of \(\bgamma\). This independence is, in fact, the desired outcome.

In a similar way we obtain the upper bound
%
\begin{equation}
\mathcal{H}(T_\phi(\bx)) 
\equiv - \mathbb{E}_{p(\bx)} \left[ \log p(T_\phi(\bx)) \right]
\;\leq\;
- \mathbb{E}_{p(\bx, \bgamma, \btheta)} 
\left[ \log q_\phi(T_\phi(\bx) \mid \bgamma, \btheta) \right] \;,
\label{eqn:entropy_bound}
\end{equation}
%
which is exact in the limit where $q_\phi$ does not depend on $\bgamma$ and $\btheta$.
\cw{Maybe extend this with more relevant examples}


\paragraph{A Variational Training Objective for Robust Summaries}

We now define a training objective that integrates four components: (1) an amortized posterior term that encourages informativeness with respect to \(\btheta\), and that is conditioned on model configurations $\bgamma$, (2) an auxiliary density estimator for the summary conditioned on \((\bgamma, \btheta)\), (3) a robustness regularizer via the variational upper bound on \(\mathcal{I}(T_\phi(\bx); \bgamma \mid \btheta)\), and (4) a sparsity regularizer based on the entropy bound from Eq.~\eqref{eqn:entropy_bound}.

This leads to the following composite loss:
%
\begin{multline}
    \mathcal{L}[\phi_T, \phi_\theta, \phi_q] =
    \\[0.5em]
    -\underbrace{
    \mathbb{E}_{p(\bx, \btheta, \bgamma)}\left[\log 
    q_{\phi_\theta}(\btheta \mid T_{\phi_T}(\bx), \bgamma)\right]
    }_{\text{parameter inference \& informative summary}}
    -\underbrace{
    \mathbb{E}_{p(\bx, \btheta, \bgamma)} \left[\log q_{\phi_q}(
    T_{\bar{\phi}_T}(\bx) \mid \btheta, \bgamma)\right]
    }_{\text{auxiliary summary modeling}}
    \\[0.5em]
    + \lambda \cdot \underbrace{
    \left(
    \mathbb{E}_{p(\btheta, \bgamma)\,p(\bx \mid \btheta, \bgamma)} 
    \left[ \log q_{\bar{\phi}_q}(T_{\phi_T}(\bx) \mid \bgamma, \btheta) \right]
    -
    \mathbb{E}_{p(\btheta, \bgamma)\,p(\bx \mid \btheta)}
    \left[ \log q_{\bar{\phi}_q}(T_{\phi_T}(\bx) \mid \bgamma, \btheta) \right]
    \right)
    }_{\text{configuration invariance regularizer}}
    \\[0.5em]
    - \beta \cdot \underbrace{
    \mathbb{E}_{p(\bx, \btheta, \bgamma)} \left[ \log q_{\bar{\phi}_q}(
    T_{\phi_T}(\bx) \mid \bgamma, \btheta) \right]
    }_{\text{sparsity regularizer}}
\end{multline}
%
Here, \(\phi_T\) parametrizes the summary network \(T_{\phi_T}(\bx)\), \(\phi_\theta\) the posterior estimator, and \(\phi_q\) the auxiliary density model used to evaluate the entropy and mutual information bounds. When computing gradients, expressions involving barred parameters (e.g., \(\bar{\phi}_T\), \(\bar{\phi}_q\)) are treated as constants and do not contribute to backpropagation. This is typically implemented via explicit gradient detachment.

The hyperparameters $\lambda$ and $\beta$ control the relative influence of the robustness and sparsity regularizers, respectively, and therefore determine the balance between invariance and compression. When $\lambda>0$ and $\beta=0$, the objective prioritizes simulator invariance by suppressing configuration-dependent variation without restricting the overall capacity of the summary. Conversely, choosing $\lambda=0$ and $\beta>0$ enforces compression alone, limiting the information content of the summary and reducing sensitivity to unmodeled sources of mismodeling. When both coefficients are positive, with $\lambda>\beta>0$, the objective achieves a controlled compromise, encouraging representations that are simultaneously robust to simulator variation and sufficiently compressed to avoid encoding extraneous or unstable features.



%\section{Finite-data effects and variance analysis}

%\cw{TODOREF - Finite sample size}

\subsection{Hessian-based uncertainty analysis}

The convergence arguments in Sec.~\ref{sec:methods} relied on the large training data limit, $N \to \infty$, where sums over training data samples can are replaced by integrals over generative model parameters and observations (see for instance Eqs.~\eqref{eqn:NPE_loss} and~\eqref{eqn:NPE_limit}).  Furthermore, we assumed that networks are expressive enough to mathematically represent the minimum of the loss function. However, in many applications, one only has a limited amount of training data available, depending on simulation and storage costs.  Unfortunately, predicting the performance of models and algorithms in situations with limited training data and network capacity is generally challenging. Still, some basic insights can be obtained mathematically by considering the sample variance of the loss function.

\paragraph{General analysis.}
SBI training losses functions, like Eqs.~\eqref{eqn:NPE_loss} or \eqref{eqn:NRE_loss}\footnote{With an additional sum over $|\mathcal{D}_c|$.}, can be generally written in the form
%
\begin{equation}
\label{eqn:general_training_loss}
\mathcal{L}_{\mathcal D}[q] = \frac1{N} \sum_{\bx, \btheta \in \mathcal{D}} \ell_q(\bx, \btheta)\;.
\end{equation}
%
We made here explicit that the loss function depends, besides on the network weights $\phi$, also on the specific training data realization $\mathcal{D}$.\footnote{Note that we sum here over \emph{all} available training data, $\mathcal{D}$. The usual sub-sampling or mini-batching during training with stochastic gradient descent introduces additional noise. We are here, however, only interested in effects on the loss function after sub-sampling related variations are averaged out. 
This corresponds to the end-phase of training with small learning rate.}
%As discussed in Sec.~\ref{sec:methods}, 
Training data is generated as $N$ i.i.d.~samples from the generative model.
It is then straightforward to show that the mean of the loss function is given by 
$\bbE_{\mathcal D}[\mathcal L[\mathcal D, \bphi]] = \bbE_{p(\bx, \btheta)}[\ell_{\bphi}(\bx, \btheta)]$,
and that its variance is given by
$\text{Var}_{\mathcal D}[\mathcal L[\mathcal D, \bphi]] = \frac{1}{N}\text{Var}_{p(\bx, \btheta)}[\ell_{\bphi}(\bx, \btheta)]$. The variance scales like $1/N$ with training data size, as expected from the variance reduction properties of averaging.

To study the impact of the variance on the loss minimum, we expand the loss function at second order in the deviations $\bepsilon$, around its large-sample minimizer $\bphi$.
%
$$
q_{\balpha}(\btheta \mid \bx) = q_0(\btheta \mid \bx)
\cdot \exp\left(1+\sum_{i=1}^K \alpha_i \delta_i(\btheta, \bx) \right)
$$
We can then investivate the loss function value as function of $\balpha$
$$
\mathcal{L}_{\mathcal D}[\balpha] \equiv
\mathcal{L}_{\mathcal D}\left[q_{\balpha}(\btheta \mid \bx)\right]
$$
Second order Talor expansion leads to 
$$
\mathcal{L}_{\mathcal D}[\balpha]
\approx
\mathcal{L}_{\mathcal D}[0]
+ \balpha^T  
\underbrace{\left.\nabla_{\balpha} \mathcal{L}_{\mathcal{D}} \right|_{\balpha = 0}}_{\equiv \mathbf g}
+ \frac12 
\balpha^T  
\underbrace{\left.(\nabla^2_{\balpha} \mathcal{L}_{\mathcal D})\right|_{\balpha = 0}}_{\equiv \mathbf H}
\balpha
$$
where we introduce gradient and Hessian.

We can trivially minimize that function, for a given dataset $\mathcal{D}$, and find
$$
\balpha^\ast \equiv  \argmin_{\balpha} \mathcal{L}_{\mathcal D}[\balpha] =  - {\mathbf H}^{-1} \mathbf g
$$

To lowest order, we can then estimate the dataset related variance of the minimiser to be
$$
\text{Cov}_{\mathcal D}[\balpha^\ast] \approx
\bbE_{\mathcal D}[{\mathbf H}]^{-1}\cdot
\text{Cov}_\mathcal{D}[\mathbf g]
\cdot \bbE_\mathcal{D}[{\mathbf H}]^{-1}
$$

Any set of functions $\delta_i(\btheta, \bx)$ spans a linear space. The standard normal modes are defined to span the same space, but with a unit covariance matrix for $\text{Cov}_{\mathcal D}[\balpha^\ast] = \mathbb 1$.
Such normal modes can be called $\delta_i$.


\paragraph{Neural posterior estimation.} 
For NPE, we find
$$
\text{Cov}[\balpha] = \frac1N\int d\btheta\, d\bx\, p(\bx, \btheta)
\delta_i(\btheta, \bx)
\delta_j(\btheta, \bx)
%\frac{q_i(\btheta \mid \bx)}{q_0(\btheta \mid \bx)}
%\frac{q_j(\btheta \mid \bx)}{q_0(\btheta \mid \bx)}
$$

In order to proceed, we make a simple ansatz that the modes have the form. For any set of basis functions $e_i(\btheta, \bx)$, we can find that $q_i$ lead to a unit variate 
$$
\delta_i(\btheta , \bx) = 
\frac{e_i(\btheta, \bx)}{\sqrt{p(\bx, \btheta)}}
$$
for which the $\balpha^\ast$ covariance becomes the unit matrix.

We define state density
$$
\rho(\bx, \btheta) \equiv \sum_i e_i^2(\bx, \btheta)
= \frac{1}{\mathcal V_{\rm eff}(\bx, \btheta)}
$$
Then the deviation
$$
\delta q(\btheta \mid \bx) = \sum_{i=1}^K \alpha_i \delta_i(\btheta, \bx) q_0(\btheta \mid \bx)
$$
has a point-wise variance that we can estimate as
$$
\text{Cov}_\mathcal{D}
\left[\frac{\delta q(\btheta \mid \bx)}{q_0(\btheta \mid \bx)}\right]
=
\frac{\rho(\bx, \btheta)} {p(\bx, \btheta) N}
=
\frac{1} {p(\bx, \btheta) N \mathcal V_{eff}(\btheta, \bx)}
$$

Few relevant observations...

\paragraph{Neural ratio estimation.}
For NRE we find instead

$$
\text{Cov}_\mathcal{D}k\left[\frac{\delta q(\btheta \mid \bx)}{q_0(\btheta \mid \bx)}\right]
=
\frac{\rho(\bx, \btheta)} {\min(p(\bx, \btheta), p(\bx) p(\btheta)) N}
$$


\subsection{Noise resampling}



%For $M\to \infty$, second term vanishes. Only variations where 
%$\delta q(\btheta \mid \bx) \approx \delta h(\btheta) p(\btheta \mid \bx)$ is not varying much with $\bx$ matter.

%Each term in Eq.~\eqref{eqn:general_training_loss} has hence the variance $\text{Var}_{p(\bx, \btheta)}[\ell_\phi(\bx, \btheta)]$. 
%Given that training examples are drawn independently, standard variance scaling arguments yield then that the total variance of the training loss,
%
%\begin{equation}
%\text{Var}_{p(\mathcal{D})} [\mathcal{L}[\mathcal D, \phi]]
% = \frac1{|\mathcal D|} \text{Var}_{p(\bx, \btheta)} [\mathcal{\ell}_\phi(\bx, \btheta)]\;.
%\label{eqn:loss_var}
%\end{equation}
%\begin{equation}
%\text{Var}_{p(\mathcal{D})} [\nabla_\phi\mathcal{L}[\mathcal D, \phi]]
% = \frac1{|\mathcal D|} \text{Var}_{p(\bx, \btheta)} [\nabla_\phi\mathcal{\ell}_\phi(\bx, \btheta)]
%\label{eqn:loss_var}
%\end{equation}
%
%
%\begin{equation}
%\text{Var}_{p(q)p(\mathcal{D})} [\mathcal{L}[\mathcal D, \phi]]
%= 
%\bbE_{p(q)}\text{Var}_{p(\mathcal{D})} [\mathcal{L}[\mathcal D, \phi]]
%+ \text{Var}_{p(q)}\bbE_{p(\mathcal{D})} [\mathcal{L}[\mathcal D, \phi]]
%\end{equation}
%%
%As expected, the sample variance related to finite training data vanishes in the large sample limit, $|\mathcal{D}| \to \infty$.
%%
%In the subsequent discussion, and in the interest of clarity, we will consider the NPE loss, where $\ell(\bx, \btheta) = -\log q_\phi(\btheta \mid \bx)$. 

%However, analogous arguments can be made for all other SBI algorithms that we discussed in Sec.~\ref{sec:core}.

%For a single training sample, $N=1$, randomly drawn from $p(\bx, \btheta)$, the variance would be $\text{Var}_{p(\bx, \btheta)}[-\log q_\phi(\btheta \mid  \bx)]$ (it is $1/2$ in the Gaussian case).  For the $N$ i.i.d.~samples that contribute to $\mathcal{L}$, we instead obtain
%%
%$$
%\text{Var}_{p(\bx, \btheta)}[\mathcal{L}_\text{NPE}] = 
%\frac1N
%\text{Var}_{p(\bx, \btheta)}[
%%\mathcal{\ell}(\bx, \btheta)
%-\log q_\phi(\btheta \mid \bx)
%]
%\overset{N\to\infty}{\to}0
%\;.
%$$
%As expected, the sample variance of $\mathcal{L}_\text{NPE}$, and derived quantities like gradients for stochastic gradient descent, vanishes in the large $N$ limit.

Simulation models often have a modular or hierarchical form, with some computationally slow and some fast components. In these situations, SBI performance can be benefit from running the fast parts of the simulator more often than the slow parts when generating training data.

As an example, let us consider a simple hierarchical simulation model,
%
$$
p(\mathbf x \mid \btheta)
= \int d\blambda\; 
\underbrace{p(\mathbf x \mid \blambda)}_{\text{(a) Fast}}
\underbrace{p(\blambda \mid \btheta)}_\text{(b) Slow }\;.
$$
%
Here, $\btheta$ represents model parameters, $\blambda$ are (stochastic or deterministic) model predictions, and $\bx$ is simulated data.  Generating model predictions $\blambda \sim p(\blambda \mid \btheta)$ given model parameters $\btheta$ can be computationally very costly and slow, because it might involve running a physics simulation code. On the other hand, generating simulation data $\bx \sim p(\bx \mid \blambda)$ for a given model prediction $\blambda$ can be very fast, because it might just amounts to adding measurement noise that can be quickly sampled, $\bx = \blambda + \textbf n$.
It makes then sense to store $\blambda$ and re-sample $\bx$ on the fly during training.

%where some components are fast to calculate and sample (for instance, adding Gaussian measurement noise to a model prediction), and other components are slow (for instance, running a physics simulation code to obtain said model prediction). 

The empirical distribution of the finite training data has an atomic structure and is defined as a sum over delta functions,
$$
p_N(\bx, \btheta) = \frac1N\sum_{i=1}^N \delta(\bx - \bx_i) \delta(\btheta - \btheta_i)\;.
$$
The resampling distribution, on the other hand, has clouds of $\bx_{i, j} \sim p(\bx \mid \btheta_i)$ samples associated to each sample $\btheta_i \sim p(\btheta)$,
%$$
%p_r(\bx, \btheta) = \frac1N\sum_{i=1}^N 
%p(\bx \mid \blambda_i, \btheta_i) \delta(\btheta - \btheta_i)\;,
%$$
$$
p_r(\bx, \btheta) =
\frac1N\sum_{i=1}^N 
\left(\frac1M\sum_{j=1}^M
\delta(\bx - \bx_{i,j})\right)
\delta(\btheta - \btheta_i)
\;,
$$
where $N$ is the number of parameter samples, and $M$ the number of data samples per parameter sample.
This brings the empirical training data distribution much closer to the true generative model distribution $p(\bx, \btheta)$.
%which brings it closer to the true distribution $p(\bx, \btheta)$.
Examples for this are shown in Fig.~\ref{fig:resampling_examples}.

\medskip

One can use the law of total variance to show that for the above empirical resampling distribution, the variance of the gradient can be computed as
$$
\text{Cov}_{\mathcal D}[\mathbf g[\mathcal D]]
= 
\frac1N \left[
\text{Cov}_{p(\btheta)}
\bbE_{p(\bx \mid \btheta)}
\left[\nabla_{\bphi}\ell_\phi(\bx, \btheta)\right]
+\frac1M 
\bbE_{p(\btheta)}
\text{Cov}_{p(\bx \mid \btheta)}\left[\nabla_{\bphi}\ell_\phi(\bx, \btheta)\right]
\right]\;.
$$
In the case without resampling, $M=1$, we obtain the results quoted above, while in the limit of online resampling, $M\to \infty$, only the first term contributes to the gradient variance.

The following expression follows from a second-order perturbative expansion around the loss minimum, combined with a variance decomposition across hierarchical data structure.
$$
\text{Cov}_\mathcal{D}
\left[\frac{\delta q(\btheta \mid \bx)}{q_0(\btheta \mid \bx)}\right]
\approx
\frac1N \left(
\frac{\int d\bx \rho(\bx, \btheta)} {p(\btheta)}
+ \frac1M\frac{\rho(\bx, \btheta)} {p(\btheta , \bx)}
\right)
$$


In the case of NPE, the first term depends on the data average $\bbE_{p(\bx \mid \btheta)}[\delta q(\btheta \mid \bx)/p(\btheta \mid \bx)]$ w.r.t.~$\btheta$. 
%Fluctuations with a ratio $\delta q(\btheta \mid \bx)/p(\btheta \mid \bx)$ is approximately constant w.r.t.~$\bx$ are the most relevant, while the $\bx$ dependence of $\delta q(\btheta \mid \bx)$ is not affected by 
As a consequence, in the online resampling limit, $M\to \infty$,
uncertainties in the $\bx$ direction of $q_\phi(\btheta \mid \bx)$ expected to be heavily suppressed, and the fitted posterior is expected to be dominated by uncertainties of the form $q_\phi(\btheta \mid \bx) \approx (1+\delta(\btheta)) p(\btheta \mid \bx)$. Even for small relatively small $N$, this strongly constraints the optimisation problem.

%\medskip
%
%We can estimate the finite sample size-induced variance of the NPE loss function, $\mathcal{L}_\text{NPE} \equiv \mathbb E_{p(\bx, \btheta)}\left[-\log q_\phi(\btheta \mid \bx)\right]$, assuming that we have $N$ samples from $\blambda$ and $\btheta$, and $M$ samples of $\bx$ for each $\blambda$ and $\btheta$ pairs,
%%
%$$
%\text{Var}\left[\mathcal{L}_\text{NPE}\right]
%=
%\frac1N
%\left[
%\frac1M
%\mathbb E_{p(\btheta)}
%\text{Var}_{p(\bx \mid \btheta)}[-\log q_\phi(\btheta \mid \bx)]
%+
%\text{Var}_{p(\btheta)}
%\mathbb E_{p(\bx \mid \btheta)}
%[-\log q_\phi(\btheta \mid \bx)]
%\right]
%$$
%
%The first term corresponds to the variance induced by varying $\bx$ for a given $\btheta$, while the second term corresponds to the variance induced by different values of the loss function for across different $\btheta$.

\cw{FIGTODO Optional figures: Show uncertainties in training in the tails, and qualitatively compare with estimates}


%\section{Sequential methods and adaptive learning}
\label{sec:sequential_sbi}

\begin{quotation}
\textit{``When solving a problem of interest, do not solve a more general problem as an intermediate step. Try to get the answer that you really need but not a more general one.''}

\hfill --- \cite{vapnik_estimation_2006}
\end{quotation}

%\cw{TODOREF - Sequential SBI}

\subsection{Basic idea}

One characteristics of simulation-based inference methods is that they typically lead to \textit{amortization} of inference results: rather than providing the posterior approximation for a specific observation $\bxobs$ of interest, they provide an inference machine that learned how to map \textit{any} data on inference results.  In some cases, where one is just interested in a cost-efficient analysis of a single observation, that might be not desirable.  In fact, obtaining precision results in these settings can require immense amounts of training data and highly flexible network architectures.

The idea is sequential inference is to focus (sequentially) on the range of parameters $\bz$ that are relevant for a specific observation $\bxobs$. This is done by replacing during the generation of training data the actual prior distribution $p(\bz)$ with a proposal distribution $\tilde p(\bz)$ that is somehow more emphasizing the interesting parameter for a given observation.  A wide range of strategies for selecting appropriate proposal distributions, and for undoing the damage done by using the `wrong' distribution for the prior, have been proposed in the literature.  Essentially, for any of the algorithms discussed above, there is a sequential version.

We sample targeted training data
%
$$
\mathcal{D} = \{(\boldsymbol{\theta}_i, \mathbf{x}_i)\}_{i=1}^N
\;,\quad
\btheta, \bx \iidsim p(\bx \mid \btheta) \tilde p(\btheta)
$$
%
$$
\text{proposal} \quad \tilde p(\btheta) \quad 
\text{approximates target posterior}
\quad
\left. p(\btheta \mid \bx) \right|_{\bx = \bx_o}
$$
%
When learning the posterior from this training data, we find
$$
p(\btheta \mid \bx) \simeq 
\frac1{Z(\bx)}
q_\phi(\btheta \mid \bx)
\frac{p(\btheta)}{\tilde p(\btheta)}
%\frac{\tilde p(\bx)}{p(\bx)}
$$


\subsection{Strategies for sequential SBI}

\begin{figure}[h]
    \centering
    \includegraphics[width=0.5\linewidth]{figures/Sequential.png}
    \caption{Simple visualization of how focusing the prior parameter range increases the simulation density in the region with high likelihood.}
    \label{fig:sequential}
\end{figure}

when learning the likelihood function
$$
p(\btheta \mid \bx) \simeq 
\frac1{Z(\bx)}
q_\phi(\bx \mid \btheta)
p(\btheta)
%\frac{p(\btheta)}{p(\bx)}
%\frac{\tilde p(\bx)}{p(\bx)}
$$

When learning the ratio
$$
p(\btheta \mid \bx) \simeq 
%\frac1{Z(\bx)}
r_\phi(\bx; \btheta)
p(\btheta)
%\frac{p(\btheta)}{p(\bx)}
%\frac{\tilde p(\bx)}{p(\bx)}
$$

Common choices are posterior
$$
\tilde p(\btheta) = q_\phi(\btheta \mid \bx)
$$
or combination of posterior and prior
$$
\tilde p(\btheta) = \sqrt{
q_\phi(\btheta \mid \bx)
p(\btheta)
}
$$
or truncated prior
$$
\tilde p(\btheta) = \frac1Z 
p(\btheta) \mathbb{1}( q_\phi(\bx \mid \btheta) > \epsilon)
$$

Problem with marginalization: If we split parameters $\btheta = (\bphi, \blambda)$, then the learned marginal likelihood function becomes
$$
q_\phi(\bphi \mid \bx) 
\simeq \int d\blambda \;
p(\bx \mid \bphi, \blambda) 
\tilde p(\bphi \mid \blambda)
$$
and generally assume a proposal function $\tilde p(\btheta) = \tilde p(\bphi) p(\blambda \mid \bphi)$.



\paragraph{Sequential Neural Likelihood/Ratio Estimation (SNLE \& SNRE).} 
Sequential techniques are relatively easy to implement in situations where we approximate the data likelihood, $q_\phi(\bx \mid \bz) \approx p(\bx \mid \bz)$, with a neural network.  The reason is that the data likelihood does not formally depend on the prior distribution.  We expect in the end that
%
\begin{equation}
    q_\phi(\bx \mid \bz) = \tilde q_\phi(\bx \mid \bz)
    \quad \text{in high-probability regions of} \quad
    \tilde p(\bz)\;.
\end{equation}
%
Outside of the support of $\tilde p(\bz)$, the behaviour of $\tilde q_\phi(\bx \mid \bz)$ will be undefined.  By focusing training data on the parameters with a high data likelihood, one can effectively increase the density of training data in the region of interest.
%
Although the full model likelihood is prior independent, marginal likelihoods are generally not, and in general $q_\phi(\bx \mid z_i) \neq \tilde q_\phi(\bx \mid z_i)$.

A very similar behaviour can be observed for likelihood-to-evidence ratio estimation, where we train a network to approximate $\log f_\phi(\bx; \bz) \approx p(\bx | \bz)/p(\bx)$. Since the likelihood is independent of the proposal distribution, and only the parameter-independent evidence $p(\bx)$ is affected, we find that
%
\begin{equation}
    \log f_\phi(\bx; \bz) = \log \tilde f_\phi(\bx; \bz) + \text{const}
    \quad \text{in high-probability regions of} \quad
    \tilde p(\bz)\;.
\end{equation}
%
However, the same limitations apply in the case of training marginals.

\paragraph{Sequential Neural Posterior Estimation (SNPE).}

When performing neural posterior estimation as discussed above, the proposal distribution $\tilde p(\bz)$ affects the outcome.

Instead of using the prior $p(\mathbf z)$, it can be useful to use a proposal distribution $\tilde p(\mathbf z)$ that focuses on likely regions of the data given a specific observation $\mathbf x_o$.  Changing the prior has an effects on the posterior, which can be undone by multiplying the variational posterior with the prior-to-proposal ratio,
%
\begin{equation}
    q_\phi(\mathbf z \mid \bxobs)
    = \frac1Z
    \tilde q_\phi(\mathbf z \mid \bxobs)
    \frac{p(\mathbf z)}{\tilde p(\mathbf z)}
    \quad \text{in high-probability regions of} \quad
    \tilde p(\bz)\;.
\end{equation}
%
The partition function, $Z$, has formally the value $\frac{p(\mathbf x_o)}{\tilde p(\mathbf x_o)}$, which is usually unknown.  MCMC type techniques are typically used to sample from the posterior distribution in that case.

we see that the effect can be corrected for by multipling the inferred posterior by the factor $p(\mathbf z)/\tilde p(\mathbf z)$.  In general the evidence ratio will not be known.  This works well as long as the correction factor $p(\mathbf z) / \tilde p(\mathbf z)$ remains small over the range of the posterior.

Like for NLE above, it is relevant to observe that correction is not possible anymore if instead marginal posteriors are generated. In general, we cannot reconstruct  $q_\phi (z_i \mid \bxobs)$ from  $\tilde q_\phi (z_i \mid \bxobs)$, even if the prior functions and the proposal distribution is tractable and known.

\paragraph{Posterior approximation.}

A number of $R$, rounds with $\tilde p_1(\mathbf z) = p(\mathbf z)$, and $\tilde p_{i}(\mathbf z) = q_{\phi, i-1}(\mathbf z \mid \mathbf x_o)$ for $i = 2, \dots, R$.

\paragraph{Prior truncation strategies.}

We can see from Eq.~\eqref{eqn:SNPE}, that in cases where $p(\bz)/\tilde p(\bz)$ is constant over the high-probability region of $\tilde q_\phi(\bz \mid \bxobs)$, the expected correction is expected to cancels exactly with $Z$.  Sampling from $\tilde q_\phi(\bx \mid \bxobs)$, which can be done efficiently for instance when $\tilde q_\phi$ was modeled as a normalizing flow, will directly generate samples from the learned posterior $p(\bz \mid \bxobs)$.

This can be also seen at the level of marginal likelihoods.  Let us consider.
%
\begin{equation}
    \tilde p(\bx \mid z_i) = \int d\bz_{-i}\; p(\bx \mid \bz) \tilde p(\bz_{-i})\;.
\end{equation}
%
If the proposal distribution $\tilde p(\bz) \propto p(\bz)$ in regions where the likelihood $p(\bx \mid \bz)$ contributes significantly to the integration, we find that $\tilde p(\bx \mid z_i) \propto p(\bx \mid z_i)$.  Two options to achieve this is to
%
\begin{equation}
    \tilde p(\bz) = \frac1Z \mathbb{1}( \bz \in \Gamma) p(\bz)
\end{equation}
Here, the high-likelihood region $\Gamma \subset\Omega$ can be for instance selected as
%
\begin{equation}
    \Gamma = \{ \bz \in \Omega \mid q_\phi(\bx | \bz) > \epsilon\}\;,
\end{equation}
%
with some suitably small parameter for $\epsilon$.  Another option is based on tempered likelihood functions,
%
\begin{equation}
    \tilde p(\bz) = \frac1Z p(\bx \mid \bz)^\gamma p(\bz)\;.
\end{equation}
%
where $0< \gamma < 1$ is the tempering factor.

\medskip

Sequential SBI techniques are generally more simulation efficient for a given observation of interest, $\bxobs$, but require a retraining of notworks for each new observation.

Same initialization, but then $p_{i}(\mathbf z) = \frac1Z \mathbb{I}(\mathbf z \in \Gamma_i) p(\mathbf z)$.  Here, $\Gamma_i$ is derived from the posterior or likelihood of the preceding round.




\subsection{Practical examples}

\cw{FIGTODO Quality of posterior estimation in multiple rounds}

- Learning $10^{-10}$ posterior width in multiple rounds, simple summary network

- Handling of training data and network-reinitialization between rounds



\chapter{Concluding Remarks}
\label{chap:conclusions}

\begin{quotation}
\textit{``If you can't solve a problem, then there is an easier problem you can solve: find it.''}

\hfill --- George Pólya, How to Solve It (1945)
\end{quotation}

\noindent
These lecture notes have introduced simulation-based inference as a principled framework for parameter estimation when likelihoods are intractable but forward simulation is feasible. The power of this approach lies in its flexibility: by replacing analytic likelihood evaluation with learned approximations, SBI enables inference for complex generative models that would be impossible to handle with traditional methods.

However, this flexibility comes at a cost. Likelihood-based methods provide access to the \textit{exact} posterior (explicitly via symbolic computation, or implicitly via MCMC sampling). Simulation-based methods provide access only to an \textit{approximate} posterior $q_\phi(\btheta|\bx) \approx p(\btheta|\bx)$, learned from finite simulation data. This approximation must be carefully validated---training convergence does not guarantee correctness.

\section{Main Takeaways}

\begin{description}[style=nextline,leftmargin=0pt]
\item[\textbf{Exact versus approximate posteriors}]
Likelihood-based methods provide access to the \textit{exact} posterior (explicitly via symbolic computation, or implicitly via MCMC sampling). Simulation-based methods provide access only to an \textit{approximate} posterior $q_\phi(\btheta\mid\bx) \approx p(\btheta\mid\bx)$, learned from finite simulation data. This approximation must be carefully validated---training convergence does not guarantee correctness.

\item[\textbf{Three types of epistemic uncertainty guide diagnosis}]
Model misspecification (Type A), lossy compression (Type B), and inference approximation (Type C) represent distinct failure modes requiring different validation strategies. Most forward-backward tests detect Type C but are blind to Type B. Type A requires model criticism that goes beyond posterior validation. Understanding which uncertainty types your application is vulnerable to guides diagnostic choices.

\item[\textbf{Different inference approaches make different trade-offs}]
NPE enables end-to-end learning of information-maximizing summaries but requires expressive density estimators and bakes in the prior. NLE provides prior-independent likelihoods but cannot learn summaries end-to-end and requires MCMC for posterior sampling. NRE offers prior-independence with end-to-end learning but struggles in high dimensions. The choice depends on whether you need prior flexibility, have tractable likelihoods for validation, or can tolerate MCMC sampling costs.

\item[\textbf{Diagnostic strategies must be combined}]
No single test suffices. Forward-backward tests (SBC, coverage, C2ST) detect inference approximation but miss information loss. Reference posterior comparisons reveal both but require tractable ground truth. Model criticism (PPCs, robustness checks) identifies misspecification but cannot validate inference algorithms. Many of these can be unified through rank-based testing with different ordering functions, but effective validation requires combining complementary approaches tailored to your problem's failure modes.

\item[\textbf{Data compression requires careful design}] 
When exact sufficient statistics do not exist, learned summaries introduce a trade-off between computational tractability and information loss. Information-theoretic principles (mutual information, data processing inequality) guide summary construction and quantify compression costs. When simulator uncertainty is anticipated, summaries must be explicitly designed for robustness---neural methods do not automatically inherit robustness properties. The choice of what information to discard is often as important as what to retain.

\end{description}

\section{Open Challenges and Future Directions}


Despite rapid progress, several fundamental challenges remain:

\begin{description}[style=nextline,leftmargin=0pt]
\item[\textbf{Scaling diagnostics to high dimensions}]
Some validation methods become weak or computationally prohibitive in moderate to high dimensions. While rank-based tests remain tractable through one-dimensional projections, systematically detecting Type B uncertainties in high-dimensional settings is not well established. Moving beyond standard HPDR coverage diagnostics and developing comprehensive testing strategies---including efficient projection-based workflows and their implementation in standard software tools---remains an active challenge.

\item[\textbf{Reliable detection of information loss}]
Type B uncertainties are largely invisible to standard calibration tests like SBC and coverage diagnostics. Model-based rank diagnostics (Section~\ref{sec:model_based_ranks}) offer promise but require tractable likelihood evaluation, which limits their applicability. Finding general-purpose tests for information loss that work without explicit model access remains an open problem.

\item[\textbf{Practical robust inference}]
The information-theoretic framework for robust summary learning (Section~\ref{sec:robust_summaries}) provides clear conceptual guidance, but translating these principles into standardized, practical implementations requires further development. Moreover, while likelihood-based inference benefits from decades of established model misspecification diagnostics, comparable techniques for SBI are still emerging. Building consensus around diagnostic workflows and best practices for model criticism in simulation-based settings remains an active area of development.

\item[\textbf{Beyond amortization}]
The enforced amortization in neural SBI---learning $p(\btheta\mid \bx)$ for all $\bx$---may be wasteful when only a single observation matters. Sequential methods that adapt to specific observations show promise but introduce new challenges: designing efficient proposals, establishing stopping criteria, and developing robust software implementations. Particularly challenging are hierarchical models with shared parameters across observations, transdimensional models with variable parameter dimensionality, and non-parametric priors where the parameter space adapts to data. Developing efficient inference strategies for these structured scenarios remains an open frontier.
\end{description}

\section{Resources for Further Learning}

\cw{TODO: Improve this list}

\textbf{Software implementations:}
\begin{itemize}
\item \texttt{sbi} (Python): Comprehensive toolkit for neural SBI methods [\url{https://sbi-dev.github.io/sbi/}]
\item \texttt{sbibm}: Benchmark problems and baseline results [\url{https://sbi-benchmark.github.io/}]
\item \texttt{lampe}: Lightweight alternative implementation [\url{https://github.com/probabilists/lampe}]
\end{itemize}

\textbf{Key review articles:}
\begin{itemize}
\item Cranmer et al. (2020): ``The frontier of simulation-based inference'' [arXiv:1911.01429]
\item Papamakarios \& Murray (2016) and Hermans et al. (2020): Foundational papers on NPE and NRE
\item Talts et al. (2018): Simulation-based calibration methodology
\end{itemize}

\textbf{On diagnostics and validation:}
\begin{itemize}
\item Lemos et al. (2023): Sampling-based accuracy testing (TARP)
\item Modr\'{a}k et al. (2025): Comprehensive review of rank-based diagnostics
\item Cannon et al. (2022): Model misspecification in neural SBI
\end{itemize}

%\appendix
%%\chapter{Derivations and further discussions}
%
%\section{Neural posterior estimation and entropy}
%
%We explicitly derive here Eq.~\eqref{eqn:NPE_limit} in Sec.~\ref{sec:NPE}. To this end, we apply the large sample limit, $N\to \infty$, refactor some of the expressions, and exploit the definitions of the KL-divergence and the differential entropy.
%
%\begin{multline}
%\mathcal{L}_\text{NPE} = - \frac1{|\mathcal{D}|}
%\sum_{\btheta, \bx \in \mathcal{D}} \ln q_\phi(\btheta \mid \bx)
%\\
%\underset{N\to\infty}{\to}
%- 
%\int d\btheta\,d\bx\; p(\btheta, \bx) \ln q_\phi(\btheta \mid \bx)
%\hfill\text{Large sample limit}
%\\
%= - \int d\btheta\,d\bx\; p(\btheta \mid \bx) p(\bx) \ln q_\phi(\btheta \mid \bx)
%\hfill\text{Conditional probability}
%\\
%=
%\int d\bx\, p(\bx) \, \left[
%\int d\btheta\; p(\btheta \mid \bx) \ln 
%\frac
%{p(\btheta \mid \bx)}
%{q_\phi(\btheta \mid \bx)}
%- 
%\int d\btheta\; p(\btheta \mid  \bx) \ln 
%p(\btheta \mid \bx)
%\right]\hfill\text{Refactoring}
%\\
%= \mathbb{E}_{p(\bx)}
%\left[
%D_{KL}\left(p(\btheta \mid \bx) \mid\mid q_\phi(\btheta \mid \bx) \right)
%+ \mathcal{H}(p(\btheta \mid \bx))
%\right]
%\hfill\text{Using definitions}
%\label{eqn:NPE_connection}
%\end{multline}
%
%\section{Derivation of NRE}
%
%To derive a loss function with the above target, let us first consider a general binary cross-entropy loss function, which just follows from the forward KL divergence
%%
%\begin{equation}
%    \mathcal L =  - \mathbb E_{p(\mathbf v \mid y)p(y)} [\ln q_\phi(y \mid \mathbf v)]
%\end{equation}
%%
%We could interpret this as a discrete posterior challenge. Now we replace $\mathbf v$ with $(\mathbf x, \mathbf z)$ and define the generative model
%\begin{equation}
%    p(\mathbf v \mid y) = 
%    \delta_{y1} p(\mathbf x \mid \mathbf z) p(\mathbf z)
%    + \delta_{y0} p(\mathbf x) p(\mathbf z)
%    \label{eqn:AuxModel}
%\end{equation}
%where $\delta_{ij}$ denotes here the Kronecker delta (with $\delta_{ij} = 1$ for $i=j$ and $\delta_{ij}=0$ otherwise).
%%
%The final loss function acquires then the form
%\begin{equation}
%    \mathcal L =  
%    - \mathbb E_{p(\mathbf x \mid z)p(\mathbf z)} [\ln \sigma (f_\phi(\mathbf x, \mathbf z)]
%    - \mathbb E_{p(\mathbf x)p(\mathbf z)} [\ln \sigma (-f_\phi(\mathbf x, \mathbf z)]
%\end{equation}
%%
%One can show now that the function is minimized by
%\begin{equation}
%    f_\phi(\mathbf x, \mathbf z)\approx
%    \ln \frac{p(\mathbf x, \mathbf z)}{p(\mathbf x)p(\mathbf z)}
%    = \ln \frac{p(\mathbf x \mid \mathbf z)}{p(\mathbf x)}
%    = \ln \frac{p(\mathbf z\mid \mathbf x)}{p(\mathbf z)}
%\end{equation}
%%
%where the last two expressions are just an application of conditional probabilities.  Depending on the choice of probability distributions in Eq.~\eqref{eqn:AuxModel}, other density ratios (like log-likelihood-ratios) could be learned as well.  Not that here $f_\phi(\mathbf x, \mathbf z)$ is just a real-valued neural network without any additional constraints.


\section{Interpretation of JS-divergence}

What does a large JS-divergence signify? To see this, we consider the average Bayesian probability of error of wrongly classifying a parameter-data pair as drawn from the prior or the posterior, which is given by
%
\begin{equation}
    \hat P_e = \mathbb{E}_{\bx \sim p(\bx)} \left[\frac12 \min(p(\bz|\bx), p(\bz))\right]
\end{equation}
%
In Ref.~\cite{XYZ} it was shown that the minimum of the NRE loss provides an upper bound on the error rate,
%
\begin{equation}
    \hat P_e
    \leq
    2\log 2 -  
    2 \mathbb{E}_{\bx\sim p(\bx)}\left[D_{JS} ( p(\bz|\bx) \;||\; p(\bz))\right]
    \leq \ell_{NRE}[F_\phi]\;.
\end{equation}

When using NRE as described above, the summary is optimized such that it minimizes (an upper bound on) the error rate when classifying points as drawn from the prior vs. drawn from the posterior, or (equivalently) whether the likelihood-to-evidence ratio is larger or smaller than one,
%
\begin{equation}
    \frac{p(T(\bx) | \btheta)}{p(T(\bx))}  = 
    \frac{p(\btheta| T(\bx))}{p(\btheta)} 
    \lessgtr 1\;.
\end{equation}

\begin{equation}
\ell[f_\phi] = 
-2 \mathbb{E}_{p(\bx)}\left[
\text{JSD}(p(\btheta|\by = F(\bx)) || p(\btheta))
\right]
\end{equation}



\section{Loss function sensitivity analysis}

Let us consider a flexible function $q^\text{NPE}_\phi(\btheta|\bx)$ that we assume to be by construction non-negative and normalized to one, $\int d\btheta \,q^\text{NPE}_\phi(\btheta|\bx)=1$, for all values of $\phi$ (an example are normalizing flows).  Consider now the loss function
%
\begin{equation}
    \ell_{NPE}[q^\text{NPE}_\phi] = - \int d\bx\,d\btheta\,p(\bx, \btheta) \log q^\text{NPE}_\phi(\btheta|\bx)\;.
\end{equation}
%
Using Jensen's inequality and the fact that the log is a concave function, one can show that this function is minimized when
%
\begin{equation}
    q^\text{NPE}_\phi(\btheta|\bx) \approx p(\btheta|\bx)\;,
\end{equation}
and hence our normalized density will approximate the posterior. If we use the same loss function for $q_\phi(\bx|\btheta)$
%
\begin{equation}
    \ell_{NLE}[q^\text{NLE}_\phi] = - \int d\bx\,d\btheta\,p(\btheta, \bx) \log q_\phi(\btheta|\bx)\;.
\end{equation}
%
we instead find that it is minimized by
%
\begin{equation}
    q^\text{NLE}_\phi(\bx|\btheta) \approx p(\bx|\btheta)\;.
\end{equation}


\paragraph{Theoretical sensitivity analysis.}

We can now expand around the minimum up to second order in $\Delta q(\btheta |\bx) \equiv q(\btheta|\bx) - p(\btheta|\bx)$.  This leads to 
%
\begin{equation}
    \ell_{NRE} \simeq 2\log 2 - 
    2 \mathbb{E}_{p(\bx)}[\text{JSD} ( p(\btheta|\bx) || p(\btheta)]
  + \frac12 \mathbb{E}_{p(\bx)} \left[
    \int d\btheta\,
    \frac{p(\btheta) p(\btheta|\bx)}{p(\btheta) + p(\btheta|\bx)}
    \cdot \left( \frac{\Delta q(\btheta|\bx)}{p(\btheta|\bx)}\right)^2
    \right]
\end{equation}
%
and
%
\begin{equation}
  \ell_{NPE} \simeq \mathbb{E}_{p(\bx)} [H(p(\btheta|\bx))]
  + \frac12 \mathbb{E}_{p(\bx)} \left[
  \int d\btheta \, 
  p(\btheta|\bx)
    \cdot \left( \frac{\Delta q(\btheta|\bx)}{p(\btheta|\bx)}\right)^2
    \right]
\end{equation}
%
and
%
\begin{equation}
  \ell_{NLE} \simeq \mathbb{E}_{p(\btheta)} [H(p(\bx|\btheta))]
  + \frac12 \mathbb{E}_{p(\btheta)} \left[
  \int d\bx\, 
  p(\bx|\btheta)
    \cdot \left(\frac{\Delta q(\bx|\btheta)}{p(\bx|\btheta)}\right)^2
    \right]
\end{equation}




\begin{figure}
\centering
\begin{tikzpicture}
  % Nodes
  \node[latent] (x) at (0,0) {$x$};
  \node[latent] (y) at (2,0) {$y_i$};
  \node[obs] (z) at (4,0) {$z$};

  % Edges
  \edge {x} {y};
  \edge {y} {z};

  % Plate
  \plate {plate1} {(y)(z)} {$N$} ;

  % Labels
%  \node[below of=plate1.south, node distance=0.5cm] {Instances of $y_i$ for $i=1,\ldots,N$};

\end{tikzpicture}
\caption{This is a test caption}
\end{figure}




%\chapter{Information theory basics}
%
%
%\section{Score function and Fisher information} 
%
%In the context of \textit{general probabilistic modeling}, the \textit{score function} of a probability density function $p(\mathbf v)$ is defined as the gradient of the log probability,
%%
%\begin{equation}
%    s(\mathbf v) \equiv 
%    \nabla_{\mathbf v} \ln p(\mathbf v) 
%    = \frac {\nabla_{\mathbf v} p(\mathbf v)}{p(\mathbf v)} 
%    \;.
%\end{equation}
%%
%The score function of a normal distribution, $\mathcal{N}(x; \mu, \sigma^2)$, is for instance $s(x) = \frac{x-\mu}{\sigma^2}$;  the score of an energy based model is $s(\mathbf v) = -E(\mathbf v)$.
%
%The above definition is prevalent in the machine learning literature.  However, in \textit{classical statistics}, the \textit{score function} has a slightly different meaning and specifically refers to the derivative of the log likelihood function w.r.t.~model parameters: $s(\mathbf z; \mathbf x) \equiv  \nabla_{\mathbf z} \ln p(\mathbf x\mid \mathbf z)$.
%Note that here the derivative is not taken w.r.t.~the random variates of the distribution, but w.r.t.~the conditionals.
%
%The \textit{Fisher information} matrix~\cite{} is defined as the covariance matrix of the classical score function\footnote{One can show that the mean of the score function is zero.}
%\begin{equation}
%\mathcal{I}(\mathbf z) = 
%\mathbb{E}_{\mathbf x \sim p(\mathbf x \mid \mathbf z)}
%\left[ 
%\mathbf s(\mathbf z; \mathbf x)^T
%\mathbf s(\mathbf z; \mathbf x)
%\right]
%\end{equation}
%
%Let $\hat {\mathbf z}(\mathbf x)$ be any unbiased estimator of $\mathbf z$.  One can show that (Carmér-Rao bound)
%%
%\begin{equation}
%    \text{Var}_{\mathbf x \sim p(\mathbf x \mid \mathbf z)}[\hat{\mathbf z}(\mathbf x)] \geq \mathcal{I}(\mathbf z)^{-1}\;.
%\end{equation}
%%
%Hence, the Fisher information matrix provides a lower limit to the amount of information we can extract. 
%\cw{Add references}
%
%
%\section{Empirical distribution} Given some observational samples $\mathbf x^{(1:N)}$, we can also define the \textit{empirical distribution}
%%
%\begin{equation}
%    q(\mathbf x) = \frac1N \sum_{i=1}^N 
%    \delta_D(\mathbf x - \mathbf x^{(i)} )
%\end{equation}
%%
%where we made use of the Dirac delta function $\delta_D$, which we will use frequently below. Averaging over the empirical distribution is equivalent to averaging over its samples
%%
%\begin{equation}
%    \mathbb{E}_{q(\mathbf x)}[f(\mathbf x)]
%    = \frac1N \sum_{i=1}^N f\left(\mathbf x^{(i)}\right)\;.
%\end{equation}
%%
%
%%\chapter{Divergence measures in probability theory}
%
%\begin{figure}[t]
%    \centering
%    \includegraphics[width=0.8\linewidth]{figures/divergences.png}
%    \caption{Visualization of the integrants of Kullback-Leibler, Jensen-Shannon and Fisher divergences.}
%    \label{fig:divergence_integrands}
%\end{figure}
%
%In Fig.~\ref{fig:divergence_integrands} we visualize different divergence measures.
%
%\section{Kullback-Leibler divergence}
%
%The Kullback-Leibler (KL) divergence is defined as
%%
%\begin{equation}
%    D_{KL}(q \mid \mid p) \equiv \mathbb{E}_{q(\mathbf v)} \left[ 
%    \ln \frac {q(\mathbf v)}{p(\mathbf v)}  \right]\;.
%\end{equation}
%%
%We see that $D_{KL}(q\mid\mid p) = 0$ if $q(\mathbf v) = p(\mathbf v)$.  It is
%furthermore straightforward to show that the KL divergence is
%non-negative, $D_{KL}(q\mid\mid p) \geq0$, for any probability density functions $q(\mathbf v)$ and $p(\mathbf v)$. We exploit that $\ln a \leq a-1$ with $a = p(\mathbf v)/q(\mathbf v)$ to write $D_{KL}(q \mid\mid p) \geq \int d\mathbf v \; q(\mathbf v)[p(\mathbf v)/q(\mathbf v) - 1] = \int d\mathbf v [p(\mathbf v) - q(\mathbf v)] = 1-1 = 0$.  We here only used that the probability densities are normalized.
%
%The KL divergence is not symmetric, which has consequences for its behaviour when using it as a fitting target.  
%%
%\begin{itemize}
%    \item $D_{KL}(p \mid\mid q)$ is mass covering, meaning that $q$ typically spreads out to cover all regions where $p$ is non-zero.
%    \item $D_{KL}(q \mid\mid p)$ is mode seeking, meaning that $q$ has the tendency to focus on a few or a single mode of $p$.
%\end{itemize}
%
%
%\section{Jensen-Shannon divergence} A common symmetrised version of the density function is the Jensen-Shannon (JS) divergence
%%
%\begin{equation}
%   D_{JS}(q\mid\mid p) 
%   \equiv \frac12 \left[D_{KL}(q\mid\mid m) + D_{KL}(p\mid\mid m)\right]\quad\text{with}
%   \quad m(\mathbf v) \equiv  \frac12\left[q(\mathbf v) + p(\mathbf v)\right]
%\end{equation}
%%
%which can be shown to be bound by $0\leq D_{JS}(q\mid\mid p) \leq \ln 2$ for the natural logarithm.  
%% CHECK
%
%\section{Fisher divergence}
%
%Another useful divergence is the Fisher divergence, defined as
%%
%\begin{equation}
%    D_F(q\mid\mid p) = \mathbb E_{q(\mathbf v)}
%    \left[
%    || 
%    \nabla_{\mathbf v} \ln q(\mathbf v)
%    - \nabla_{\mathbf v} \ln p(\mathbf v)
%    ||^2
%    \right]
%\end{equation}
%%
%which in contrast to all above distributions does not depend on the distributions $q$ and $p$ being normalized. The reason is that the score is in general invariant under multiplying the probability density with a constant factorj.
%%\ja{Does it require they are normalised to the same thing?}
%
%Interestingly, there is a relation (\textit{Bruijin’s identity}) which connects the Fisher and KL divergences,
%%
%\begin{equation}
%\frac{d}{d t} D_{K L}\left(q_t(\tilde{\mathbf{v}}) \| p_{\boldsymbol{\theta}, t}(\tilde{\mathbf{v}})\right)=-\frac{1}{2} D_F\left(q_t(\tilde{\mathbf{v}}) \| p_{\boldsymbol{\theta}, t}(\tilde{\mathbf{v}})\right) .
%\end{equation}
%
%Here, $q_t(\mathbf {\tilde v})$ and $p_t(\mathbf {\tilde v})$ denote smoothed versions of the original distributions,
%resulting from adding Gaussian noise to $\mathbf v$ with variance $t$ 
%($\mathbf {\tilde x}
%\sim \mathcal{N}(\mathbf x, t\mathbf I$).
%%CHECK \ja{Is this relation useful at all?}
%\cw{Explain how this relates or motivates diffusion-based models.}
%
%
%\section{Entropy.} We can define the (differential) entropy 
%
%%\ja{How much of this will you actually cover?} Quite a bit
%%
%\begin{equation}
%    \mathcal{H}(p) = -\mathbb{E}_{p(\mathbf v)}
%    \left[\ln p(\mathbf v)\right]
%\end{equation}
%\cw{Explain details about entropy}



\chapter{Finite samples}

The behavior of the loss minimiser in the large sample limit can be studied using standard statistical methods for M-estimators~\cite{M_estimators}. This neglects effects related to stochastic optimisation and neural network-specific regularisation, but still provides useful results. 
%
We start by Taylor expanding the loss function around the minimum, $\bphi$, in the large sample limit.  We keep terms up to second order in $\bepsilon$,
%
$$
\mathcal{L}[\mathcal D, \bphi + \bepsilon] \approx
\mathcal{L}[\mathcal D, \bphi]
+ \bepsilon^T 
\underbrace{\nabla_{\bphi} \mathcal{L}[\mathcal D, \bphi]}_{\equiv \textbf g[\mathcal D]}
+ \frac12 
\bepsilon^T  
\underbrace{\nabla_{\bphi}^2 \mathcal{L}[\mathcal D, \bphi]}_{\equiv \textbf{H}[\mathcal D]}\;,
\bepsilon 
$$
%
%In the large sample limit, the average gradient term vanishes, since $\bphi$ minizes the loss. 
where we introduced the gradient $\textbf g[\mathcal D]$ and the Hessian matrix $\textbf H[\mathcal D]$.  One can show that the covariance of the second term on the right-hand side is given by
%
$$
\text{Cov}_\mathcal{D}(\textbf g[\mathcal D])
= 
\underbrace{\frac1N \text{Cov}_{p(\bx, \btheta)}\left[\nabla_{\bphi}\ell_\phi(\bx, \btheta)\right]
}_{\equiv \textbf C}\;,
$$
%
which exploits the independent of $\mathcal D$ samples.
Furthermore, the expected (large sample limit) Hessian is given by
$$
\bbE_\mathcal{D}[\textbf H [\mathcal D]]
= 
\underbrace{\bbE_{p(\bx, \btheta)}\left[\nabla^2_{\bphi}\ell_\phi(\bx, \btheta)\right]
}_{\equiv \textbf H}\;.
$$
Based on these results, we can derive the variance of the loss funciton minimizer and estimate the covariance of the minimum solution to the loss.  We find
$\text{Cov}(\bepsilon) = \textbf H^{-1} \textbf C \textbf H^{-1}$.

\bigskip

Let us assume that $\bepsilon$ is defined through $q_\phi(\btheta \mid \bx) = p(\btheta \mid \bx) + \sum_i \epsilon_i \delta q^{(i)}(\btheta \mid \bx)$.
For NPE we then find that
$$
H_{ij} = \int d\bx\, d\btheta\, p(\bx, \btheta)
\frac{
\delta q^{(i)}(\btheta \mid \bx)
\delta q^{(j)}(\btheta \mid \bx)
}
{p(\btheta \mid \bx)^2}
$$
and $C_{ij} = \frac1N H_{ij}$. If we assume that $H_{ij}$ is roughly diagonal, we find that

$$
\frac{
\delta q(\btheta^\ast \mid \bx^\ast)
}
{p(\btheta^\ast \mid \bx^\ast)}
\lesssim \frac C{\sqrt{N\cdot p(\bx^\ast, \btheta^\ast) }}
$$
where $\bx^\ast, \btheta^\ast = \argmax \delta q(\btheta \mid \bx)$, and is the fractional volume of the generative model.

For NRE, we instead obtain,
$$
\frac{
\delta q(\btheta^\ast \mid \bx^\ast)
}
{p(\btheta^\ast \mid \bx^\ast)}
\lesssim 
\frac C{\sqrt{N\cdot\min[
p(\bx^\ast, \btheta^\ast),
p(\bx^\ast)p(\btheta^\ast) 
]}}
$$




\chapter{Test-function-based comparison (optional)}

\textit{Requires: Samples from $p(\btheta \mid \bxobs)$ and $q_\phi(\btheta \mid \bxobs)$.}

The binning-based divergence tests discussed above are useful for comparing one- or two-dimensional marginals, but they do not scale well with dimensionality and can obscure structure in the joint posterior. When working with high-dimensional samples, it is often more practical to use non-parametric, sample-based comparisons that avoid explicit binning. 

In principle, classical tests like the Kolmogorov-Smirnov statistic can be applied to individual posterior marginals, but they are limited to one dimension and insensitive to structural mismatches. In this context, we highlight MMD, which generalize naturally to higher dimensions and capture both global and local discrepancies.

\medskip

The MMD is a kernel-based distance between two sets of samples in arbitrary dimension. It compares distributions based on differences in expectations over a reproducing kernel Hilbert space (RKHS). Given two sample sets $\{\btheta_i\}_{i=1}^N \sim q_\phi$ and $\{\btheta'_j\}_{j=1}^M \sim p$, the unbiased estimator for MMD is
\[
\text{MMD}^2 = \frac{1}{N(N-1)} \sum_{i \neq j} k(\btheta_i, \btheta_j) + \frac{1}{M(M-1)} \sum_{i \neq j} k(\btheta'_i, \btheta'_j) - \frac{2}{NM} \sum_{i, j} k(\btheta_i, \btheta'_j),
\]
where $k(\cdot,\cdot)$ is a positive-definite kernel, typically a Gaussian kernel. In practice, performance depends on the kernel bandwidth and on appropriate scaling of the parameter space. Whitening the samples—e.g., by transforming them to zero mean and unit variance—can improve sensitivity to discrepancies. MMD is especially useful in moderate dimensions where binning-based methods become infeasible.


- MMD provides a principled and efficient approach to comparing posteriors in higher dimensions

- Other sample-based options include Wasserstein distances and energy statistics, 

- These methods typically do not provide insight into *what* is wrong with the posterior approximation—only *that* there is a discrepancy. 

- Nonetheless, they are valuable tools for validating the quality of approximate inference results, especially when paired with methods that reveal structural errors, such as coverage tests or classifier-based diagnostics discussed in Sec.~\ref{sec:calibration_and_rank_tests}

\cw{FIGTODO Optional figure: Showing for different misemodeling the discrepancies in a table - only makes sense with uncertainty estimate based on sample number}

Use JS divergence for simple sanity checks in low-dim, binned comparisons.

Use KS and Wasserstein for interpretable 1D projections (e.g., marginals or summary directions).

Use MMD as a general-purpose scalar metric, especially if kernel bandwidth is adapted to the posterior shape.

Use C2ST when you want a flexible, high-dimensional test that can also help locate why and where things go wrong.



\medskip

One can show that the TV and the Pearson's $\chi^2$ divergence are connected through the inequality
$$
\text{TV}(q_\phi, p)^2 \leq \frac{1}{4} D_{\chi^2}(q_\phi \parallel p)\;.
$$




\chapter{Gradient signal on summary statistics.}

\emph{Amortization is not only a computational strategy—it is also essential for shaping the summary representation.} The effective learning signal for the summary network \( T_\phi \) arises only when data points with different parameters are contrasted during training. Assuming the conditional estimator \( q_\phi(\btheta \mid T_\phi(\bx)) \) has converged for a given \( T = T_\phi(\bx) \), the summary updates according to
\[
\delta T \propto \nabla_T \log q(T \mid \btheta) - \nabla_T \log q(T),
\quad \text{with} \quad
q(T) = \int d\btheta\, q(T \mid \btheta) \, p(\btheta).
\]
Here, \( q(T \mid \btheta) \) is not modeled explicitly, but defined conceptually via Bayes’ rule from the trained conditional \( q_\phi(\btheta \mid T) \) and the known prior \( p(\btheta) \). This gradient contains an attractive term that pulls \( T \) toward the region associated with \( \btheta \), and a repulsive term that pushes it away from other \( \btheta \). 

The result is an emergent organization of the summary space: representations \( T \) self-cluster according to the parameters of interest. In this sense, the training dynamics \emph{can be seen as a form of supervised learning in disguise}, where \( \btheta \) implicitly guides the structure of the latent representation. Crucially, this effect only arises because amortization exposes the model to a population of diverse \( (\bx, \btheta) \) pairs—providing both “positive” and “negative” examples to guide the geometry of \( T_\phi(\bx) \).
These insights underscore the importance of choosing architectures and training strategies that support both flexible posterior modeling and the emergence of informative data summaries.


\bibliographystyle{unsrtnat}
\bibliography{references}

\end{document}
