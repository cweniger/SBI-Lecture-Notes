\section{Model Comparison in Simulation-Based Inference.}

In many applications, we wish not only to assess the adequacy of a single simulator, but to compare competing models and quantify which provides a better explanation of the observed data. In the Bayesian setting, this is formalized via the \emph{Bayes factor}, often denoted \( \mathcal{K}_{01} \), and defined as
\[
\mathcal{K}_{01}(\bxobs) = \frac{p(\bxobs \mid \mathcal{H}_0)}{p(\bxobs \mid \mathcal{H}_1)} \,,
\]
where \( \mathcal{H}_0 \) and \( \mathcal{H}_1 \) are competing models, and the model evidence or marginal likelihood is given by
\[
p(\bx \mid \mathcal{H}_j) = \int p(\bx \mid \btheta, \mathcal{H}_j)\, p_j(\btheta)\, d\btheta \,.
\]
In simulation-based inference, these marginal likelihoods are typically intractable, so exact Bayes factor evaluation is not possible.

A practical alternative is to learn a discriminative classifier to distinguish between samples from the models. Given simulators for \( \mathcal{H}_0 \) and \( \mathcal{H}_1 \), we train a scalar function \( f_\phi(\bx) \in \mathbb{R} \) to approximate the log-density ratio,
\[
f_\phi(\bx) \approx \log \frac{p(\bx \mid \mathcal{H}_1)}{p(\bx \mid \mathcal{H}_0)} = - \log \mathcal{K}_{01}(\bx) \,.
\]
The sign of \( f_\phi(\bx) \) indicates which model is preferred, and its magnitude reflects the strength of evidence.

In practice, \( \bx \) may first be mapped through a learnable summary network \( T_\phi(\bx) \), and the classifier is applied to this representation. The effectiveness of the comparison then hinges on how well \( T_\phi(\bx) \) captures model-distinguishing features. A natural loss for training \( T_\phi \) is the \emph{binary cross-entropy} (BCE),
\[
\mathcal{L}_\mathrm{BCE} = \mathbb{E}_{\bx \sim p_1} \log\left(1 + e^{-f_\phi(T_\phi(\bx))} \right) + \mathbb{E}_{\bx \sim p_0} \log\left(1 + e^{f_\phi(T_\phi(\bx))} \right) \,,
\]
which is minimized when \( f_\phi(\bx) \) approximates the log-density ratio. This loss corresponds to the logit form of standard neural ratio estimation (NRE).

\medskip

It is instructive to examine the gradient signal used to update the summary representation. Assuming that \( f_\phi \) has been trained to optimality, the update to the summary network is given by
\[
\frac{\partial \mathcal{L}_\mathrm{BCE}}{\partial T_\phi(\bx)} = \sigma(\mp f_\phi(\bx)) \cdot \frac{\partial f_\phi(\bx)}{\partial T_\phi(\bx)} \,,
\]
where the sign \( \mp \) corresponds to the class label (i.e., \( \bx \sim p_1 \) or \( \bx \sim p_0 \)), and \( \sigma(f) = (1 + e^{-f})^{-1} \) is the sigmoid. As \( f_\phi(\bx) \to \pm \infty \), the sigmoid saturates and the gradient decays exponentially, leading to vanishing learning signal for the summary network \( T_\phi \). This limits the ability to further refine representations when the Bayes factor becomes large—i.e., when the models are already well-separated.

\medskip

To address this limitation, the \emph{POP-III loss}~\cite{popiii2023} modifies the BCE objective to maintain informative gradients even for extreme evidence ratios. It is defined as
\[
\mathcal{L}_\mathrm{POP\text{-}III} = \mathbb{E}_{\bx \sim p_1} \left[ \left( \log(1 + e^{-f_\phi(T_\phi(\bx))}) \right)^2 \right]
+ \mathbb{E}_{\bx \sim p_0} \left[ \left( \log(1 + e^{f_\phi(T_\phi(\bx))}) \right)^2 \right] \,.
\]
This loss defines the squared logit as a test statistic,
\[
t_\mathrm{POP\text{-}III}(\bx) = f_\phi(T_\phi(\bx))^2 \,,
\]
and yields the gradient
\[
\frac{\partial \mathcal{L}_\mathrm{POP\text{-}III}}{\partial T_\phi(\bx)} = \pm 2 f_\phi(\bx) \cdot \sigma(\mp f_\phi(\bx)) \cdot \frac{\partial f_\phi(\bx)}{\partial T_\phi(\bx)} \,.
\]
Here, the multiplicative factor \( f_\phi(\bx) \) ensures that the gradient does not vanish for large values of the log-density ratio, allowing the summary network to keep improving representations even in the presence of extreme Bayes factors.

\medskip

Classifier-based model comparison thus benefits from calibrated loss functions that preserve learning signals across the full range of evidence strengths. Losses like POP-III provide robust approximations to Bayes factors in simulation-based settings and enable effective training of summary networks without saturation.