\chapter{Advanced Topics}
\label{chap:adv}

\section{Inference Under Simulator Uncertainty}
\label{sec:adv:uncertainty}

Model uncertainty is a central concern in any form of model-driven inference, and simulation-based inference is no exception. In practice, simulators are inevitably approximations: they incorporate unknown biases, omit physical effects, and make simplifying assumptions. Using an incorrect model introduces type~A epistemic uncertainties, arising from mismatches between the assumed and the true data-generating process. A critical step toward increasing the practical applicability of SBI is therefore to develop methods that make inference resilient to such mismodeling---ideally, procedures that remain reliable even when the simulator is inaccurate in certain respects. In what follows, we formalize the problem and identify principled strategies for achieving robustness.


\subsection{Defining Robustness in Simulation-Based Inference}
\label{sec:adv:uncertainty:defining}


\subsubsection{Model Misspecification in SBI}

The starting point for discussions of model misspecification and Bayesian inference is the (hypothetical) true data-generating process \(p_0(\bx)\)~\citep[\fex][]{kleijn_misspecification_2006, shalizi_dynamics_2009}. It represents the distribution of observations produced by nature, including all physical, instrumental, and environmental effects. In practice, $p_0$ is neither known nor can it be directly simulated; it serves only as a conceptual reference for reasoning about correctness and bias.

In context of the machine-learning literature, a model \(p(\bx\mid\bgamma)\) is often considered well-specified if there exists a configuration \(\bgamma^\ast\) such that \(p(\bx\mid\bgamma^\ast)=p_0(\bx)\)~\citep[\fex][]{cannon_investigating_2022}. However, in the physical sciences we are not primarily interested in matching the overall distribution of observed data.  What matters is whether the simulator captures the \emph{conditional} mechanism linking the physical parameters of interest, \(\btheta\), to observations. 

To formalise this, we write the simulator-based generative model as
\begin{equation}
    p(\bx \mid \btheta, \bgamma)\;,
    \label{eq:sbi-generative-model}
\end{equation}
where $\btheta$ denotes the physical parameters of interest and $\bgamma$ 
indexes the simulator configuration, implementation choices, approximations, 
or unmodelled effects. Adequacy of the theoretical model requires that, for some simulator configuration setting \(\bgamma^\ast\),
%
\begin{equation}
p(\bx\mid\btheta,\bgamma^\ast)\approx p_0(\bx\mid\btheta)
\quad\text{for all relevant }\btheta\;,
\end{equation}
%
where \(p_0(\bx\mid\btheta)\) denotes the (hypothetical) \emph{true conditional data-generating mechanism}. This distinction between matching the data distribution and capturing the conditional mechanism is essential for defining robustness in SBI.

The role of \(\bgamma\) remains here intentionally broad. It may index different simulator versions, parametrise uncertain or imperfectly known components, encode missing physical effects, or simply stand in for aspects of the generative mechanism that the simulator does not reliably capture. In practice, the corresponding “true’’ configuration \(\bgamma^\ast\) is unknown and often not even meaningfully parameterisable; it serves only as the conceptual limit in which the simulator reproduces the correct conditional mechanism. Robustness in SBI therefore means designing inference procedures that remain reliable even when \(\bgamma^\ast\) is unknown and only approximated by the available simulator configurations.


\subsubsection{The Asymptotic Meaning of Robustness}

If we knew the correct simulator configuration $\bgamma^\ast$,  Bayesian inference would proceed in the usual way,
%
\begin{equation}
    p(\btheta \mid \bxobs, \bgamma^\ast)
    \propto p(\bxobs \mid \btheta, \bgamma^\ast)\, p(\btheta)\;.
\end{equation}
%
In this idealised setting, the physical parameters take some true value  $\btheta^\ast$, so that the true data-generating distribution satisfies
%
\begin{equation}
    p_0(\bx) = p_0(\bx \mid \btheta^\ast)\;.
\end{equation}
%
If the simulator contains a configuration $\bgamma^\ast$ that reproduces  the correct conditional mechanism, then inference based on $p(\bx\mid\btheta,\bgamma^\ast)$ will concentrate around $\btheta^\ast$ as more data become available.

\medskip
In practice, however, the correct configuration $\bgamma^\ast$ is unknown,  and using an incorrect value $\bgamma$ alters the posterior  $p(\btheta \mid \bxobs, \bgamma)$.   To understand what “wrong’’ means in a precise sense, it is useful to  consider the asymptotic regime in which many independent observations $\bx_{1:n}$ are drawn from $p_0(\bx)$ and analysed with the  misspecified model $p(\bx \mid \btheta, \bgamma)$.   Classical results on Bayesian asymptotics under misspecification \citep[\fex][]{white_maximum_1982, kleijn_misspecification_2006} show that the posterior then concentrates around the \emph{pseudo-true parameter}
%
\begin{equation}
    \btheta^\dagger(\bgamma)
    = \arg\min_{\btheta}
    D_{\mathrm{KL}}\!\left(
        p_0(\bx) \,\middle\|\, 
        p(\bx \mid \btheta , \bgamma)
    \right)\!,
\end{equation}
that is, the value of $\btheta$ whose likelihood is closest in the  Kullback--Leibler sense to the true distribution under the assumed  configuration~$\bgamma$.   For the correct configuration one recovers  $\btheta^\dagger(\bgamma^\ast)=\btheta^\ast$, while for any other value  of $\bgamma$ the limiting parameter will in general be biased.

Moreover, in the large-$n$ limit the posterior becomes increasingly concentrated around $\theta^{\dagger}(\gamma)$, shrinking at the usual $1/\sqrt{n}$ rate (a Bernstein-von Mises-type behaviour; see \cite{belomestny_bernsteinvon_2023} for a recent review). This means that even small biases induced by choosing an incorrect configuration $\gamma$ will be amplified across many observations, leading to combined inferences that do not converge to the true value $\theta^\ast$.

\paragraph{Why asymptotics matter} Although many experiments in the physical sciences are not repeatable in the strict sense, we routinely combine information from multiple independent observations. Ensuring that each single-event posterior behaves correctly in the asymptotic sense is therefore essential for the consistency of joint or hierarchical analyses~\citep{koers_misspecified_2023}.  Robustness in SBI thus ideally aims at designing inference procedures whose posteriors remain well behaved---in particular, concentrate near  $\btheta^\ast$ with appropriate uncertainty scaling---even when the  unknown configuration $\bgamma^\ast$ is only imperfectly approximated by  the available simulator family.


\subsubsection{Three Strategies for Robust SBI}

The asymptotic considerations above suggest that robustness requires controlling how inference depends on the unknown simulator configuration~$\bgamma^\ast$. In practice, one may distinguish three very broad strategies for addressing such misspecification, which align with the three components of SBI in Fig.~\ref{fig:sbi_overview}:\footnote{In actual scientific applications, approaches are mixed, calibration observations play a role, not everything in the literature strictly maps onto the presented categories. But we find it useful to draw here in broad strokes to frame the discussion and highlight main differences.} \emph{model augmentation}, \emph{robust summary learning}, and \emph{general Bayesian updating}. Each modifies a different component of the inference pipeline---the simulation model, the compressed data representation, or the Bayesian update rule---and each tends to exhibit different asymptotic behaviour.


\paragraph{Model augmentation}

A first approach treats $\bgamma$ as a latent nuisance variable and integrates it out,
%
\begin{equation}
    p(\btheta \mid \bxobs)
    \;\longrightarrow\;
    \frac{1}{Z}\!\int p(\bxobs \mid \btheta,\bgamma)\,p(\bgamma)\,p(\btheta)\,d\bgamma\;.
\end{equation}
%
Examples include modeling systematic uncertainties (parametrized though $\bgamma$) with uninformed priors, or the use of Gaussian processes (represented by $\bgamma$) to account for unmodeled components.  This strategy is expected to be asymptotically reliable if each observation has an independent configuration $\bgamma_i \sim p(\bgamma)$, and if its prior is correctly specified, $p(\bgamma)=p_0(\bgamma)$. Otherwise, the $\bgamma$–averaged simulator may itself become misspecified and the posterior concentrates at a pseudo-true parameter $\btheta^\dagger$.  
If $\bgamma$ is instead a global calibration parameter or systematic effect shared across all observations, treating it as i.i.d.\ noise is invalid, and it should be be jointly estimated as global parameter with $\btheta$. 

\paragraph{Robust summary learning}
A second strategy constructs summaries $T(\bx)$ such that inference depends on
$p(T(\bx)\mid\btheta,\bgamma)$ rather than on $p(\bx\mid\btheta,\bgamma)$,
\begin{equation}
    p(\btheta \mid \bxobs)
    \;\longrightarrow\;
    \frac{1}{Z}\,
    p\!\big(T(\bxobs)\mid\btheta,\bgamma\big)\,p(\btheta)\,,
    \qquad\text{for any }\bgamma\;.
\end{equation}
This essentially corresponds to masking those aspects of the data for which the simulator is misspecified.  Robustness is achieved when $T$ successfully removes the dependence on $\bgamma$, i.e.\ when $p(T(\bx)\mid\btheta,\bgamma)\approx p(T(\bx)\mid\btheta)$ for all relevant $\bgamma$, including the unknown $\bgamma^\ast$.   In this ideal situation the summary likelihood becomes well specified, and the posterior  is expected to asymptotically focus on $\btheta^\ast$.  In practice, however, removing all $\bgamma$-dependence maybe be difficult to achieve.  Robust summary learning therefore introduces a bias-variance trade-off: reducing sensitivity to $\bgamma$ inevitably discards some information about~$\btheta$.  

\paragraph{General Bayes}
A third strategy encompasses everything where the Bayesian update rule itself is modified with the intend to increase robustness. In a general form, it may be formalized as replacing the likelihood by an appropriate loss function~(see \cite{bissiri_general_2016}, and 
\cite{guedj_primer_2019} for connections to PAC Bayesian learning)
%
\begin{equation}
    p(\btheta \mid \bxobs)
    \;\longrightarrow\;
    \frac{1}{Z}\,\exp\!\big(-\ell(\btheta,\bxobs)\big)\,p(\btheta)\;,
\end{equation}
%
where $\ell(\btheta, \bx)$ is chosen to reduce the impact of misspecification.  Examples include tempered likelihoods, but also (in the broad sense that we adopt here) the analysis of denoised data.  In general, however, such updates would asymptotically concentrate at the loss-risk minimiser $\btheta_\ell^\dagger=\arg\min_\theta \mathbb{E}_{p_0}[\ell(\theta,X)]$, which typically differs from both $\btheta^\ast$ and the KL pseudo-true value.   Posterior uncertainty is not automatically calibrated unless $\ell$ is correctly scaled, and only special choices yield desirable asymptotic behaviour.

\medskip
These three approaches respectively correspond to modifying the simulator, the representation, or the posterior update rule.  In the remainder of this section we focus on \emph{robust summary learning}, which may offer the most principled path to asymptotic correctness.


\subsection{Robust Summary Learning}
\label{sec:adv:uncertainty:robust}

As motivated in Sec.~\ref{sec:adv:uncertainty:defining}, we will consider a generative model that includes both the parameters of interest \(\btheta\) and the latent model configuration \(\bgamma\), with the full joint distribution
%
\begin{equation}
    p(\bx, \btheta, \bgamma) = p(\bx \mid \btheta, \bgamma) p(\btheta \mid \bgamma) p(\bgamma)\;.
    \label{eqn:joint_model}
\end{equation}
%
Since we focus on situations where \(\bgamma\) captures uncertainties in the data-generating process rather than in our prior beliefs about \(\btheta\), we subsequently assume \(p(\btheta \mid \bgamma) = p(\btheta)\).   Furthermore, we are interested in the summary-induced generative model, where all dependence on \(\bx\) enters only through the learned summary \(T(\bx)\),
%
\begin{equation}
    p(T(\bx), \btheta, \bgamma) = p(T(\bx) \mid \btheta, \bgamma) p(\btheta) p(\bgamma)\;.
    \label{eqn:joint_model}
\end{equation}
%
For this specific but very generally applicable setting, we will explore how variations in the simulator configuration, $\bgamma$, biases inferential tasks, using concepts of information theory.  This will provide the necessary basis for defining optimization strategies and criteria for robust inference.


\subsubsection{Three Faces of Configuration Bias}
\label{sec:adv:uncertainty:robust:three}

Above, we assumed that the model parameters \(\btheta\) and simulator configurations \(\bgamma\) are statistically independent. (Even if they were not, modifying the data summaries \(T(\bx)\) would not alter that dependence.) This leaves three channels through which the simulator configuration can enter the inference procedure: the posterior, the data summary, and the likelihood.


\paragraph{Posterior bias: parameter estimates that depend on $\bgamma$}

When model configurations \(\bgamma\) vary, inference based on a given summary statistics \(T_\phi(\bx)\) can become biased or unstable.  Our goal is therefore to learn a summary mapping \(T_\phi(\bx)\) such that the resulting posterior becomes (approximately) independent of the simulator configuration. That is, we seek summaries for which
\[
p(\btheta \mid T_\phi(\bx), \bgamma) \simeq p(\btheta \mid T_\phi(\bx)) \quad \text{for all plausible } \bgamma\;.
\]
%
This expresses the requirement that, once the data are summarized by \(T_\phi(\bx)\), changing the simulator configuration \(\bgamma\) no longer affects inference over \(\btheta\).\footnote{Note that here $p(\btheta \mid T_\phi(\bx))$ is technically marginalized over $\bgamma$, but its main role is to simply provide a constant reference point. The specifically adopted prior is not relevant.} 

A natural way to quantify deviations from this ideal is via the mutual information between \(\btheta\) and \(\bgamma\), conditioned on \(T_\phi(\bx)\):
\begin{equation}
\mathbb{E}_{p(\bx \mid \bgamma)p(\bgamma)}
\left[D_{\mathrm{KL}}\left(p(\btheta \mid T_\phi(\bx), \bgamma) \mid\mid p(\btheta \mid T_\phi(\bx))\right)\right] 
\equiv
\mathcal{I}(\btheta; \bgamma \mid T_\phi(\bx)) 
\leq \varepsilon\;.
\label{eqn:posterior_bias}
\end{equation}
%
This quantity measures how much knowing \(\bgamma\) improves inference about \(\btheta\), given the summary. We use this mutual information as a proxy for posterior bias throughout. When \(\varepsilon = 0\), the posterior is completely robust to simulator configurations, independent of the prior over \(\bgamma\) (as long as it has sufficient support). 

\smallskip

Small values of \(\varepsilon\) indicate that inference is stable across different simulator configurations and approximates the result we would obtain under the correct (but unknown) \(\bgamma^\ast\). At the same time, this invariance must be balanced against the requirement that \(T_\phi(\bx)\) remains informative about \(\btheta\); overly aggressive bias suppression may lead to summaries that are too coarse for precise inference.

\paragraph{Summary bias: representation drift across $\bgamma$}

Even if posterior inference is stable across \(\bgamma\), the learned summary \(T_\phi(\bx)\) itself may retain residual dependence on the simulator configuration. This can occur when \(T_\phi(\bx)\) encodes aspects of the data that are irrelevant for inference over \(\btheta\) but still vary with \(\bgamma\). Such dependence is typically undesirable, as it makes the influence of \(\bgamma\) on the representation---and therefore on the overall inference pipeline---harder to diagnose and control.

It is therefore natural to require that the distribution of \(T_\phi(\bx)\) be invariant to changes of the simulator configuration \(\bgamma\), i.e.,
\[
p(T_\phi(\bx) \mid \bgamma) \simeq p(T_\phi(\bx)) \quad \text{for all plausible } \bgamma.
\]
This condition ensures that \(T_\phi(\bx)\) and \(\bgamma\) are statistically independent, meaning that observing the summary provides little or no information about the underlying simulator configuration. 

As before, this requirement can be formalized using mutual information:
\begin{equation}
\mathbb{E}_{p(\bgamma)}\left[D_{\mathrm{KL}}\left(p(T_\phi(\bx) \mid \bgamma) \,\|\, p(T_\phi(\bx))\right)\right]
\equiv
\mathcal{I}(T_\phi(\bx); \bgamma)  
\leq \varepsilon.
\label{eqn:summary_bias}
\end{equation}
%
In the limit \(\varepsilon \to 0\), this enforces strict simulator invariance of the summary, independent of the specific prior over \(\bgamma\). Note that while this promotes interpretability, it does \emph{not} by itself guarantee robustness of the posterior as in Eq.~\eqref{eqn:posterior_bias}: \(T_\phi(\bx)\) and \(\bgamma\) may still be dependent once \(\btheta\) is fixed.
\footnote{Marginal independence does not imply conditional independence. 
Let $\theta,\gamma\in\{0,1\}$ be independent and uniform. Define  
$T = \gamma$ when $\theta=0$ and $T = 1-\gamma$ when $\theta=1$.  
Then $T$ and $\gamma$ are deterministically dependent given $\theta$,
but marginally $p(T,\gamma)=p(T)p(\gamma)=\frac14$ for all $T$ and $\gamma$.}


\paragraph{Likelihood bias: $\bgamma$-dependence at fixed parameters}

The most direct impact on the inference process comes through the likelihood function.   Following the same logic as above, we can write the condition for a summary likelihood that is independent of the simulator configuration $\bgamma$,
\[
p(T_\phi(\bx) \mid \btheta, \bgamma) \simeq p(T_\phi(\bx) \mid \btheta) \quad \text{for all plausible } \bgamma\;.
\]

Again, this condition can be formalised and written in terms of mutual information  between the summary \(T_\phi(\bx)\) and the simulator configuration \(\bgamma\), conditioned on the parameter \(\btheta\):
\begin{equation}
\mathbb{E}_{p(\bx, \btheta, \bgamma)} 
\left[ 
\log \frac{p(T(\bx), \bgamma \mid \btheta)}{p(T(\bx) \mid \btheta)\, p(\bgamma \mid \btheta)} 
\right]
 \equiv \mathcal{I}(T(\bx); \bgamma \mid \btheta) 
    \leq \varepsilon \,.
    \label{eqn:likelihood_bias}
\end{equation}
%
This quantity captures how much additional information about the summary \(T(\bx)\) is provided by \(\bgamma\), beyond what is already explained by \(\btheta\). It therefore quantifies the stability of the summary likelihood function across simulator configurations.


\subsubsection{The Simulator Bias Identity}

\begin{figure}[t]
    \includegraphics[width=\linewidth]{figures/Venn2.pdf}
    \caption{Different contributions to inference bias visualised.}
    \label{fig:venn_simulator_bias_identity}
\end{figure}

All of the mutual-information–based bias measures introduced above can be unified in a single quantity that captures the total dependence of the inference process on the simulator configuration \(\bgamma\). This is the mutual information between the summarized simulator outputs \((\btheta, T_\phi(\bx))\) and the configuration \(\bgamma\), given by
%
\begin{equation}
\mathcal{I}(\btheta, T_\phi(\bx); \bgamma)
\equiv
\int d\btheta \, d\bx \, d\bgamma\;
p(\btheta, \bx, \bgamma) \log \left(
\frac{p(\btheta, T_\phi(\bx) \mid \bgamma)}{p(\btheta, T_\phi(\bx))} 
\right) \;,
\label{eqn:total_bias}
\end{equation}
%
and it quantifies the overall bias induced by varying simulator configurations. If this quantity vanishes, then the joint distribution of \(\btheta\) and \(T_\phi(\bx)\) is completely independent of \(\bgamma\), implying fully robust inference across simulator variants.


\paragraph{Simulator bias decomposition identity}  The various bias terms introduced above are linked by a simple and illuminating relation,
%
\begin{equation}
    \underbrace{\mathcal{I}(T_\phi(\bx); \gamma)}_{\text{Summary bias}}
    +
    \underbrace{\mathcal{I}(\btheta; \gamma \mid T_\phi(\bx))}_{\text{Posterior bias}}
    \;=\;
    \underbrace{\mathcal{I}(\btheta, T_\phi(\bx); \gamma)}_{\text{Simulator bias}}
    \;=\;
    \underbrace{\mathcal{I}(\btheta; \gamma)}_{\text{Prior bias}}
    +
    \underbrace{\mathcal{I}(\gamma ; T_\phi(\bx) \mid \btheta)}_{\text{Likelihood bias}}
    \label{eqn:bias_identity}
\end{equation}
which follows directly from the chain rule of mutual information. For a fixed summary map \(T_\phi(\bx)\), each term quantifies a different way in which the simulator configuration \(\bgamma\) can influence the inference problem. The identity therefore shows how summary, posterior, prior, and likelihood biases decompose the \emph{simulator bias}—the total dependence of the inference pipeline on \(\bgamma\).

A key consequence is that an upper bound on the likelihood bias automatically bounds both the summary and posterior bias. Since we assume that \(\btheta\) and \(\bgamma\) are independent a priori, the prior bias term vanishes. Thus, enforcing a sufficiently small likelihood bias is enough to suppress all remaining forms of configuration-induced bias.

\cw{TODO: Explain figure and add caption} This is shown in Fig.~\ref{fig:venn_simulator_bias_identity}.


\subsubsection{Constrained Optimisation for Robust Minimal Summaries}

As discussed in Sec.~\ref{sec:adv:uncertainty:robust}, robustness requires data summaries  $T(\bx)$ that remove sensitivity to simulator configurations~$\bgamma$ while  retaining predictive information about~$\btheta$.   Two complementary information-theoretic principles govern this trade-off:  (i) \emph{invariance} to different simulator configurations, and  (ii) controlled \emph{compression} to suppress spurious or unstable features.  We now formalize both ideas.

The simulator bias identity, Eq.~\eqref{eqn:bias_identity}, shows that robustness to simulator configurations~$\bgamma$ is achieved most directly by suppressing the posterior and summary bias, or equivalently the likelihood bias $\mathcal{I}(T(\bx);\bgamma\mid\btheta)$.  Minimizing this term with respect to the summary $T_\phi(\bx)$ alone, however, admits the trivial solution of a summary that is uninformative about $\btheta$. To avoid such collapse, one must simultaneously \emph{retain} parameter-relevant information, as given by $\mathcal{I}(T_\phi(\bx); \btheta)$, while \emph{removing} configuration-relevant information.  

\paragraph{Lagrangian objective for robust minimal summaries}

It is standard to express the trade-off discussed above through a Lagrangian objective. With the goal in mind that $T_\phi(\bx)$ 
should be \emph{informative}, \emph{robust} and \emph{minimal}, we can write as optimisation objective:
\begin{equation}
\label{eqn:IIB_objective}
\max_{T_\phi}\;
\underbrace{\mathcal{I}(T_\phi(\bx);\btheta)}_{
\substack{\text{Parameter}\\ \text{information}}}
-\lambda
\underbrace{\mathcal{I}(\btheta; \bgamma \mid T_\phi(\bx))}_{
\substack{\text{Posterior bias}}}
- \eta
\underbrace{\mathcal{I}(T_\phi(\bx);\bgamma)}_{
\substack{\text{Summary bias}}}
- \beta
\underbrace{\mathcal{I}(T_\phi(\bx);\bx)}_{
\substack{\text{Information}\\ \text{bottleneck}}}
\;.
\label{eqn:robust_minimal_summary_objective}
\end{equation}
%
Here, $\lambda, \eta, \beta \geq0$ are tunable hyperparameters that control the trade-off balance. We will discuss each of the four components of the optimisation objective in Eq.~\eqref{eqn:robust_minimal_summary_objective} separately.
\begin{itemize}
    \item \textbf{Parameter information}: Maximizing the first term in Eq.~\eqref{eqn:robust_minimal_summary_objective} is necessary to retain information about the parameters of interest, $\btheta$. This is often a automatic consequence of end-to-end learning of data summaries, see \fex\ the NPE loss in Eq.~\eqref{eqn:NPE_MI}.
    
    \item \textbf{Posterior bias}: As discussed in Sec.~\ref{sec:adv:uncertainty:robust:three}, this term controls how much the posterior $p(\btheta \mid T_\phi(\bx), \bgamma)$ depends on the simulator configuration $\bgamma$.  It appears in the literature in context of invariant risk minimization  \citep{arjovsky_invariant_2020} and invariant information bottleneck \citep{li_invariant_2022}.
    
    \item \textbf{Summary bias}: As discussed in Sec.~\ref{sec:adv:uncertainty:robust:three}, this term controls how much the data summary distribution, $p(T_\phi(\bx) \mid \bgamma)$, varies with $\bgamma$.  It appears in the literature in context of fair and invariant representation learning,  \citep{zemel_learning_2013, zhao_fundamental_2020}.  If $\lambda = \eta$, the term and the previous one can be replaced equivalently by the likelihood bias, but we kept the contributions here separate for clarity.
    
    \item \textbf{Information bottleneck}: This term penalises how much information the summary $T_\phi(\bx)$ retains of the full data $\bx$.  It suppresses unnecessary information in the summary $T_\phi(\bx)$, promoting a \emph{minimal} summary. In the literature it appears in context of information bottleneck ideas \citep{tishby_information_2000}.
\end{itemize}

The introduction of the information bottleneck term does not only promote minimalism, but can also enhance robustness. Simulator configuration invariance alone does not guarantee robustness of inference to \emph{unmodeled forms of mismodeling}, i.e.\ deviations not captured by~$\bgamma$. In particular high-dimensional complex summaries may still encode aspects of $\bx$ that are sensitive to unmodeled imperfections of the simulator.  The information bottleneck term can help to further reduce such pathways for mismodeling to propagate, into the inference results. 

\cw{MAYBE: Individual components visualised in Fig.~\ref{fig:Venn2}}

\cw{MAYBE: Technical implementation }

%\subsubsection{Technical Implementation Challenges}

The conditional mutual information in Eq.~\eqref{eqn:IIB_objective} can be, \fex\ estimated through an approximate variational upper bound based on the CLUB estimator~\citep{cheng_club_2020},
%
\begin{equation}
\mathcal{I}(T_\phi(\bx); \bgamma \mid \btheta) \lesssim
\mathbb{E}_{p(\bx, \btheta, \bgamma)} 
\left[ \log q_\psi(T_\phi(\bx) \mid \bgamma, \btheta) \right]
- 
\mathbb{E}_{p(\btheta, \bgamma)p(\bx \mid \btheta)}
\left[ \log q_\psi(T_\phi(\bx) \mid \bgamma, \btheta) \right] \;
\label{eqn:likelihood_bias_bound}
\end{equation}
Here, we introduced an auxiliary network $q_\psi(T_\phi(\bx) \mid \gamma, \theta)$ that approximates the conditional distribution of summaries.  Alternatively to CLUB, for instance adversarial estimates of the conditional mutual information are possible~\citep{xxx}.

For deterministic and continuous summaries, the term 
$\mathcal{I}(T_\phi(\bx);\bx)$ is actually formally ill-defined.\footnote{Under common 
regularization schemes, 
$\mathcal{I}(T_\phi(\bx);\bx)$ reduces to a differential entropy term 
$\mathcal{H}(T_\phi(\bx))$ plus a regularization-dependent constant.  
However, differential entropy is scale-dependent, and the penalty can always be reduced  simply by shrinking the scale of $T_\phi(\bx)$, rendering it unsuitable as a direct  optimization target.} 
A standard remedy—used, for example, in the variational information bottleneck (VIB)~\citep{xxx}—is  to inject a small, fixed amount of noise into the representation, which fixes the scale  and renders the mutual information well-defined. In practice on often uses structured proxies for compression, such as dimensionality reduction, sparsity  constraints, or gating mechanisms~\citep{xxx}.   These serve as practical surrogates for the idealized SIB objective.   

\paragraph{Variational Bound for Invariant Information Bottleneck}

We illustrate here, for a simple example, how the IIB and SIB objectives, Eqs.~\eqref{eqn:IIB_objective} and~\eqref{eqn:SIB_objective}, can be implemented in practice. Our aim is to demonstrate how type~A epistemic uncertainty---stemming from mismatches between the simulator and the true data-generating process---can be transformed into type~B uncertainty, arising from information loss. This trade-off enables more robust inference by reducing the influence of unreliable or ambiguous aspects of the simulator.

\paragraph{Variational Bounds for Mutual Information and Entropy}

In order estimate the conditional mutual information in Eq.~\eqref{eqn:IIB_objective} in a tractable way, we use an approximate variational upper bound inspired by the CLUB estimator~\citep{cheng_club_2020}
%
\begin{equation}
\mathcal{I}(T_\phi(\bx); \bgamma \mid \btheta) \lesssim
\mathbb{E}_{p(\bx, \btheta, \bgamma)} 
\left[ \log q_\phi(T_\phi(\bx) \mid \bgamma, \btheta) \right]
- 
\mathbb{E}_{p(\btheta, \bgamma)p(\bx \mid \btheta)}
\left[ \log q_\phi(T_\phi(\bx) \mid \bgamma, \btheta) \right] \;.
\label{eqn:likelihood_bias_bound}
\end{equation}
%
In practice, we can then minimise the upper bound as surrogate for minimising the mutual information directly. The second term is here estimated using mismatched pairs: summaries generated from samples \( \bx \sim p(\bx \mid \btheta) \), but evaluated under different configurations \( \bgamma \).  Note that the variational upper bound is only approximate (it becomes strict for exact $q_\phi$), but it features a useful structural property: the gradient with respect to \( T_\phi(\bx) \) vanishes when \( q_\phi \) becomes independent of \(\bgamma\). This independence is, in fact, the desired outcome.

In a similar way we obtain the upper bound
%
\begin{equation}
\mathcal{H}(T_\phi(\bx)) 
\equiv - \mathbb{E}_{p(\bx)} \left[ \log p(T_\phi(\bx)) \right]
\;\leq\;
- \mathbb{E}_{p(\bx, \bgamma, \btheta)} 
\left[ \log q_\phi(T_\phi(\bx) \mid \bgamma, \btheta) \right] \;,
\label{eqn:entropy_bound}
\end{equation}
%
which is exact in the limit where $q_\phi$ does not depend on $\bgamma$ and $\btheta$.
\cw{Maybe extend this with more relevant examples}


\paragraph{A Variational Training Objective for Robust Summaries}

We now define a training objective that integrates four components: (1) an amortized posterior term that encourages informativeness with respect to \(\btheta\), and that is conditioned on model configurations $\bgamma$, (2) an auxiliary density estimator for the summary conditioned on \((\bgamma, \btheta)\), (3) a robustness regularizer via the variational upper bound on \(\mathcal{I}(T_\phi(\bx); \bgamma \mid \btheta)\), and (4) a sparsity regularizer based on the entropy bound from Eq.~\eqref{eqn:entropy_bound}.

This leads to the following composite loss:
%
\begin{multline}
    \mathcal{L}[\phi_T, \phi_\theta, \phi_q] =
    \\[0.5em]
    -\underbrace{
    \mathbb{E}_{p(\bx, \btheta, \bgamma)}\left[\log 
    q_{\phi_\theta}(\btheta \mid T_{\phi_T}(\bx), \bgamma)\right]
    }_{\text{parameter inference \& informative summary}}
    -\underbrace{
    \mathbb{E}_{p(\bx, \btheta, \bgamma)} \left[\log q_{\phi_q}(
    T_{\bar{\phi}_T}(\bx) \mid \btheta, \bgamma)\right]
    }_{\text{auxiliary summary modeling}}
    \\[0.5em]
    + \lambda \cdot \underbrace{
    \left(
    \mathbb{E}_{p(\btheta, \bgamma)\,p(\bx \mid \btheta, \bgamma)} 
    \left[ \log q_{\bar{\phi}_q}(T_{\phi_T}(\bx) \mid \bgamma, \btheta) \right]
    -
    \mathbb{E}_{p(\btheta, \bgamma)\,p(\bx \mid \btheta)}
    \left[ \log q_{\bar{\phi}_q}(T_{\phi_T}(\bx) \mid \bgamma, \btheta) \right]
    \right)
    }_{\text{configuration invariance regularizer}}
    \\[0.5em]
    - \beta \cdot \underbrace{
    \mathbb{E}_{p(\bx, \btheta, \bgamma)} \left[ \log q_{\bar{\phi}_q}(
    T_{\phi_T}(\bx) \mid \bgamma, \btheta) \right]
    }_{\text{sparsity regularizer}}
\end{multline}
%
Here, \(\phi_T\) parametrizes the summary network \(T_{\phi_T}(\bx)\), \(\phi_\theta\) the posterior estimator, and \(\phi_q\) the auxiliary density model used to evaluate the entropy and mutual information bounds. When computing gradients, expressions involving barred parameters (e.g., \(\bar{\phi}_T\), \(\bar{\phi}_q\)) are treated as constants and do not contribute to backpropagation. This is typically implemented via explicit gradient detachment.

The hyperparameters $\lambda$ and $\beta$ control the relative influence of the robustness and sparsity regularizers, respectively, and therefore determine the balance between invariance and compression. When $\lambda>0$ and $\beta=0$, the objective prioritizes simulator invariance by suppressing configuration-dependent variation without restricting the overall capacity of the summary. Conversely, choosing $\lambda=0$ and $\beta>0$ enforces compression alone, limiting the information content of the summary and reducing sensitivity to unmodeled sources of mismodeling. When both coefficients are positive, with $\lambda>\beta>0$, the objective achieves a controlled compromise, encouraging representations that are simultaneously robust to simulator variation and sufficiently compressed to avoid encoding extraneous or unstable features.



%\section{Finite-data effects and variance analysis}

%\cw{TODOREF - Finite sample size}

\subsection{Hessian-based uncertainty analysis}

The convergence arguments in Sec.~\ref{sec:methods} relied on the large training data limit, $N \to \infty$, where sums over training data samples can are replaced by integrals over generative model parameters and observations (see for instance Eqs.~\eqref{eqn:NPE_loss} and~\eqref{eqn:NPE_limit}).  Furthermore, we assumed that networks are expressive enough to mathematically represent the minimum of the loss function. However, in many applications, one only has a limited amount of training data available, depending on simulation and storage costs.  Unfortunately, predicting the performance of models and algorithms in situations with limited training data and network capacity is generally challenging. Still, some basic insights can be obtained mathematically by considering the sample variance of the loss function.

\paragraph{General analysis.}
SBI training losses functions, like Eqs.~\eqref{eqn:NPE_loss} or \eqref{eqn:NRE_loss}\footnote{With an additional sum over $|\mathcal{D}_c|$.}, can be generally written in the form
%
\begin{equation}
\label{eqn:general_training_loss}
\mathcal{L}_{\mathcal D}[q] = \frac1{N} \sum_{\bx, \btheta \in \mathcal{D}} \ell_q(\bx, \btheta)\;.
\end{equation}
%
We made here explicit that the loss function depends, besides on the network weights $\phi$, also on the specific training data realization $\mathcal{D}$.\footnote{Note that we sum here over \emph{all} available training data, $\mathcal{D}$. The usual sub-sampling or mini-batching during training with stochastic gradient descent introduces additional noise. We are here, however, only interested in effects on the loss function after sub-sampling related variations are averaged out. 
This corresponds to the end-phase of training with small learning rate.}
%As discussed in Sec.~\ref{sec:methods}, 
Training data is generated as $N$ i.i.d.~samples from the generative model.
It is then straightforward to show that the mean of the loss function is given by 
$\bbE_{\mathcal D}[\mathcal L[\mathcal D, \bphi]] = \bbE_{p(\bx, \btheta)}[\ell_{\bphi}(\bx, \btheta)]$,
and that its variance is given by
$\text{Var}_{\mathcal D}[\mathcal L[\mathcal D, \bphi]] = \frac{1}{N}\text{Var}_{p(\bx, \btheta)}[\ell_{\bphi}(\bx, \btheta)]$. The variance scales like $1/N$ with training data size, as expected from the variance reduction properties of averaging.

To study the impact of the variance on the loss minimum, we expand the loss function at second order in the deviations $\bepsilon$, around its large-sample minimizer $\bphi$.
%
$$
q_{\balpha}(\btheta \mid \bx) = q_0(\btheta \mid \bx)
\cdot \exp\left(1+\sum_{i=1}^K \alpha_i \delta_i(\btheta, \bx) \right)
$$
We can then investivate the loss function value as function of $\balpha$
$$
\mathcal{L}_{\mathcal D}[\balpha] \equiv
\mathcal{L}_{\mathcal D}\left[q_{\balpha}(\btheta \mid \bx)\right]
$$
Second order Talor expansion leads to 
$$
\mathcal{L}_{\mathcal D}[\balpha]
\approx
\mathcal{L}_{\mathcal D}[0]
+ \balpha^T  
\underbrace{\left.\nabla_{\balpha} \mathcal{L}_{\mathcal{D}} \right|_{\balpha = 0}}_{\equiv \mathbf g}
+ \frac12 
\balpha^T  
\underbrace{\left.(\nabla^2_{\balpha} \mathcal{L}_{\mathcal D})\right|_{\balpha = 0}}_{\equiv \mathbf H}
\balpha
$$
where we introduce gradient and Hessian.

We can trivially minimize that function, for a given dataset $\mathcal{D}$, and find
$$
\balpha^\ast \equiv  \argmin_{\balpha} \mathcal{L}_{\mathcal D}[\balpha] =  - {\mathbf H}^{-1} \mathbf g
$$

To lowest order, we can then estimate the dataset related variance of the minimiser to be
$$
\text{Cov}_{\mathcal D}[\balpha^\ast] \approx
\bbE_{\mathcal D}[{\mathbf H}]^{-1}\cdot
\text{Cov}_\mathcal{D}[\mathbf g]
\cdot \bbE_\mathcal{D}[{\mathbf H}]^{-1}
$$

Any set of functions $\delta_i(\btheta, \bx)$ spans a linear space. The standard normal modes are defined to span the same space, but with a unit covariance matrix for $\text{Cov}_{\mathcal D}[\balpha^\ast] = \mathbb 1$.
Such normal modes can be called $\delta_i$.


\paragraph{Neural posterior estimation.} 
For NPE, we find
$$
\text{Cov}[\balpha] = \frac1N\int d\btheta\, d\bx\, p(\bx, \btheta)
\delta_i(\btheta, \bx)
\delta_j(\btheta, \bx)
%\frac{q_i(\btheta \mid \bx)}{q_0(\btheta \mid \bx)}
%\frac{q_j(\btheta \mid \bx)}{q_0(\btheta \mid \bx)}
$$

In order to proceed, we make a simple ansatz that the modes have the form. For any set of basis functions $e_i(\btheta, \bx)$, we can find that $q_i$ lead to a unit variate 
$$
\delta_i(\btheta , \bx) = 
\frac{e_i(\btheta, \bx)}{\sqrt{p(\bx, \btheta)}}
$$
for which the $\balpha^\ast$ covariance becomes the unit matrix.

We define state density
$$
\rho(\bx, \btheta) \equiv \sum_i e_i^2(\bx, \btheta)
= \frac{1}{\mathcal V_{\rm eff}(\bx, \btheta)}
$$
Then the deviation
$$
\delta q(\btheta \mid \bx) = \sum_{i=1}^K \alpha_i \delta_i(\btheta, \bx) q_0(\btheta \mid \bx)
$$
has a point-wise variance that we can estimate as
$$
\text{Cov}_\mathcal{D}
\left[\frac{\delta q(\btheta \mid \bx)}{q_0(\btheta \mid \bx)}\right]
=
\frac{\rho(\bx, \btheta)} {p(\bx, \btheta) N}
=
\frac{1} {p(\bx, \btheta) N \mathcal V_{eff}(\btheta, \bx)}
$$

Few relevant observations...

\paragraph{Neural ratio estimation.}
For NRE we find instead

$$
\text{Cov}_\mathcal{D}k\left[\frac{\delta q(\btheta \mid \bx)}{q_0(\btheta \mid \bx)}\right]
=
\frac{\rho(\bx, \btheta)} {\min(p(\bx, \btheta), p(\bx) p(\btheta)) N}
$$


\subsection{Noise resampling}



%For $M\to \infty$, second term vanishes. Only variations where 
%$\delta q(\btheta \mid \bx) \approx \delta h(\btheta) p(\btheta \mid \bx)$ is not varying much with $\bx$ matter.

%Each term in Eq.~\eqref{eqn:general_training_loss} has hence the variance $\text{Var}_{p(\bx, \btheta)}[\ell_\phi(\bx, \btheta)]$. 
%Given that training examples are drawn independently, standard variance scaling arguments yield then that the total variance of the training loss,
%
%\begin{equation}
%\text{Var}_{p(\mathcal{D})} [\mathcal{L}[\mathcal D, \phi]]
% = \frac1{|\mathcal D|} \text{Var}_{p(\bx, \btheta)} [\mathcal{\ell}_\phi(\bx, \btheta)]\;.
%\label{eqn:loss_var}
%\end{equation}
%\begin{equation}
%\text{Var}_{p(\mathcal{D})} [\nabla_\phi\mathcal{L}[\mathcal D, \phi]]
% = \frac1{|\mathcal D|} \text{Var}_{p(\bx, \btheta)} [\nabla_\phi\mathcal{\ell}_\phi(\bx, \btheta)]
%\label{eqn:loss_var}
%\end{equation}
%
%
%\begin{equation}
%\text{Var}_{p(q)p(\mathcal{D})} [\mathcal{L}[\mathcal D, \phi]]
%= 
%\bbE_{p(q)}\text{Var}_{p(\mathcal{D})} [\mathcal{L}[\mathcal D, \phi]]
%+ \text{Var}_{p(q)}\bbE_{p(\mathcal{D})} [\mathcal{L}[\mathcal D, \phi]]
%\end{equation}
%%
%As expected, the sample variance related to finite training data vanishes in the large sample limit, $|\mathcal{D}| \to \infty$.
%%
%In the subsequent discussion, and in the interest of clarity, we will consider the NPE loss, where $\ell(\bx, \btheta) = -\log q_\phi(\btheta \mid \bx)$. 

%However, analogous arguments can be made for all other SBI algorithms that we discussed in Sec.~\ref{sec:core}.

%For a single training sample, $N=1$, randomly drawn from $p(\bx, \btheta)$, the variance would be $\text{Var}_{p(\bx, \btheta)}[-\log q_\phi(\btheta \mid  \bx)]$ (it is $1/2$ in the Gaussian case).  For the $N$ i.i.d.~samples that contribute to $\mathcal{L}$, we instead obtain
%%
%$$
%\text{Var}_{p(\bx, \btheta)}[\mathcal{L}_\text{NPE}] = 
%\frac1N
%\text{Var}_{p(\bx, \btheta)}[
%%\mathcal{\ell}(\bx, \btheta)
%-\log q_\phi(\btheta \mid \bx)
%]
%\overset{N\to\infty}{\to}0
%\;.
%$$
%As expected, the sample variance of $\mathcal{L}_\text{NPE}$, and derived quantities like gradients for stochastic gradient descent, vanishes in the large $N$ limit.

Simulation models often have a modular or hierarchical form, with some computationally slow and some fast components. In these situations, SBI performance can be benefit from running the fast parts of the simulator more often than the slow parts when generating training data.

As an example, let us consider a simple hierarchical simulation model,
%
$$
p(\mathbf x \mid \btheta)
= \int d\blambda\; 
\underbrace{p(\mathbf x \mid \blambda)}_{\text{(a) Fast}}
\underbrace{p(\blambda \mid \btheta)}_\text{(b) Slow }\;.
$$
%
Here, $\btheta$ represents model parameters, $\blambda$ are (stochastic or deterministic) model predictions, and $\bx$ is simulated data.  Generating model predictions $\blambda \sim p(\blambda \mid \btheta)$ given model parameters $\btheta$ can be computationally very costly and slow, because it might involve running a physics simulation code. On the other hand, generating simulation data $\bx \sim p(\bx \mid \blambda)$ for a given model prediction $\blambda$ can be very fast, because it might just amounts to adding measurement noise that can be quickly sampled, $\bx = \blambda + \textbf n$.
It makes then sense to store $\blambda$ and re-sample $\bx$ on the fly during training.

%where some components are fast to calculate and sample (for instance, adding Gaussian measurement noise to a model prediction), and other components are slow (for instance, running a physics simulation code to obtain said model prediction). 

The empirical distribution of the finite training data has an atomic structure and is defined as a sum over delta functions,
$$
p_N(\bx, \btheta) = \frac1N\sum_{i=1}^N \delta(\bx - \bx_i) \delta(\btheta - \btheta_i)\;.
$$
The resampling distribution, on the other hand, has clouds of $\bx_{i, j} \sim p(\bx \mid \btheta_i)$ samples associated to each sample $\btheta_i \sim p(\btheta)$,
%$$
%p_r(\bx, \btheta) = \frac1N\sum_{i=1}^N 
%p(\bx \mid \blambda_i, \btheta_i) \delta(\btheta - \btheta_i)\;,
%$$
$$
p_r(\bx, \btheta) =
\frac1N\sum_{i=1}^N 
\left(\frac1M\sum_{j=1}^M
\delta(\bx - \bx_{i,j})\right)
\delta(\btheta - \btheta_i)
\;,
$$
where $N$ is the number of parameter samples, and $M$ the number of data samples per parameter sample.
This brings the empirical training data distribution much closer to the true generative model distribution $p(\bx, \btheta)$.
%which brings it closer to the true distribution $p(\bx, \btheta)$.
Examples for this are shown in Fig.~\ref{fig:resampling_examples}.

\medskip

One can use the law of total variance to show that for the above empirical resampling distribution, the variance of the gradient can be computed as
$$
\text{Cov}_{\mathcal D}[\mathbf g[\mathcal D]]
= 
\frac1N \left[
\text{Cov}_{p(\btheta)}
\bbE_{p(\bx \mid \btheta)}
\left[\nabla_{\bphi}\ell_\phi(\bx, \btheta)\right]
+\frac1M 
\bbE_{p(\btheta)}
\text{Cov}_{p(\bx \mid \btheta)}\left[\nabla_{\bphi}\ell_\phi(\bx, \btheta)\right]
\right]\;.
$$
In the case without resampling, $M=1$, we obtain the results quoted above, while in the limit of online resampling, $M\to \infty$, only the first term contributes to the gradient variance.

The following expression follows from a second-order perturbative expansion around the loss minimum, combined with a variance decomposition across hierarchical data structure.
$$
\text{Cov}_\mathcal{D}
\left[\frac{\delta q(\btheta \mid \bx)}{q_0(\btheta \mid \bx)}\right]
\approx
\frac1N \left(
\frac{\int d\bx \rho(\bx, \btheta)} {p(\btheta)}
+ \frac1M\frac{\rho(\bx, \btheta)} {p(\btheta , \bx)}
\right)
$$


In the case of NPE, the first term depends on the data average $\bbE_{p(\bx \mid \btheta)}[\delta q(\btheta \mid \bx)/p(\btheta \mid \bx)]$ w.r.t.~$\btheta$. 
%Fluctuations with a ratio $\delta q(\btheta \mid \bx)/p(\btheta \mid \bx)$ is approximately constant w.r.t.~$\bx$ are the most relevant, while the $\bx$ dependence of $\delta q(\btheta \mid \bx)$ is not affected by 
As a consequence, in the online resampling limit, $M\to \infty$,
uncertainties in the $\bx$ direction of $q_\phi(\btheta \mid \bx)$ expected to be heavily suppressed, and the fitted posterior is expected to be dominated by uncertainties of the form $q_\phi(\btheta \mid \bx) \approx (1+\delta(\btheta)) p(\btheta \mid \bx)$. Even for small relatively small $N$, this strongly constraints the optimisation problem.

%\medskip
%
%We can estimate the finite sample size-induced variance of the NPE loss function, $\mathcal{L}_\text{NPE} \equiv \mathbb E_{p(\bx, \btheta)}\left[-\log q_\phi(\btheta \mid \bx)\right]$, assuming that we have $N$ samples from $\blambda$ and $\btheta$, and $M$ samples of $\bx$ for each $\blambda$ and $\btheta$ pairs,
%%
%$$
%\text{Var}\left[\mathcal{L}_\text{NPE}\right]
%=
%\frac1N
%\left[
%\frac1M
%\mathbb E_{p(\btheta)}
%\text{Var}_{p(\bx \mid \btheta)}[-\log q_\phi(\btheta \mid \bx)]
%+
%\text{Var}_{p(\btheta)}
%\mathbb E_{p(\bx \mid \btheta)}
%[-\log q_\phi(\btheta \mid \bx)]
%\right]
%$$
%
%The first term corresponds to the variance induced by varying $\bx$ for a given $\btheta$, while the second term corresponds to the variance induced by different values of the loss function for across different $\btheta$.

\cw{FIGTODO Optional figures: Show uncertainties in training in the tails, and qualitatively compare with estimates}


%\section{Sequential methods and adaptive learning}
\label{sec:sequential_sbi}

\begin{quotation}
\textit{``When solving a problem of interest, do not solve a more general problem as an intermediate step. Try to get the answer that you really need but not a more general one.''}

\hfill --- \cite{vapnik_estimation_2006}
\end{quotation}

%\cw{TODOREF - Sequential SBI}

\subsection{Basic idea}

One characteristics of simulation-based inference methods is that they typically lead to \textit{amortization} of inference results: rather than providing the posterior approximation for a specific observation $\bxobs$ of interest, they provide an inference machine that learned how to map \textit{any} data on inference results.  In some cases, where one is just interested in a cost-efficient analysis of a single observation, that might be not desirable.  In fact, obtaining precision results in these settings can require immense amounts of training data and highly flexible network architectures.

The idea is sequential inference is to focus (sequentially) on the range of parameters $\bz$ that are relevant for a specific observation $\bxobs$. This is done by replacing during the generation of training data the actual prior distribution $p(\bz)$ with a proposal distribution $\tilde p(\bz)$ that is somehow more emphasizing the interesting parameter for a given observation.  A wide range of strategies for selecting appropriate proposal distributions, and for undoing the damage done by using the `wrong' distribution for the prior, have been proposed in the literature.  Essentially, for any of the algorithms discussed above, there is a sequential version.

We sample targeted training data
%
$$
\mathcal{D} = \{(\boldsymbol{\theta}_i, \mathbf{x}_i)\}_{i=1}^N
\;,\quad
\btheta, \bx \iidsim p(\bx \mid \btheta) \tilde p(\btheta)
$$
%
$$
\text{proposal} \quad \tilde p(\btheta) \quad 
\text{approximates target posterior}
\quad
\left. p(\btheta \mid \bx) \right|_{\bx = \bx_o}
$$
%
When learning the posterior from this training data, we find
$$
p(\btheta \mid \bx) \simeq 
\frac1{Z(\bx)}
q_\phi(\btheta \mid \bx)
\frac{p(\btheta)}{\tilde p(\btheta)}
%\frac{\tilde p(\bx)}{p(\bx)}
$$


\subsection{Strategies for sequential SBI}

\begin{figure}[h]
    \centering
    \includegraphics[width=0.5\linewidth]{figures/Sequential.png}
    \caption{Simple visualization of how focusing the prior parameter range increases the simulation density in the region with high likelihood.}
    \label{fig:sequential}
\end{figure}

when learning the likelihood function
$$
p(\btheta \mid \bx) \simeq 
\frac1{Z(\bx)}
q_\phi(\bx \mid \btheta)
p(\btheta)
%\frac{p(\btheta)}{p(\bx)}
%\frac{\tilde p(\bx)}{p(\bx)}
$$

When learning the ratio
$$
p(\btheta \mid \bx) \simeq 
%\frac1{Z(\bx)}
r_\phi(\bx; \btheta)
p(\btheta)
%\frac{p(\btheta)}{p(\bx)}
%\frac{\tilde p(\bx)}{p(\bx)}
$$

Common choices are posterior
$$
\tilde p(\btheta) = q_\phi(\btheta \mid \bx)
$$
or combination of posterior and prior
$$
\tilde p(\btheta) = \sqrt{
q_\phi(\btheta \mid \bx)
p(\btheta)
}
$$
or truncated prior
$$
\tilde p(\btheta) = \frac1Z 
p(\btheta) \mathbb{1}( q_\phi(\bx \mid \btheta) > \epsilon)
$$

Problem with marginalization: If we split parameters $\btheta = (\bphi, \blambda)$, then the learned marginal likelihood function becomes
$$
q_\phi(\bphi \mid \bx) 
\simeq \int d\blambda \;
p(\bx \mid \bphi, \blambda) 
\tilde p(\bphi \mid \blambda)
$$
and generally assume a proposal function $\tilde p(\btheta) = \tilde p(\bphi) p(\blambda \mid \bphi)$.



\paragraph{Sequential Neural Likelihood/Ratio Estimation (SNLE \& SNRE).} 
Sequential techniques are relatively easy to implement in situations where we approximate the data likelihood, $q_\phi(\bx \mid \bz) \approx p(\bx \mid \bz)$, with a neural network.  The reason is that the data likelihood does not formally depend on the prior distribution.  We expect in the end that
%
\begin{equation}
    q_\phi(\bx \mid \bz) = \tilde q_\phi(\bx \mid \bz)
    \quad \text{in high-probability regions of} \quad
    \tilde p(\bz)\;.
\end{equation}
%
Outside of the support of $\tilde p(\bz)$, the behaviour of $\tilde q_\phi(\bx \mid \bz)$ will be undefined.  By focusing training data on the parameters with a high data likelihood, one can effectively increase the density of training data in the region of interest.
%
Although the full model likelihood is prior independent, marginal likelihoods are generally not, and in general $q_\phi(\bx \mid z_i) \neq \tilde q_\phi(\bx \mid z_i)$.

A very similar behaviour can be observed for likelihood-to-evidence ratio estimation, where we train a network to approximate $\log f_\phi(\bx; \bz) \approx p(\bx | \bz)/p(\bx)$. Since the likelihood is independent of the proposal distribution, and only the parameter-independent evidence $p(\bx)$ is affected, we find that
%
\begin{equation}
    \log f_\phi(\bx; \bz) = \log \tilde f_\phi(\bx; \bz) + \text{const}
    \quad \text{in high-probability regions of} \quad
    \tilde p(\bz)\;.
\end{equation}
%
However, the same limitations apply in the case of training marginals.

\paragraph{Sequential Neural Posterior Estimation (SNPE).}

When performing neural posterior estimation as discussed above, the proposal distribution $\tilde p(\bz)$ affects the outcome.

Instead of using the prior $p(\mathbf z)$, it can be useful to use a proposal distribution $\tilde p(\mathbf z)$ that focuses on likely regions of the data given a specific observation $\mathbf x_o$.  Changing the prior has an effects on the posterior, which can be undone by multiplying the variational posterior with the prior-to-proposal ratio,
%
\begin{equation}
    q_\phi(\mathbf z \mid \bxobs)
    = \frac1Z
    \tilde q_\phi(\mathbf z \mid \bxobs)
    \frac{p(\mathbf z)}{\tilde p(\mathbf z)}
    \quad \text{in high-probability regions of} \quad
    \tilde p(\bz)\;.
\end{equation}
%
The partition function, $Z$, has formally the value $\frac{p(\mathbf x_o)}{\tilde p(\mathbf x_o)}$, which is usually unknown.  MCMC type techniques are typically used to sample from the posterior distribution in that case.

we see that the effect can be corrected for by multipling the inferred posterior by the factor $p(\mathbf z)/\tilde p(\mathbf z)$.  In general the evidence ratio will not be known.  This works well as long as the correction factor $p(\mathbf z) / \tilde p(\mathbf z)$ remains small over the range of the posterior.

Like for NLE above, it is relevant to observe that correction is not possible anymore if instead marginal posteriors are generated. In general, we cannot reconstruct  $q_\phi (z_i \mid \bxobs)$ from  $\tilde q_\phi (z_i \mid \bxobs)$, even if the prior functions and the proposal distribution is tractable and known.

\paragraph{Posterior approximation.}

A number of $R$, rounds with $\tilde p_1(\mathbf z) = p(\mathbf z)$, and $\tilde p_{i}(\mathbf z) = q_{\phi, i-1}(\mathbf z \mid \mathbf x_o)$ for $i = 2, \dots, R$.

\paragraph{Prior truncation strategies.}

We can see from Eq.~\eqref{eqn:SNPE}, that in cases where $p(\bz)/\tilde p(\bz)$ is constant over the high-probability region of $\tilde q_\phi(\bz \mid \bxobs)$, the expected correction is expected to cancels exactly with $Z$.  Sampling from $\tilde q_\phi(\bx \mid \bxobs)$, which can be done efficiently for instance when $\tilde q_\phi$ was modeled as a normalizing flow, will directly generate samples from the learned posterior $p(\bz \mid \bxobs)$.

This can be also seen at the level of marginal likelihoods.  Let us consider.
%
\begin{equation}
    \tilde p(\bx \mid z_i) = \int d\bz_{-i}\; p(\bx \mid \bz) \tilde p(\bz_{-i})\;.
\end{equation}
%
If the proposal distribution $\tilde p(\bz) \propto p(\bz)$ in regions where the likelihood $p(\bx \mid \bz)$ contributes significantly to the integration, we find that $\tilde p(\bx \mid z_i) \propto p(\bx \mid z_i)$.  Two options to achieve this is to
%
\begin{equation}
    \tilde p(\bz) = \frac1Z \mathbb{1}( \bz \in \Gamma) p(\bz)
\end{equation}
Here, the high-likelihood region $\Gamma \subset\Omega$ can be for instance selected as
%
\begin{equation}
    \Gamma = \{ \bz \in \Omega \mid q_\phi(\bx | \bz) > \epsilon\}\;,
\end{equation}
%
with some suitably small parameter for $\epsilon$.  Another option is based on tempered likelihood functions,
%
\begin{equation}
    \tilde p(\bz) = \frac1Z p(\bx \mid \bz)^\gamma p(\bz)\;.
\end{equation}
%
where $0< \gamma < 1$ is the tempering factor.

\medskip

Sequential SBI techniques are generally more simulation efficient for a given observation of interest, $\bxobs$, but require a retraining of notworks for each new observation.

Same initialization, but then $p_{i}(\mathbf z) = \frac1Z \mathbb{I}(\mathbf z \in \Gamma_i) p(\mathbf z)$.  Here, $\Gamma_i$ is derived from the posterior or likelihood of the preceding round.




\subsection{Practical examples}

\cw{FIGTODO Quality of posterior estimation in multiple rounds}

- Learning $10^{-10}$ posterior width in multiple rounds, simple summary network

- Handling of training data and network-reinitialization between rounds


